%Initially by AJR, Apr 2017 -- Mar 2019
%!TEX root = eqnFreeDevMan.tex
\chapter{Create, document and test algorithms}
\label{sec:contribute}
\secttoc
\def\LaTeX{LaTeX}% for some unknown reason we need this!! 2018-12-22

For developers to create and document the various functions, we use an idea due to Neil~D. Lawrence of the University of Sheffield.

\begin{itemize}
\item Each class of toolbox functions is located in separate folders in the repository, say~\verb|Dir|.

\item Create a \LaTeX\ file~\verb|Dir/funs.tex|: establish as one \LaTeX\ chapter that \verb|\input{Dir/*.m}|s the  files of the functions in the class, example scripts of use, and possibly test scripts, \autoref{tbl:funtex}.

\item Each such \verb|Dir/funs.tex| file is to be included from the main \LaTeX\ file \verb|Doc/docBody.tex| so that people can most easily work on one chapter at a time: 
\begin{itemize}
\item put \verb|\include{funs}| into \verb|Doc/docBody.tex|\,;
\item to include, create a `link' file \verb|Doc/funs.tex| whose only active content is the command \verb|\input{../Dir/funs.tex}|\,;
\item in \verb|Doc/docBody.tex| modify the \verb|\graphicspath| command to include \verb|{../Dir/Figs}|\,.
\end{itemize}

\item Each toolbox function is documented as a separate section, within its chapter, with tests and examples as separate sections.

\item Each function-section and test-section is to be created as a \script\ \verb|Dir/*.m| file, say \verb|Dir/fun1.m|, so that users simply invoke the function in \script\ as usual by \verb|fun1(...)|.

Some editors may need to be told that \verb|fun1.m| is a \LaTeX\ file.  For example, TexShop on the Mac requires one to execute in a Terminal
\begin{verbatim}
defaults write TeXShop OtherTeXExtensions -array-add "m"
\end{verbatim}

\item \autoref{tbl:format} gives the template for the \verb|Dir/*.m| function-sections.
The format for a example\slash test-section is similar.

\item Any figures from examples should be generated and then saved for later inclusion with the following (which finally works properly for \textsc{Matlab} 2017+)
\begin{verbatim}
set(gcf,'PaperPosition',[0 0 14 10]);% cm
print('-depsc2','filename')
\end{verbatim}
If it is a suitable replacement for an existing graphic, then move it into the \verb|Dir/Figs| folder.
Include such a graphic into the \LaTeX\ document with (do \emph{not} postfix with \verb|.eps| or \verb|.pdf|)
\begin{verbatim}
\includegraphics[scale=0.9]{filename}
\end{verbatim}

\item   In figures and other graphics, do \emph{not} resize\slash scale fixed width constructs: instead use \verb|\linewidth| to configure large-scale layout, \verb|em| for small-widths, and \verb|ex| for small-heights. 


\item For every function, generally include at the start of the function a simple example of its use.  The example is only to be executed when the function is invoked with no input arguments (\verb|if nargin==0|).

When appropriate, if a function is invoked with no output arguments (\verb|if nargout==0|), then draw some reasonable graph of the results.


\item In all \script\ code, prefer camal case for variable names (not underscores).

\item When a function is `finalised', wrap (most) of the lines to be no more than 60~characters so that readers looking at the source can read the plain text reasonably.


\item In the documentation \cite[e.g.,][Ch.~4]{Higham98}: 
write actively, not passively (e.g., avoid ``--tion'' words);
avoid wishy-washy ``can'';
use the present tense;
cross-reference precisely;
and so on.

\end{itemize}


\begin{table}
\caption{\label{tbl:funtex}example \texttt{Dir/*.tex} file to typeset in the master document a function-section, say \texttt{fun.m}, and maybe the test\slash example-sections.}
\VerbatimInput[numbers=left]{../chapterTemplate.tex}
\end{table}
\begin{table}
\caption{\label{tbl:format}template for a function-section \texttt{Dir/*.m} file.}
\VerbatimInput[numbers=left]{../functionTemplate.m}
\end{table}


