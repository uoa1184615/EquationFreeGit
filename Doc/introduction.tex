%Initially by AJR, Apr 2017 -- Apr 2019
%!TEX root = eqnFreeDevMan.tex
\chapter{Introduction}
%\localtableofcontents

\begin{devMan}
This Developers Manual contains complete descriptions of the code in each function in the toolbox, and of each example.  For concise descriptions of each function, quick start guides, and some basic examples, see the User Manual.
\end{devMan}


\paragraph{Users}
Download via \url{https://github.com/uoa1184615/EquationFreeGit}.
Place the folders \verb|Patch| and \verb|ProjInt| of this toolbox in a path searched by \script\ (but \emph{not} the folder \verb|SandpitPlay| as that contains experimental development functions).
Then read the section(s) that documents the function of interest.


\paragraph{Quick start}
Maybe start by adapting one of the included examples. 
Many of the main functions include, at their beginning, example code of their use---code which is executed when the function is invoked without any arguments.
\begin{itemize}
\item To projectively integrate over time a multiscale, slow-fast, system of \ode{}s you could use \verb|PIRK2()|, or \verb|PIRK4()| for higher-order accuracy: adapt the Michaelis--Menten example at the beginning of \verb|PIRK2.m| (\cref{sec:pirk2eg}).
\item You may use forward bursts of simulation in order to simulate the slow dynamics backward in time, as in \verb|egPIMM.m| (\cref{sec:egPIMM}).
\item To only resolve the slow dynamics in the projective integration, use lifting and restriction functions by adapting the singular perturbation \ode\ example at the beginning of \verb|PIG.m| (\cref{sec:pigeg}).

\paragraph{Space-time systems}
Consider an evolving system over a large spatial domain when all you have is a microscale code.  
To efficiently simulate over the large domain, one can simulate in just small patches of the domain, appropriately coupled.
\item In 1D space adapt the code at the beginning of \verb|configPatches1.m| for Burgers' \pde\ (\cref{sec:configPatches1eg})%
\ifcsname r@sec:wave2D\endcsname, or the staggered patches of 1D water wave equations in \verb|waterWaveExample.m| (\cref{sec:waterWaveExample})\fi.

\item In 2D space adapt the code at the beginning of \verb|configPatches2.m| for nonlinear diffusion (\cref{sec:configPatches2eg})%
\ifcsname r@sec:wave2D\endcsname, or the regular patches of the 2D wave \pde\ of \verb|wave2D.m| (\cref{sec:wave2D})\fi.

\item In 3D space adapt the code at the beginning of \verb|configPatches3.m| for wave propagation through a heterogeneous medium (\cref{sec:configPatches3eg})%
\ifcsname r@sec:homoDiffEdgy3\endcsname, or the patches of the 3D heterogeneous diffusion of \verb|homoDiffEdgy3.m| (\cref{sec:homoDiffEdgy3})\fi.

\item Other provided examples include cases of macroscale \emph{computational homogenisation} of microscale heterogeneity.

\item In \textsc{Matlab}, the axis labelling works best if one executes
\\\verb|set(groot,'defaultTextInterpreter','latex')|
\end{itemize}


\paragraph{Verification}
Most of these schemes have analytically proven `accuracy' when compared to the underlying specified microscale system.
In the spatial patch schemes, we measure `accuracy' by the order of consistency between macroscale dynamics and the given specified microscale.  
\begin{itemize}
\item \cite{Roberts06d} and \cite{Roberts2011a} proved reasonably general high-order consistency for the 1D and 2D patch schemes, respectively.
\item In wave-like systems, \cite{Cao2014a} established high-order consistency for the 1D staggered patch scheme, and \cite{Divahar2022b, Bunder2019d} established excellent 2D staggered patch schemes for waves.
\item A heterogeneous microscale is more difficult, but \cite{Bunder2020a, Bunder2022b} developed a new `edgy' inter-patch interpolation that is analytically proven to be excellent for simulating the macroscale homogenised dynamics of microscale heterogeneous systems in multiple space dimensions---now coded in the toolbox.
\end{itemize}



\paragraph{Blackbox scenarios} 
Suppose that you have a \emph{detailed and trustworthy} computational simulation of some problem of interest.
Let's say the simulation is coded in terms of detailed (microscale) variable values~\(\uv(t)\), in~\(\RR^{p}\) for some number~\(p\) of field variables, and evolving in time~\(t\).
The details~\uv\ could represent particles, agents, or states of a system.
When the computation is too time consuming to simulate all the times of interest, then Projective Integration may be able to predict long-time dynamics, both forward and backward in time.  
In this case, provide your detailed computational simulation as a `black box' to the Projective Integration functions of \cref{sec:ProjInt}.

In many scenarios, the problem of interest involves space or a `spatial' lattice.
Let's say that indices~\(i\) correspond to `spatial' coordinates~\(\xv_i(t)\), which are often fixed: in lattice problems the positions~\(\xv_i\) would be fixed in time (unless employing a moving mesh on the microscale); however, in particle problems the positions would evolve.
And suppose your detailed and trustworthy simulation is coded also in terms of micro-field variable values~\(\uv_i(t)\in\RR^p\) at time~\(t\).
Often the detailed computational simulation is too expensive over all the desired spatial domain \(\xv\in\XX\subset\RR^{d}\).
In this case, the toolbox functions of \cref{sec:patch} empower you to simulate on only small, well-separated, patches of space by appropriately coupling between patches your simulation code, as a `black box', executing on each small patch. 
The computational savings may be enormous, especially if combined with projective integration.

\cref{sec:parallel} provides small examples of how to parallelise the patch computations over multiple processors.  
But such parallelisation may be only useful for scenarios where the microscale code has many millions of operations per time-step.



\paragraph{Contributors}
The aim of this project is to collectively develop a powerful and flexible \script\ toolbox of equation-free algorithms.
Initially the algorithms are basic, and the ongoing program is developing more and more capability.

\Matlab\ appears a good choice for a first version since it is widespread, efficient, supports various parallel modes, and development costs are reasonably low.
Further it is built on \textsc{blas} and \textsc{lapack} so the cache and superscalar \cpu{} are well utilised.
We aim to develop functions that work for~\script.
\ifcsname r@sec:contribute\endcsname\cref{sec:contribute} outlines some details for contributors.\fi



