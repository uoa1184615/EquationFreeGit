%Initially by AJR, Apr 2017 -- Nov 2018
%!TEX root = eqnFreeDevMan.tex
\chapter{Introduction}
%\localtableofcontents

\begin{devMan}
This Developers Manual contains complete descriptions of the code in each function in the toolbox, and each example.  For concise descriptions of each function, quick start guides, and some basic examples, see the User Manual.
\end{devMan}


\paragraph{Users}
Place the folder of this toolbox in a path searched by \script.
Then read the section(s) that documents the function of interest.


\paragraph{Quick start}
Maybe start by adapting one of the included examples. Many of the main functions include, at their start, example code of their use (code which is executed if the function is invoked without any arguments).
\begin{itemize}
\item To projectively integrate over time a multiscale, slow-fast, system of \ode{}s you could use \verb|PIRK2()|: adapt the Michaelis--Menten example at the start of \verb|PIRK2.m| (\cref{sec:pirk2eg}).
\item You may use forward bursts of simulation in order to simulate the slow dynamics backward in time, as in \verb|egPIMM.m| (\cref{sec:egPIMM}).
\item To only resolve the slow dynamics in the projective integration, use lifting and restriction functions by adapting the singular perturbation \ode\ example at the start of \verb|PIG.m| (\cref{sec:pigeg}).
\item Consider an evolving system over a large spatial domains when all you have is a microscale code.  To efficiently simulate over the large domain, one can simulate in just small patches of the domain, appropriately coupled:
\begin{itemize}
\item in 1D adapt the code at the start of \verb|configPatches1.m| for Burgers' \pde\ (\cref{sec:configPatches1eg})%
\ifcsname r@sec:wave2D\endcsname, or the staggered patches of 1D water wave equations in \verb|waterWaveExample.m| (\cref{sec:waterWaveExample})\fi;
\item in 2D adapt the code at the start of \verb|configPatches2.m| for nonlinear diffusion (\cref{sec:configPatches2eg})%
\ifcsname r@sec:wave2D\endcsname, or the regular patches of the 2D wave equation of \verb|wave2D.m| (\cref{sec:wave2D})\fi.
\end{itemize}
\item The above are for systems that have \emph{smooth} spatial structures on the microscale: when the microscale is `rough' with a known period (so far only in 1D), then adapt 
the example of \verb|HomogenisationExample.m| (\cref{sec:HomogenisationExample}).
\end{itemize}


\paragraph{Blackbox scenarios} 
Suppose that you have a \emph{detailed and trustworthy} computational simulation of some problem of interest.
Let's say the simulation is coded in terms of detailed (microscale) variable values~\(\uv(t)\), in~\(\RR^{p}\) for any \(p=1,2,\ldots,\infty\), and evolving time~\(t\).
The details~\uv\ could represent particles, agents, states of a system.
When the computation is too time consuming to simulate all the times of interest, then Projective Integration may be able to predict long-time dynamics.  
In this case, provide your detailed computational simulation as a `black box' to the Projective Integration functions of \cref{sec:ProjInt}.

In many scenarios, the problem of interest involves space or a `spatial' lattice.
Let's say that indices~\(i\) correspond to `spatial' coordinates~\(\xv_i(t)\), which are often fixed: in lattice problems the positions~\(\xv_i\) would be fixed in time (unless employing a moving mesh on the microscale); in particle problems the positions would evolve.
And suppose your detailed and trustworthy simulation is coded in terms of micro-field variable values~\(\uv_i(t)\in\RR^p\) at time~\(t\).
Often the detailed computational simulation is too expensive over all the desired spatial domain \(\xv\in\XX\subset\RR^{d}\).
In this case, the toolbox functions of \cref{sec:patch} empower you to simulate on only small, well-separated, patches of space by appropriately coupling between patches your simulation code, as a `black box', executing on each patch. 
The computational savings may be enormous, especially if combined with projective integration.



\paragraph{Contributors}
The aim of this project is to collectively develop a \script\ toolbox of equation-free algorithms.
Initially the algorithms are basic, and the plan is to subsequently develop more and more capability.

\Matlab\ appears a good choice for a first version since it is widespread, efficient, supports various parallel modes, and development costs are reasonably low.
Further it is built on \textsc{blas} and \textsc{lapack} so the cache and superscalar \cpu{} are potentially well utilised.
We aim to develop functions that work for~\script.
\ifcsname r@sec:contribute\endcsname\cref{sec:contribute} outlines some details for contributors.\fi






%\chapter{Overview of major functions and example scripts}
%\label{sec:smf}
%\localtableofcontents
%
%{% excluding the body gives overview of each function/script
%    \renewcommand{\label}[1]{}%
%    \let\section\section%
%    \let\section\subsection%
%    \let\subsection\paragraph%
%    \let\paragraph\subparagraph%
%    \fancyvrbStartStop%
%    \excludeversion{devMan}
%    % input *.m files for Projective Integration  AJR, Oct 2017
%!TEX root = ../Doc/eqnFreeDevMan.tex
\chapter{Projective integration of deterministic ODEs}
\label{sec:ProjInt}
\localtableofcontents


\section{Introduction}

This section provides some good projective integration functions \cite[e.g.]{Gear02b, Gear03c, Givon06, Maclean2015, Sieber2018}.
The goal is to enable computationally expensive multiscale dynamic simulations\slash integrations to efficiently compute over very long time~scales.

\paragraph{Quick start} 
\cref{sec:pirk2eg} shows the most basic use of a projective integration function.
\cref{sec:egPIMM} shows how to code more variations of the introductory example of a long time simulation of the Michaelis--Menton multiscale system of differential equations.
Then see \cref{fig:constructPI,fig:PIchoosemacro}

\paragraph{Scenario}
When you are interested in a complex system with many interacting parts or agents, you usually are primarily interested in the self-organised emergent macroscale characteristics.
Projective integration empowers us to efficiently simulate such long-time emergent dynamics.
We suppose you have coded some accurate, fine scale simulation of the complex system, and call such code a microsolver.

The Projective Integration section of this toolbox consists of several functions.
Each function implements over a long-time scale a variant of a standard numerical method to simulate\slash integrate the emergent dynamics of the complex system.
Each function has standardised inputs and outputs.


\begin{figure}
\caption{\label{fig:constructPI}The Projective Integration method greatly accelerates simulation\slash integration of a system exhibiting multiple time scales.
The Projective Integration \cref{sec:ProjInt} presents several separate functions, as well as several optional wrapper functions that may be invoked. 
This chart overviews constructing a Projective Integration simulation, whereas \cref{fig:PIchoosemacro} roughly guides which top-level Projective Integration functions should be used.
\cref{sec:ProjInt} fully details each function.}
\centering
\setlength{\WD}{0.05\linewidth}%%%%%%%%%%%%%%%%
\begin{tikzpicture}[node distance = 3ex, auto]
\tikzstyle{bigblock} = [rectangle, draw, thick,   text width=20.5\WD, text badly centered,
    rounded corners, minimum height=4ex]
\tikzstyle{block} = [rectangle, draw=red!80!black, thick, anchor=west, fill=white,
    text width=9.6\WD, text ragged, rounded corners, minimum height=8ex]
 \tikzstyle{smallblock} = [rectangle, draw=red!80!black, thick, anchor=west, fill=white,
    text width=6\WD, text ragged, rounded corners, minimum height=8ex]   
 \tikzstyle{tinyblock} = [rectangle, draw=red!80!black, thick, anchor=west, fill=white,
    text width=4.3\WD, text ragged, rounded corners, minimum height=8ex]    
     \tikzstyle{smallenclose} = [rectangle, draw=red!80!black, thick, anchor=west, fill=white,
    text width=6.2\WD, text ragged, rounded corners, minimum height=8ex]   
     \tikzstyle{refblock} = [rectangle, draw=blue!80!black, thick, anchor=west, fill=blue!5,
    text width=1\WD, rounded corners, minimum height=1.05em] 
\tikzstyle{line} = [draw, -latex']
\tikzstyle{lined} = [draw, latex'-latex']
\node [bigblock,draw=red!80!black,fill=red!10] (gaptooth) {\textbf{Schematic for Projective Integration scheme}

    \begin{tikzpicture}[node distance = 3ex, auto]
    \node [block] (setmicro) {\textbf{Set microsolver}
     
    Define or construct the function \texttt{solver()} that calls a black-box microsolver. Set \texttt{bT}, the time to run microsolver for. Possible aids:
\begin{itemize}
\item     Use the Patch functions (\cref{fig:constructpatch}) to simulate a large-scale \pde, lattice, etc.
%\item    Use \texttt{bbgen()} if the microsolver is a standard solver, \texttt{ode45} e.g., and needs to be converted into a black-box.
\item    Use \texttt{cmdc()} as a wrapper for the microsolver if the slow variables would otherwise change significantly over the microsolver.
\end{itemize}};

    \node[block, right=2ex of setmicro] (setmacro){\textbf{Set macrosolver, define problem}\\[1ex]
        \begin{tikzpicture}
    \node [tinyblock]%, below=0.6cm of pig]
     (pirk) {\textbf{If using \texttt{PIRKn()}:}
      
    Set the vector of output times \texttt{tspan}. Intervals between times are the projective time-steps. Set initial values \texttt{x0}.};
   \node [tinyblock, right=1ex of pirk]%, below right=0.2cm and -1cm of lift]
    (pig) {\textbf{If using \texttt{PIG()}:}
     
    Set the solver \texttt{macro.solver} to be used on the macro scale. Set any needed time inputs or time-step data in \texttt{macro.tspan}. Set initial values \texttt{x0}.};
    \end{tikzpicture}
    };
   
       \node [smallblock, below=2ex of setmacro] (lift) 
       {\textbf{Set lifting\slash restriction}\quad
    If needed, set functions \texttt{restrict()} and \texttt{lift()} to convert between macro and micro problems\slash variables. These are optional arguments to the Projective Integration functions.};
    
 \node [block, below right=5ex and -9\WD of setmicro] (dopi) {\textbf{Do Projective Integration}\quad 
    Invoke the appropriate Projective Integration function as, e.g., 
    \verb|[t,x]=PIRK2(solver,tspan,x0,bT)|, or
    \verb|[t,x]=PIG(solver,macro,x0)|. 
    Additional optional outputs inform you of the microscale.};
             \path [line, thick] (setmicro) to[out=-90,in=120] (dopi);
         \path [line, thick] (lift) to[out=180,in=3] (dopi);
         \path [line, thick] (setmacro) to[out=-150,in=30] (dopi);
%          \path [line, thick] (pig) to[out=180,in=0] (dopi);
%          \path [line, thick] (pirk) to[out=180,in=0] (dopi);
%          \draw [dashed, very thick] (pig) -- (pirk);
    \end{tikzpicture}
    };   
\end{tikzpicture}
\end{figure}










\begin{figure}
\centering
\caption{\label{fig:PIchoosemacro}The Projective Integration method greatly accelerates simulation\slash integration of a system exhibiting multiple time scales.
In conjunction with \cref{fig:constructPI}, this chart roughly guides which top-level Projective Integration functions should be used.
\cref{sec:ProjInt} fully details each function.}
\setlength{\WD}{0.081\linewidth}%%%%%%%%%%%%%%%%
\begin{tikzpicture}[node distance = 0.5cm, auto]
\tikzstyle{bigblock} = [rectangle, draw, thick,   text badly centered, 
    text width=12\WD, rounded corners, minimum height=2em]
\tikzstyle{block} = [rectangle, draw=red!80!black, thick, anchor=west, fill=white,
    text width=5\WD, rounded corners, minimum height=4em, text ragged]
 \tikzstyle{smallblock} = [rectangle, draw=red!80!black, thick, anchor=west, fill=white,
    text width=3.4\WD, rounded corners, minimum height=4em, text ragged]  
 \tikzstyle{yesblock} = [rectangle, draw=red!80!black, thick, anchor=west, fill=white,
    text width=1.2\WD, rounded corners, minimum height=1.2em]   
\tikzstyle{line} = [draw, -latex']
\tikzstyle{lined} = [draw, latex'-latex']
\node [bigblock,draw=red!80!black,fill=red!10] (gaptooth) {\textbf{Choosing the macro solver in Projective Integration}\\[2ex]
    \begin{tikzpicture}[node distance = 4ex, auto]
    \node [block,right=5\WD] (timestep) {{Is an appropriate time-step known for the slow dynamics?}};
    \node [yesblock, below =2ex of timestep] (timeyes) {Yes};
    \node [yesblock, right=2ex of timestep] (timeno) {No};
    \node [block, below=2ex of timeyes] (slowsol) {{Is a specific solver desired to simulate the slow dynamics?}};
     \node [smallblock, below right=2ex and 2ex of timeno] (pig) {Choose \texttt{PIG()} to simulate\slash integrate};
         \node [yesblock, below =2ex of slowsol] (slowno) {No};
    \node [yesblock,  right=2ex of slowsol] (slowyes) {Yes};
         \node [block, below =2ex of slowno] (pirk) {Choose \texttt{PIRK2()} or \texttt{PIRK4()} to  simulate\slash integrate};
         \path [line, thick] (timestep) to[out=-90,in=90] (timeyes);
         \path [line, thick] (timestep) to[out=0,in=180] (timeno);
          \path [line, thick] (timeno) to[out=0,in=90] (pig);
          \path [line, thick] (timeyes) to[out=-90,in=90] (slowsol);
          \path [line, thick] (slowsol) to[out=0,in=180] (slowyes);
          \path [line, thick] (slowsol) to[out=-90,in=90] (slowno);
          \path [line, thick] (slowno) to[out=-90,in=90] (pirk);
          \path [line, thick] (slowyes) to[out=0,in=-90] (pig);
    \end{tikzpicture}
    };   
\end{tikzpicture}
\end{figure}





\paragraph{Main functions}
\begin{itemize}
\item Projective Integration by second or fourth-order Runge--Kutta, \verb|PIRK2()| and \verb|PIRK4()| respectively.
These schemes are suitable for precise simulation of the slow dynamics, provided the time period spanned by an application of the microsolver is not too large.

\item Projective Integration with a General solver, \verb|PIG()|.
This function enables a Projective Integration implementation of any solver with macroscale time-steps.
It does not matter whether the solver is a standard Matlab algorithm, or one supplied by the user.
As explored in later examples, \verb|PIG()| should only be used in very stiff systems. 

\item `Constraint-defined manifold computing', \verb|cdmc()|.
This helper function, based on the method introduced in \cite{Gear04},  iteratively applies the microsolver and projects the output backwards in time.
The result is to constrain the fast variables close to the slow manifold, without advancing the current time by the duration of an application of the microsolver.
This function can be used to reduce errors related to the simulation length of the microsolver in either the \verb|PIRK| or \verb|PIG| functions.
In particular, it enables \verb|PIG()| to be used on problems that are not particularly stiff.
%\item Black box microsolver generator, \verb|bbgen()|.
%This simple function takes as input a standard solver with a recommended time-step for microscale simulation, and returns a `black-box' microsolver for the Projective Integration functions.
\end{itemize}

The above functions share dependence on a user-specified `microsolver', that accurately simulates some problem of interest. 


The following sections describe the \verb|PIRK2()| and \verb|PIG()| functions in detail, providing an example for each.
Then \verb|PIRK4()| is very similar to \verb|PIRK2()|.
Descriptions for the minor functions follow, and an example of the use of~\verb|cdmc()|.

\input{../ProjInt/PIRK2.m}
\input{../ProjInt/egPIMM.m}
\input{../ProjInt/PIG.m}
\input{../ProjInt/PIRK4.m}
\input{../ProjInt/cdmc.m}

\begin{devMan}
%\input{../ProjInt/bbgen.m}
\input{../ProjInt/PIRK_Example.m}
\input{../ProjInt/PIGExample.m}
\input{../ProjInt/PIGExplore.m}



\section{To do/discuss}
\begin{itemize}
\item Can we implement for Octave?  We would like to use nested functions for some examples, because the function code then inherits parameter(s) from the main function.  However, in Octave we cannot then use handles to these nested functions due to the error ``handles to nested functions are not yet supported"---which apparently is not going to be fixed anytime soon (as at March 2019).

\item could implement Projective Integration by `arbitrary' Runge--Kutta scheme; that is, by having the user input a particular Butcher table---surely only specialists would be interested.

\item can `reverse' the order of projection and microsolver applications with a little fiddling.
Then output at each user-requested coarse time is the end point of an application of the microsolver---better predictions for fast variables.

\item Can maybe implement microsolvers that terminate a burst when the fast dynamics have settled using, for example, the 'Events' function handle in ode23. 

\item Need projective integration of systems with fast oscillations, perhaps by DMD.

\item Need projective integration for stochastic systems.

\end{itemize}
\end{devMan}



%    % input *.m files for the Patch scheme in 1D and 2D. AJR,
% Nov 2017 -- Feb 2019
%!TEX root = ../Doc/eqnFreeDevMan.tex
\chapter{Patch scheme for given microscale discrete space system}
\label{sec:patch}
\localtableofcontents

\section{Introduction}


The patch scheme applies to spatio-temporal systems where
the spatial domain is larger than what can be computed in
reasonable time in a given complicated microscale code.  In
the scheme we compute the microscale details only on small
patches of the space-time domain, and produce correct
macroscale predictions by craftily coupling the patches
across unsimulated space \cite[e.g.]{Hyman2005, Samaey03b,
Samaey04, Roberts06d, Liu2015}.  The resulting macroscale
predictions were generally proved to be consistent with the
microscale dynamics, to some specified order of accuracy, in
a series of papers: 1D-space dissipative systems
\cite[]{Roberts06d, Bunder2013b}; 2D-space dissipative
systems \cite[]{Roberts2011a}; and 1D-space wave-like
systems \cite[]{Cao2014a}.

The microscale spatial structure is to be on a lattice such
as obtained from finite difference approximation of a \pde.
Usually continuous in time.

\paragraph{Quick start}
See \cref{sec:configPatches1eg,sec:configPatches2eg} which
respectively list example basic code that uses the provided 
functions to simulate the 1D Burgers'~\pde, and a 2D 
nonlinear `diffusion'~\pde.  Then see \cref{fig:constructpatch}.

\begin{figure}
\begin{maxipage}
\setlength{\WD}{0.048\linewidth}%%%%%%%%%%%%%%%%
\centering
\caption{\label{fig:constructpatch}The Patch methods, \cref{sec:patch}, accelerate simulation\slash integration of multiuscale systems with interesting spatial (or network) structure\slash patterns. The patch methods use your given microsimulators whether coded from \textsc{pde}s, lattice systems, or agent\slash particle microscale simulators.
The patch functions require that a user configure the patches, and interface the coupled patches with a time integrator\slash simulator. 
This chart overviews the main functions involved and their interrelationships.
}
\begin{tikzpicture}[node distance = 3ex, auto]
\tikzstyle{bigblock} = [rectangle, draw, thick, text width=20.5\WD, text badly centered, rounded corners, minimum height=4ex]
\tikzstyle{block} = [rectangle, draw=blue!80!black, thick, anchor=west, fill=white,
    text width=10\WD, text ragged, rounded corners, minimum height=8ex]
 \tikzstyle{smallblock} = [rectangle, draw=blue!80!black, thick, anchor=west, fill=white,
    text width=6\WD, text ragged, rounded corners, minimum height=8ex]   
 \tikzstyle{tinyblock} = [rectangle, draw=blue!80!black, thick, anchor=west, fill=white,
    text width=4.5\WD, text ragged, rounded corners, minimum height=8ex]      
\tikzstyle{line} = [draw, -latex']
\tikzstyle{lined} = [draw, latex'-latex']
\node [bigblock,draw=blue!80!black,fill=blue!10] (gaptooth) {\textbf{Patch scheme for \textsc{pde}s}\\[1ex]
    \begin{tikzpicture}[node distance = 3ex, auto]
    \node [block] (configPatches) {\textbf{Define problem and construct patches}
    
    Invoke \texttt{configpatches1} (for 1D) or \texttt{configpatches2} (for 2D) to define the microscale problem (\textsc{pde}, domain, boundary conditions, etc) and the desired patch structure (number of patches, patch size, coupling order, etc).
These functions initialise the global struct \texttt{patches}. 
The components of \texttt{patches} contain all information required to solve the microscale problem within each patch. 
If necessary, define additional components for struct \texttt{patches} (e.g., see \texttt{EnsembleAverageExample.m}).};
    \node [block, below=of configPatches] (microPDE) {\textbf{Solve microscale problem within each patch}\\
    Call the \textsc{pde} solver which is to evaluate the microscale problem within each patch. This solver may be a Matlab defined function (such as \texttt{ode15s} or \texttt{ode45}) or a user defined function (such as Runge--Kutta).
    
    Input of the \textsc{pde} solver is the function \texttt{patchSmooth1} (for 1D) or \texttt{patchSmooth2} (for 2D) which  interfaces with the \textsc{pde} solver and the microscale \textsc{pde}. Other inputs are the time span and initial conditions. Output of the \textsc{pde} solver is the solution of the patch \textsc{pde} over the given time span, but only evaluated within the defined patches.};
    \node [smallblock, above right=-2.2\WD and 2\WD of microPDE] (patchSmooth1) {\textbf{Interface to time integrators}
    
    The \textsc{pde} function (\texttt{patchSmooth1} or \texttt{patchSmooth2}) interfaces with the \textsc{pde} solve, the microscale \textsc{pde} and the patch coupling conditions. Input is the \textsc{pde} field at one time-step and output is the field at the next time-step.};
    \node [tinyblock, below left=3ex and -3\WD of patchSmooth1] (coupling) {\textbf{Coupling conditions}
    
    Coupling conditions are evaluated in \texttt{patchEdge1} (for 1D) or \texttt{patchEdge2} (for 2D) with the coupling order defined by global struct component \texttt{patches.ordCC}.};
    \node [tinyblock, below right=3ex and -3\WD of patchSmooth1] (micropde) {\textbf{Microscale \textsc{pde}}
    
    This \textsc{pde} is defined by the global struct \texttt{patches}, for example component \texttt{patches.fun} defines the function (e.g.,  \texttt{BurgersPDE} or \texttt{heteroDiff}) and \texttt{patches.x} defines the domain of the patches};
    \node [block,draw=red!80!black,fill=red!10, below=of microPDE] (pi) {\hyperref[fig:constructPI]{\textbf{\textbf{Projective integration scheme (if needed)}}}\\
    };    
    \path [lined,very thick,-latex] (configPatches) -- (microPDE);
    \path [lined,very thick] (microPDE) to[out=0,in=180] (patchSmooth1);
    \path [lined,very thick] (patchSmooth1) to[out=270,in=90] (coupling);
    \path [lined,very thick] (patchSmooth1) to[out=270,in=90] (micropde);
    \path [lined,very thick,latex-latex] (microPDE) -- (pi);
    \end{tikzpicture}
    };   
\node [bigblock,draw=black,below=of gaptooth] (process) {\textbf{Process results and plot}};
 \path [lined,very thick,-latex] (gaptooth) -- (process);
\end{tikzpicture}
\end{maxipage}
\end{figure}



\input{../Patch/configPatches1.m}
\input{../Patch/patchSmooth1.m}
\input{../Patch/patchEdgeInt1.m}
\input{../Patch/homogenisationExample.m}
\begin{devMan}
\input{../Patch/BurgersExample.m}
\input{../Patch/ensembleAverageExample.m}
\input{../Patch/waterWaveExample.m}
\end{devMan}

% 2D stuff
\input{../Patch/configPatches2.m}
\input{../Patch/patchSmooth2.m}
\input{../Patch/patchEdgeInt2.m}
\begin{devMan}
\input{../Patch/wave2D.m}



\section{To do}
\begin{itemize}
\item Testing needs to be quantitative.
\item more than two space dimensions??
\item Heterogeneous microscale via averaging regions---but I suspect should be separated from simple homogenisation
\item Parallel processing versions.
\item ??
\item Adapt to maps in micro-time?  Surely easy, just an example.
\end{itemize}


\section{Miscellaneous tests}
\input{../Patch/patchEdgeInt1test.m}
\input{../Patch/patchEdgeInt2test.m}

\end{devMan}

%    \includeversion{devMan}
%}%end-exclusion

