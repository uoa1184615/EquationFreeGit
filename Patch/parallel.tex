% input *.m files for Matlab's Parallel Computation of the
% Patch scheme in 1D, 2D, and 3D. AJR, Nov -- Dec 2020
%!TEX root = ../Doc/eqnFreeDevMan.tex
\chapter{Matlab parallel computation of the patch scheme}
\label{sec:parallel}
\localtableofcontents


Get familiar with the patch scheme of \cref{sec:patch}.

For large-scale simulations, we here assume you have a
compute cluster with many independent computer processors
linked by a high-speed network. The functions we provide
here aims to distribute computations in parallel across the
cluster. \Matlab's Parallel Computing Toolbox empowers a
reasonably straightforward way to implement this.

The examples listed herein are all \emph{Proof of
Principle}: as coded they are all small enough that
non-parallel execution is here actually much quicker that
the parallel execution.\footnote{Although I have so far only
tested with two parallel workers??}  One needs significantly
larger and/or more detailed problems than these examples
before parallel execution is effective.

The patch scheme has a clear domain decomposition of
assigning relatively few patches to each processor.

As in all parallel cluster computing, interprocessor
communication time all to often dominates.  It is important
to reduce communication as much as possible compared to
computation.  Consequently, parallel computing is only
effective when there is a very large amount of microscale
computation done on each processor per communication.

To minimise communication in time-dependent problems we have
drafted a special integrator \verb|RK2mesoPatch|, 
\cref{sec:RK2mesoPatch}, that communicates between patches
only on a meso-time.


\input{../Patch/chanDispSpmd.m}

\input{../Patch/spmdHomoDiff31.m}

\input{../Patch/RK2mesoPatch.m}



\begin{devMan}


\input{../Patch/rotFilmSpmd.m}


\section{To do}
\begin{itemize}
\item Lots ??
\end{itemize}

\end{devMan}
