% input *.m files for Matlab's Parallel Computation of the
% Patch scheme in 1D, 2D, and 3D. AJR, Nov -- Dec 2020
%!TEX root = ../Doc/eqnFreeDevMan.tex
\chapter{Matlab parallel computation of the patch scheme}
\label{sec:parallel}
\localtableofcontents


For large-scale simulations, we here assume you have a
compute cluster with many independent computer processors
linked by a high-speed network. The functions we provide in
our toolbox aim to distribute computations in parallel
across the cluster. \emph{\Matlab's Parallel Computing
Toolbox} empowers a reasonably straightforward way to
implement this parallelisation.\footnote{This
parallelisation is not written for, nor tested for, Octave.}
The reason is that the patch scheme (\cref{sec:patch}) has a
clear domain decomposition of assigning relatively few
patches to each processor.

The examples listed herein are all \emph{Proof of
Principle}: as coded they are all small enough that
non-parallel execution is here much quicker that the
parallel execution.  One needs significantly larger and/or
more detailed problems than these examples before parallel
execution is effective.  

As in all parallel cluster computing, interprocessor
communication time all to often dominates.  It is important
to reduce communication as much as possible compared to
computation.  Consequently, parallel computing is only
effective when there is a very large amount of microscale
computation done on each processor per communication---all
of the examples listed herein are quite small and so the
parallel computation of these is much slower than serial
computation.  We guesstimate that the microscale code may
need, per time-step, of the order of many millions of
operations per processor in order for the parallelisation to
be useful.

To help minimise communication in time-dependent problems we
have drafted a special integrator \verb|RK2mesoPatch|,
\cref{sec:RK2mesoPatch}, that communicates between patches
only on a meso-time \cite[]{Bunder2015a}.


\input{../Patch/chanDispSpmd.m}

\input{../Patch/rotFilmSpmd.m}

\input{../Patch/homoDiff31spmd.m}

\input{../Patch/RK2mesoPatch.m}



\begin{devMan}

\section{To do}
\begin{itemize}
\item Detailed profiling of the spmd communication to seek better parallelisation.
\end{itemize}

\end{devMan}
