% input *.m files for the Patch scheme in 1D and 2D. AJR,
% Nov 2017 -- Feb 2019
%!TEX root = ../Doc/eqnFreeDevMan.tex
\chapter{Patch scheme for given microscale discrete space system}
\label{sec:patch}
\localtableofcontents

The patch scheme applies to spatio-temporal systems where
the spatial domain is larger than what can be computed in
reasonable time.  In the scheme we compute only on small patches
of the space-time domain, and produce correct macroscale
predictions by craftily coupling the patches across
unsimulated space \cite[e.g.]{Hyman2005, Samaey03b,
Samaey04, Roberts06d, Liu2015}.

The spatial structure is to be on a lattice such as obtained
from finite difference approximation of a \pde. Usually
continuous in time.

\paragraph{Quick start}
See \cref{sec:configPatches1eg,sec:configPatches2eg} which
list example basic code that uses the provided functions to
simulate 1D Burgers'~\pde\ and a 2D nonlinear
`diffusion'~\pde.

\input{../Patch/configPatches1.m}
\input{../Patch/patchSmooth1.m}
\input{../Patch/patchEdgeInt1.m}
\begin{devMan}
\input{../Patch/BurgersExample.m}
\input{../Patch/HomogenisationExample.m}
\input{../Patch/waterWaveExample.m}
\end{devMan}

% 2D stuff
\input{../Patch/configPatches2.m}
\input{../Patch/patchSmooth2.m}
\input{../Patch/patchEdgeInt2.m}
\begin{devMan}
\input{../Patch/wave2D.m}



\section{To do}
\begin{itemize}
\item Testing needs to be quantitative.
\item more than two space dimensions??
\item Heterogeneous microscale via averaging regions---but I suspect should be separated from simple homogenisation
\item Parallel processing versions.
\item ??
\item Adapt to maps in micro-time?  Surely easy, just an example.
\end{itemize}


\section{Miscellaneous tests}
\input{../Patch/patchEdgeInt1test.m}
\input{../Patch/patchEdgeInt2test.m}

\end{devMan}
