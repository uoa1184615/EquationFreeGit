\documentclass[11pt]{article}
\usepackage{amsmath,amssymb}
\usepackage[usenames,dvipsnames]{xcolor}
\definecolor{matlab1}{rgb}{0,0.4470,0.7410}
\definecolor{matlab2}{rgb}{0.8500,0.3250,0.0980}
\definecolor{matlab3}{rgb}{0.9290,0.6940,0.1250}
\definecolor{matlab4}{rgb}{0.4940,0.1840,0.5560}
\definecolor{matlab5}{rgb}{0.4660,0.6740,0.1880}
\definecolor{matlab6}{rgb}{0.3010,0.7450,0.9330}
\definecolor{matlab7}{rgb}{0.6350,0.0780,0.1840}

\usepackage{tikz,pgfplots}
\pgfplotsset{compat=newest}
\def\tikzsetnextfilename#1{} % define if not using following
\usepgfplotslibrary{external}
\tikzexternalize
\usetikzlibrary{positioning}

% required for picture of micro-grid
\newcounter{i}
\def\nw{\textsc{nw}}\def\ne{\textsc{ne}}
\def\sw{\textsc{sw}}\def\se{\textsc{se}}
% need following definitions for microgrid symbols
\def\temp#1{#1}% by default do nothing
\def\oSym{\temp{$\color{green!70!black}\circledcirc$}}
\def\xSym{\temp{$\color{green!70!black}\otimes$}}
\def\uSym{\temp{$\color{blue}\blacktriangleright$}}
\def\vSym{\temp{$\color{red}\blacktriangle$}}

\newcommand{\dx}{\delta_x}
\newcommand{\dy}{\delta_y}
\newcommand{\E}{\cdot 10^}
\def\half{\frac12}
\begin{document}


%\begin{figure}
%%\def\nx{6}
%%\def\ny{3}
%\begin{tikzpicture}[x=14mm,y=14mm]
%\pgfmathsetmacro\nx{6};
%\pgfmathsetmacro\ny{4};
%\pgfmathsetmacro\nxp{\nx+1};
%\pgfmathsetmacro\nyp{\ny+1};
%\pgfmathsetmacro\nxpp{\nx+2}
%\pgfmathsetmacro\nypp{\ny+2}
% \foreach \j in {1,...,\ny}
% 	{
% 	\draw[draw=yellow,thick,fill=yellow!20] (1.5,\j)--(1,\j+0.5)--(1.5,\j+1); 
% 	\draw[draw=yellow,thick,fill=yellow!20] (\nxp,\j+0.5)--(\nxp+0.5,\j+1)--(\nxp,\j+1.5);
% 	}  
% 	\fill[yellow!20] (1.5,1.5) rectangle (\nxp,\nyp);
% \foreach \i in {1,...,\nx}
% 	{
% 	\draw[draw=yellow,thick,fill=yellow!20] (\i,1.5)--(\i+0.5,1)--(\i+1,1.5); 
% 	\draw[draw=yellow,thick,fill=yellow!20] (\i+0.5,\nyp)--(\i+1,\nyp+0.5)--(\i+1.5,\nyp); 
% 	}  	
%\draw [step=1,magenta!40,thin] (0.5,0.5) grid (\nxpp,\nypp);
%\foreach \i in {1,...,\nxpp}   \foreach \j in {1,...,\nypp} % 
%      {
%        \node[] at (\i,\j) {\vSym};     
%        \node[] at (\i-0.5,\j-0.5) {\uSym};            
%      }
% \foreach \i in {1,...,\nxp}   \foreach \j in {1,...,\nyp} % 
%      {  
%        \node[] at (\i,\j+0.5) {\oSym};     
%        \node[] at (\i+0.5,\j) {\xSym};           
%      }        
%      \foreach \i in {1,...,\nx} % x-axis integers
%    {
%     \node[below right] at (\i+1,1) {$\i$};  
%    }   	 
%\node[below] at (1,1) {$i=0$};     
%\foreach \j in {1,...,\ny} % y-axis integers
%    {
%     \node[above right] at (1,1+\j) {$\j$};  
%    }   	  
%\node[above] at (1,1) {$j=0$}; 
%\draw[black,<->] (1,0.25)-- node[midway, below] {$h_x=\dx n_x$} (\nxp,0.25);    
%\draw[black,<->] (0,1)-- node[midway,above,rotate=90] {$h_y=\dy n_y$} (0,\nyp);               
%\end{tikzpicture}
%\caption{Patch for a 2D membrane which is large in both dimensions.
%The blue triangle at \((0.5, 4.5)\) must be be calculated in a two-step process: first interpolate horizontally to calculate the blue triangle at \((0.5, 1.5)\); then interpolate vertically to calculate the blue triangle at \((0.5, 4.5)\).
%Similarly, to calculate the red triangle at \((0,6)\): first interpolate vertically to calculate the red triangle at \((0, 1)\); then interpolate horizontally to calculate the red triangle at \((0, 6)\).
%}
%\end{figure};

\begin{figure}
\def\nxfigs{7}
\def\nyfigs{4}
\begin{tikzpicture}[x=14mm,y=14mm]
\pgfmathsetmacro{\nx}{\nxfigs}
\pgfmathsetmacro{\ny}{\nyfigs}
\pgfmathsetmacro{\nxp}{\nx+1}
\pgfmathsetmacro{\nyp}{\ny+1}
\pgfmathsetmacro{\nxpp}{\nx+2}
\pgfmathsetmacro{\nypp}{\ny+2}
 \foreach \j in {1,...,\ny} %left and right edges
 	{
 	\draw[draw=yellow,thick,fill=yellow!20] (1,\j+0.5)--(0.5,\j+1)--(1,\j+1.5); 
 	\draw[draw=yellow,thick,fill=yellow!20] (\nxp-0.5,\j)--(\nxp,\j+0.5)--(\nxp-0.5,\j+1);
 	}  
	\fill[yellow!20] (1,1.5) rectangle (\nx+0.5,\nyp);
 \foreach \i in {1,...,\nx} %bottom and top edges
 	{
 	\draw[draw=yellow,thick,fill=yellow!20] (\i,1.5)--(\i+0.5,1)--(\i+1,1.5); 
 	\draw[draw=yellow,thick,fill=yellow!20] (\i-0.5,\nyp)--(\i,\nyp+0.5)--(\i+0.5,\nyp); 
 	}  	
\draw [step=1,magenta!40,thin] (0,0.5) grid (\nxpp-0.5,\nypp);
\foreach \i in {1,...,\nxpp}   \foreach \j in {1,...,\nypp} % 
      {
        \node[] at (\i-1,\j) {\vSym};     
        \node[] at (\i-0.5,\j-0.5) {\uSym};            
      }
 \foreach \i in {1,...,\nxp}   \foreach \j in {1,...,\nyp} % 
      {  
        \node[] at (\i,\j+0.5) {\oSym};     
        \node[] at (\i-0.5,\j) {\xSym};           
      }  
\foreach \i in {1,...,\nx} % top and bottom edge fields
      {  
	   \node[] at (\i+0.5,\ny+1.5) {$\square$};    
        \node[] at (\i+0.5,0.5) {$\square$};   
	   \node[] at (\i,\ny+2) {$\square$};    
        \node[] at (\i,1) {$\square$};  
      }     
\foreach \j in {1,...,\ny} % left and right edge fields
      {  
	   \node[] at (\nx+1,\j+1) {$\square$};    
        \node[] at (0,\j+1) {$\square$};   
	   \node[] at (\nx+1.5,\j+0.5) {$\square$};    
        \node[] at (0.5,\j+0.5) {$\square$};   
      }          
\node[] at (0.5,\ny+1.5) {$\square$}; % u top left
 \node[] at (\nx+1,1) {$\square$}; % v bottom right                
\foreach \i in {1,...,\nxp} % x-axis integers
    {
     \node[below right] at (\i,1) {$\i$};  
    }   	 
\node[below] at (0,1) {$i=0$};     
\foreach \j in {1,...,\nyp} % y-axis integers
    {
     \node[above right] at (0,1+\j) {$\j$};  
    }   	  
\node[above] at (0,1) {$j=0$};              
\draw[black,<->] (0.5,0.25)-- node[midway, below] {$h_x=\dx n_x$} (\nx+0.5,0.25);    
\draw[black,<->] (-0.5,1)-- node[midway,above,rotate=90] {$h_y=\dy n_y$} (-0.5,\nyp);               
\end{tikzpicture}
\caption{A patch with \(n_x=\nxfigs\) and \(n_y=\nyfigs\) has width \(h_x=\delta_xn_x\) and height \(h_y=\delta_y n_y\)\,.
The yellow region indicates the main part of the patch in which field values are determined by the elasticity differential equations. 
Symbols in squares~$\square$, just outside the yellow region, indicate patch edge fields that are determined by interpolation. 
Symbols represent horizontal displacement~\(u\)~(\uSym), vertical displacement~\(v\)~(\vSym), strains~\(\sigma_{xx/yy}\) and stresses~\(\varepsilon_{xx/yy}\)~(\oSym), and strains~\(\sigma_{xy}\) and stresses~\(\varepsilon_{xy}\)~(\xSym).
Indices \((i,j)\) are integer valued on the magenta grid (and red triangles), and the grid values that lie in the patch are \(i=1,2,\ldots,n_x\) and \(j=1,2,\ldots,n_y\)\,.
}
\end{figure}

\begin{figure}
\def\nxfigs{7}
\def\nyfigs{4}
\begin{tikzpicture}[x=14mm,y=14mm]
\pgfmathsetmacro{\nx}{\nxfigs}
\pgfmathsetmacro{\ny}{\nyfigs}
\pgfmathsetmacro{\nxp}{\nx+1}
\pgfmathsetmacro{\nyp}{\ny+1}
\pgfmathsetmacro{\nxpp}{\nx+2}
\pgfmathsetmacro{\nypp}{\ny+2}
 \foreach \j in {1,...,\ny} %left and right edges
 	{
 	\draw[draw=yellow,thick,fill=yellow!20] (1,\j+0.5)--(0.5,\j+1)--(1,\j+1.5); 
 	\draw[draw=yellow,thick,fill=yellow!20] (\nxp-0.5,\j)--(\nxp,\j+0.5)--(\nxp-0.5,\j+1);
 	}  
	\fill[yellow!20] (1,1.5) rectangle (\nx+0.5,\nyp);
 \foreach \i in {1,...,\nx} %bottom and top edges
 	{
 	\draw[draw=yellow,thick,fill=yellow!20] (\i,1.5)--(\i+0.5,1)--(\i+1,1.5); 
 	\draw[draw=yellow,thick,fill=yellow!20] (\i-0.5,\nyp)--(\i,\nyp+0.5)--(\i+0.5,\nyp); 
 	}  	
\draw [step=1,magenta!40,thin] (0,0.5) grid (\nxpp-0.5,\nypp);
\foreach \i in {1,...,\nxpp}   \foreach \j in {1,...,\nypp} % 
      {
        \node[] at (\i-1,\j) {\vSym};     
        \node[] at (\i-0.5,\j-0.5) {\uSym};            
      }
 \foreach \i in {1,...,\nxp}   \foreach \j in {1,...,\nyp} % 
      {  
        \node[] at (\i,\j+0.5) {\oSym};     
        \node[] at (\i-0.5,\j) {\xSym};           
      }  
\foreach \i in {1,...,\nx} % top and bottom edge fields
      {  
	   \node[] at (\i+0.5,\ny+1.5) {$\square$};    
        \node[] at (\i+0.5,0.5) {$\square$};   
	   \node[] at (\i,\ny+2) {$\square$};    
        \node[] at (\i,1) {$\square$};  
      }     
\foreach \j in {1,...,\ny} % left and right edge fields
      {  
	   \node[] at (\nx+1,\j+1) {$\square$};    
        \node[] at (0,\j+1) {$\square$};   
	   \node[] at (\nx+1.5,\j+0.5) {$\square$};    
        \node[] at (0.5,\j+0.5) {$\square$};   
      }          
\node[] at (0.5,\ny+1.5) {$\square$}; % u top left
 \node[] at (\nx+1,1) {$\square$}; % v bottom right                           
\foreach \j in {2,...,\nyp}
    {
    \draw[-latex] (1,\j) to[bend right=6] (\nxp-0.1,\j);
    \draw[-latex] (1.5,\j-0.5) to[bend right=6] (\nxp+0.5-0.1,\j-0.5);
    \draw[-latex]  (\nxp-1,\j) to[bend right=6]  (0+0.1,\j);
    \draw[-latex] (\nxp-0.5,\j-0.5) to[bend right=6] (0.5+0.1,\j-0.5);
    } 
\draw[-latex,thick] (1,1) to[bend right=6] (\nxp-0.1,1);
\draw[-latex,thick] (\nxp-0.5,\nyp+0.5) to[bend right=6] (0.5+0.1,\nyp+0.5);    
\end{tikzpicture}
\begin{tikzpicture}[x=14mm,y=14mm]
\pgfmathsetmacro{\nx}{\nxfigs}
\pgfmathsetmacro{\ny}{\nyfigs}
\pgfmathsetmacro{\nxp}{\nx+1}
\pgfmathsetmacro{\nyp}{\ny+1}
\pgfmathsetmacro{\nxpp}{\nx+2}
\pgfmathsetmacro{\nypp}{\ny+2}
 \foreach \j in {1,...,\ny} %left and right edges
 	{
 	\draw[draw=yellow,thick,fill=yellow!20] (1,\j+0.5)--(0.5,\j+1)--(1,\j+1.5); 
 	\draw[draw=yellow,thick,fill=yellow!20] (\nxp-0.5,\j)--(\nxp,\j+0.5)--(\nxp-0.5,\j+1);
 	}  
	\fill[yellow!20] (1,1.5) rectangle (\nx+0.5,\nyp);
 \foreach \i in {1,...,\nx} %bottom and top edges
 	{
 	\draw[draw=yellow,thick,fill=yellow!20] (\i,1.5)--(\i+0.5,1)--(\i+1,1.5); 
 	\draw[draw=yellow,thick,fill=yellow!20] (\i-0.5,\nyp)--(\i,\nyp+0.5)--(\i+0.5,\nyp); 
 	}  	
\draw [step=1,magenta!40,thin] (0,0.5) grid (\nxpp-0.5,\nypp);
\foreach \i in {1,...,\nxpp}   \foreach \j in {1,...,\nypp} % 
      {
        \node[] at (\i-1,\j) {\vSym};     
        \node[] at (\i-0.5,\j-0.5) {\uSym};            
      }
 \foreach \i in {1,...,\nxp}   \foreach \j in {1,...,\nyp} % 
      {  
        \node[] at (\i,\j+0.5) {\oSym};     
        \node[] at (\i-0.5,\j) {\xSym};           
      }  
\foreach \i in {1,...,\nx} % top and bottom edge fields
      {  
	   \node[] at (\i+0.5,\ny+1.5) {$\square$};    
        \node[] at (\i+0.5,0.5) {$\square$};   
	   \node[] at (\i,\ny+2) {$\square$};    
        \node[] at (\i,1) {$\square$};  
      }     
\foreach \j in {1,...,\ny} % left and right edge fields
      {  
	   \node[] at (\nx+1,\j+1) {$\square$};    
        \node[] at (0,\j+1) {$\square$};   
	   \node[] at (\nx+1.5,\j+0.5) {$\square$};    
        \node[] at (0.5,\j+0.5) {$\square$};   
      }          
\node[] at (0.5,\ny+1.5) {$\square$}; % u top left
 \node[] at (\nx+1,1) {$\square$}; % v bottom right                           
\foreach \i in {1,...,\nx}
    {
    \draw[-latex] (\i,\nyp) to[bend left=6] (\i,1+0.1);
    \draw[-latex] (\i+0.5,1.5) to[bend left=6] (\i+0.5,\nyp+0.5-0.1);
    \draw[-latex]  (\i,2) to[bend left=6]  (\i,\nypp-0.1);
    \draw[-latex] (\i+0.5,\nyp-0.5) to[bend left=6] (\i+0.5,0.5+0.1);
    } 
\draw[-latex,thick] (0.5,1.5) to[bend left=6] (0.5,\nyp+0.5-0.1);  
\draw[-latex,thick] (\nx+1,\nyp) to[bend left=6] (\nx+1,1+0.1); 
\end{tikzpicture}
\caption{(top)~horizonal and (bottom)~vertical interpolation in a patch. 
Fields in squares~$\square$ outside and to the left (right) of the yellow patch are determined from horizontal interpolations of fields on the right (left) of this patch and adjacent patches, as indicated by the black arrows. 
Similarly, fields in squares~$\square$ outside and to the top (bottom) of the yellow patch are determined from vertical interpolations of fields on the bottom (top) of this patch and adjacent patches.
Two exceptions to these interpolation statements are the top-left horizontal displacement at \((0.5,\nyfigs.5)\) and the bottom-right vertical displacement at \((0,\numexpr\nxfigs+1\relax)\).
}
\end{figure}
\end{document}

rc87RE63srtu3@

