\documentclass[11pt,a4paper]{article}
\pagestyle{empty}
\usepackage[landscape]{geometry}
%\renewcommand{\topfraction}{0.85}
%\renewcommand{\bottomfraction}{0.85}
%\renewcommand{\textfraction}{0.15}
%\renewcommand{\floatpagefraction}{0.7}


\usepackage{pgfplots} 
\pgfplotsset{compat=1.15} %,tikz
%\IfFileExists{ajr.sty}{}{
%    \pgfplotsset{compat=1.10}
 %   }
\usepgfplotslibrary{external} %exports an external pdf for each tikz figure
\tikzexternalize
% The following forces redraw of all tikz graphics;
\tikzset{external/force remake}
\usetikzlibrary{decorations.markings}
\usetikzlibrary{shapes,arrows,fit}
\usetikzlibrary{positioning}

\usepackage{bold-extra}
\usepackage[T1]{fontenc}

\begin{document}
 
\begin{figure}
\centering
\begin{tikzpicture}[node distance = 3ex, auto]
\tikzstyle{bigblock} = [rectangle, draw, thick,   text width=20.5cm, text badly centered,
    rounded corners, minimum height=4ex]
\tikzstyle{block} = [rectangle, draw=blue!80!black, thick, anchor=west, fill=white,
    text width=10cm, text ragged, rounded corners, minimum height=8ex]
 \tikzstyle{smallblock} = [rectangle, draw=blue!80!black, thick, anchor=west, fill=white,
    text width=6cm, text ragged, rounded corners, minimum height=8ex]   
 \tikzstyle{tinyblock} = [rectangle, draw=blue!80!black, thick, anchor=west, fill=white,
    text width=4.5cm, text ragged, rounded corners, minimum height=8ex]      
\tikzstyle{line} = [draw, -latex']
\tikzstyle{lined} = [draw, latex'-latex']
\node [bigblock,draw=blue!80!black,fill=blue!10] (gaptooth) {\textbf{Gap-tooth scheme for \textsc{pde}s}\\
    \begin{tikzpicture}[node distance = 3ex, auto]
    \node [block] (configPatches) {\textbf{Define problem and construct patches}\\ Call \texttt{configpatches1} (for 1D) or \texttt{configpatches2} (for 2D) with inputs which define the microscale problem (\textsc{pde}, domain, boundary conditions etc) and the desired patch structure (number of patches, patch size, coupling order etc).\\
    Output of \texttt{configpatches1} or \texttt{configpatches2}  is the global struct \texttt{patches}. The components of this struct should contain all information required to solve the microscale problem within each patch (function, microscale lattice points in each patch etc). If necessary, the user should define additional components for struct \texttt{patches} (e.g., as in \texttt{HomogenisationExample}).};
    \node [block, below=of configPatches] (microPDE) {\textbf{Solve microscale problem within each patch}\\
    Call the \textsc{pde} solver which is to evaluate the microscale problem within each patch. This solver may be a Matlab defined function (such as \texttt{ode15s} or \texttt{ode45}) or a user defined function (such as Runge--Kutta).\\
    Input of the \textsc{pde} solver is the function \texttt{patchSmooth1} (for 1D) or \texttt{patchSmooth2} (for 2D) which  interfaces with the \textsc{pde} solver and the microscale \textsc{pde}. Other inputs are the time span and initial conditions. Output of the \textsc{pde} solver is the solution of the patch \textsc{pde} over the given time span, but only evaluated within the defined patches.};
    \node [smallblock, above right=-2.2cm and 2cm of microPDE] (patchSmooth1) {\textbf{Interface to time integrators}\\
    The \textsc{pde} function (\texttt{patchSmooth1} or \texttt{patchSmooth2}) interfaces with the \textsc{pde} solve, the microscale \textsc{pde} and the patch coupling conditions. Input is the \textsc{pde} field at one time-step and output is the field at the next time-step.};
    \node [tinyblock, below left=0.5cm and -3cm of patchSmooth1] (coupling) {\textbf{Coupling conditions}\\
    Coupling conditions are evaluated in \texttt{patchEdge1} (for 1D) or \texttt{patchEdge2} (for 2D) with the coupling order defined by global struct component \texttt{patches.ordCC}.};
    \node [tinyblock, below right=0.5cm and -3cm of patchSmooth1] (micropde) {\textbf{Microscale \textsc{pde}}\\ 
    This \textsc{pde} is defined by the global struct \texttt{patches}, for example component \texttt{patches.fun} defines the function (e.g.,  \texttt{BurgersPDE} or \texttt{heteroDiff}) and \texttt{patches.x} defines the domain of the patches};
    \node [block,draw=red!80!black,fill=red!10, below=of microPDE] (pi) {\textbf{\textbf{Projective integration scheme}}\\
    };    
    \path [lined,very thick,-latex] (configPatches) -- (microPDE);
    \path [lined,very thick] (microPDE) to[out=0,in=180] (patchSmooth1);
    \path [lined,very thick] (patchSmooth1) to[out=270,in=90] (coupling);
    \path [lined,very thick] (patchSmooth1) to[out=270,in=90] (micropde);
    \path [lined,very thick,latex-latex] (microPDE) -- (pi);
    \end{tikzpicture}
    };   
\node [bigblock,draw=black,below=of gaptooth] (process) {\textbf{Process results and plot}};
 \path [lined,very thick,-latex] (gaptooth) -- (process);
\end{tikzpicture}
\end{figure}


\begin{figure}
\centering
\begin{tikzpicture}[node distance = 0.5cm, auto]
\tikzstyle{bigblock} = [rectangle, draw, thick,   text badly centered, 
    text width=12cm, rounded corners, minimum height=2em]
\tikzstyle{block} = [rectangle, draw=red!80!black, thick, anchor=west, fill=white,
    text width=5cm, rounded corners, minimum height=4em]
 \tikzstyle{smallblock} = [rectangle, draw=red!80!black, thick, anchor=west, fill=white,
    text width=3.4cm, rounded corners, minimum height=4em]  
 \tikzstyle{yesblock} = [rectangle, draw=red!80!black, thick, anchor=west, fill=white,
    text width=1.2cm, rounded corners, minimum height=1.2em]   
\tikzstyle{line} = [draw, -latex']
\tikzstyle{lined} = [draw, latex'-latex']
\node [bigblock,draw=red!80!black,fill=red!10] (gaptooth) {\textbf{Choosing the macro solver in PI:}\\
\vspace{2pt}
    \begin{tikzpicture}[node distance = 1cm, auto]
    \node [block,right=5cm] (timestep) {\textbf{Is an appropriate time-step known for the slow dynamics?}};
    \node [yesblock, below =0.5cm of timestep] (timeyes) {Yes};
    \node [yesblock, right=0.4cm of timestep] (timeno) {No};
    \node [block, below=0.5cm of timeyes] (slowsol) {\textbf{Is a particular solver desired to simulate the slow dynamics?}};
     \node [smallblock, below right=0.5cm and 0.5cm of timeno] (pig) {Choose \texttt{PIG()} to do simulation};
         \node [yesblock, below =0.4cm of slowsol] (slowno) {No};
    \node [yesblock,  right=0.4cm of slowsol] (slowyes) {Yes};
         \node [smallblock, below =0.5cm of slowno] (pirk) {Choose \texttt{PIRK2()} or \texttt{PIRK4()} to do simulation};
         \path [line, thick] (timestep) to[out=-90,in=90] (timeyes);
         \path [line, thick] (timestep) to[out=0,in=180] (timeno);
          \path [line, thick] (timeno) to[out=0,in=90] (pig);
          \path [line, thick] (timeyes) to[out=-90,in=90] (slowsol);
          \path [line, thick] (slowsol) to[out=0,in=180] (slowyes);
          \path [line, thick] (slowsol) to[out=-90,in=90] (slowno);
          \path [line, thick] (slowno) to[out=-90,in=90] (pirk);
          \path [line, thick] (slowyes) to[out=0,in=-90] (pig);
    \end{tikzpicture}
    };   
\end{tikzpicture}
\end{figure}


\begin{figure}
\centering
\begin{tikzpicture}[node distance = 3ex, auto]
\tikzstyle{bigblock} = [rectangle, draw, thick,   text width=20.5cm, text badly centered,
    rounded corners, minimum height=4ex]
\tikzstyle{block} = [rectangle, draw=red!80!black, thick, anchor=west, fill=white,
    text width=10cm, text ragged, rounded corners, minimum height=8ex]
 \tikzstyle{smallblock} = [rectangle, draw=red!80!black, thick, anchor=west, fill=white,
    text width=6cm, text ragged, rounded corners, minimum height=8ex]   
 \tikzstyle{tinyblock} = [rectangle, draw=red!80!black, thick, anchor=west, fill=white,
    text width=4.5cm, text ragged, rounded corners, minimum height=8ex]    
     \tikzstyle{smallenclose} = [rectangle, draw=red!80!black, thick, anchor=west, fill=white,
    text width=6.2cm, text ragged, rounded corners, minimum height=8ex]   
     \tikzstyle{refblock} = [rectangle, draw=blue!80!black, thick, anchor=west, fill=blue!10,
    text width=3.2cm, rounded corners, minimum height=1.05em] 
\tikzstyle{line} = [draw, -latex']
\tikzstyle{lined} = [draw, latex'-latex']
\node [bigblock,draw=red!80!black,fill=red!10] (gaptooth) {\textbf{Projective Integration scheme}\\
    \begin{tikzpicture}[node distance = 3ex, auto]
    \node [block] (setmicro) {\textbf{Set microsolver}\\ 
    Define or construct the function \texttt{solver()} that calls a black-box microsolver. Possible aids:\\
    Use the toolbox
    \begin{tikzpicture}[node distance = 0.5cm, auto]   
    \node [refblock] (gapref) {Gap-tooth scheme};
    \end{tikzpicture}
    to discretise a pde.\\
    Use \texttt{bbgen()} if the microsolver is a standard solver, \texttt{ode45} e.g., and needs to be converted into a black-box.\\
    Use \texttt{cmdc()} as a wrapper for the microsolver if the slow variables would otherwise change significantly over the microsolver.};
    \node [tinyblock, right=0.3cm of setmicro] (lift) {\textbf{Set lifting/restriction}\\ 
    If needed, set functions \texttt{restrict()} and \texttt{lift()} to convert between macro and micro problems/variables. These are optional arguments to the \texttt{PI} functions.};
    \node[smallenclose, below right=0.2cm and -1.4cm of lift] (setmacro){\textbf{Set macrosolver, define problem}\\ 
    \vspace{1pt}
        \begin{tikzpicture}
    \node [smallblock]%, below=0.6cm of pig]
     (pirk) {\textbf{If using \texttt{PIRK()}:}\\ 
    Set the vector of times \texttt{tspan} at which solutions are desired, noting that the intervals between times will be used as time-steps in the numerical scheme. Set initial conditions \texttt{IC}.};
    \end{tikzpicture}
    
    \vspace{2pt}
    
        \begin{tikzpicture}
   \node [smallblock]%, below right=0.2cm and -1cm of lift]
    (pig) {\textbf{If using \texttt{PIG()}:}\\ 
    Set the solver to be used on the macro scale. Set any needed time inputs or time-step data in \texttt{tspan}. Set initial conditions \texttt{IC}.};
    \end{tikzpicture}

    };
   
 \node [block, below right=1.5cm and -8cm of setmicro] (dopi) {\textbf{Do PI}\\ 
    Call the appropriate PI function as e.g. \\
    \texttt{[t,x] = PIRK2(solver,tspan,IC)}.\\
    Additional outputs may be requested to obtain details on the micro states and estimates of the slow vector field over the simulation.};
             \path [line, thick] (setmicro) to[out=-90,in=120] (dopi);
         \path [line, thick] (lift) to[out=-90,in=20] (dopi);
         \path [line, thick] (setmacro) to[out=180,in=0] (dopi);
%          \path [line, thick] (pig) to[out=180,in=0] (dopi);
%          \path [line, thick] (pirk) to[out=180,in=0] (dopi);
%          \draw [dashed, very thick] (pig) -- (pirk);
    \end{tikzpicture}
    };   
\end{tikzpicture}
\end{figure}


\end{document}


    \path [lined,very thick] (patchSmooth1) to[out=240,in=60] (coupling);
    \path [lined,very thick] (patchSmooth1) to[out=300,in=120] (micropde);