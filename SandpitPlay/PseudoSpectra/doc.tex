\documentclass[11pt,a5paper]{article}
\usepackage[a5paper,margin=9mm,bmargin=15mm]{geometry}
%\pagestyle{headings}\let\MakeUppercase\sf

\title{Pseudospectra of the patch scheme in various scenarios shows mostly insensitive}

\author{A.~J.~Roberts%
\thanks{School of Mathematical Sciences, University of Adelaide, South Australia.
\url{https://profajroberts.github.io},
\url{http://orcid.org/0000-0001-8930-1552}}
}



\usepackage{pgfplots} 
\pgfplotsset{compat=newest} 
\usepgfplotslibrary{external} %exports an external pdf for each tikz 
% have not externalized yet though

% fancyvrb does code listing, including line numbers
\usepackage{fancyvrb}
\newenvironment{matlab}%
    {\Verbatim[numbers=left,firstnumber=\the\inputlineno]}%
    {\endVerbatim}
% Also get fancyvrb to omit %{ and %} pairs, 
% although this requires they always be used
\makeatletter
\def\fancyvrbStartStop{%
  \edef\FancyVerbStartString{\@percentchar\@charrb} 
  \edef\FancyVerbStopString{\@percentchar\@charlb} }
\makeatother
% could change appearance with e.g. \renewcommand{\theFancyVerbLine}{%
%  \textcolor{red}{\small 8.\alph{FancyVerbLine}}}


\usepackage{url,microtype,amsmath,amssymb,defns,graphicx,hyperref,doi}
\usepackage{mybiblatex}
\usepackage[leftcaption,raggedright]{sidecap}
\hypersetup{colorlinks
    ,linkcolor=blue,citecolor=blue,pagecolor=blue%
    ,urlcolor=magenta,filecolor=magenta,breaklinks%
    ,dvips,bookmarks,bookmarksopen}
\usepackage[capitalise,nameinlink,noabbrev]{cleveref}
\crefname{equation}{}{}
% Default "Item" useless, use enumitem and ref=
\crefname{enumi}{}{}
\crefname{enumii}{}{}
\crefname{enumiii}{}{}
\crefname{enumiv}{}{}

% These are recommended by Rob J Hyndman (2011)
% \footnote{\url{http://robjhyndman.com/researchtips/latex-floats/}}
\setcounter{topnumber}{2}
\setcounter{bottomnumber}{2}
\setcounter{totalnumber}{4}
\renewcommand{\topfraction}{0.85}
\renewcommand{\bottomfraction}{0.85}
\renewcommand{\textfraction}{0.15}
\renewcommand{\floatpagefraction}{0.7}

\def\figurename{\sl Figure}

\def\sign{\operatorname{sign}}
%\newenvironment{devMan}{}{}
\newcommand{\todo}[1]{\footnote{\textbf{ToDo:}\quad #1}}
\def\localtableofcontents{}
\def\into{${}\mapsto{}$}
%\graphicspath{{Figs/}{../../Patch/Figs/}}

%\def\E#1{\textsc{e}{-}#1}
%\newcounter{i}
%\def\Ra{\ensuremath{\operatorname{Ra}}}
%\def\bc{\textsc{bc}}
\Vec v
\Bb C

\newcommand\PSfig[2]{\begin{SCfigure}\centering
    \caption{\label{fig:#1}#2}
    \includegraphics[scale=0.75]{#1}
    \end{SCfigure}}


\begin{document}
\fancyvrbStartStop

\maketitle

\tableofcontents


\section{Introduction}

Recall that eigenvalues~\(\lambda\) of a matrix~\(A\) are complex numbers~\(z\in\CC\) such that the \emph{resolvent}~\((zI-A)^{-1}\) does not exist---is `infinite' \citep[p.3]{Trefethen2005}.

\begin{definition}
For a given \(N\times N\) matrix~\(A\), the \emph{\(\epsilon\)-pseudospectrum} of~\(A\) is the (open) set \(z\in\CC\) such that \citep[pp.13--17]{Trefethen2005} any of the following four equivalent conditions hold:\footnote{Define \(\|\vv\|:=\sqrt{\vv\cdot\vv}\), and  \(\|A\|:=\max_{\|\vv\|=1}\|A\vv\|\).}\begin{itemize}
\item \(\|(zI-A)^{-1}\|>\epsilon^{-1}\);
\item \(z\) is an eigenvalue of~\(A+E\) for some matrix~\(E\) with \(\|E\|<\epsilon\);
\item \(\min_{\|\vv\|=1}\|(zI-A)\vv\|<\epsilon\);
\item the smallest singular value of \(zI-A\) is\({}<\epsilon\).
\end{itemize}
The \(\epsilon\)-pseudospectrum is denoted~\(\sigma_\epsilon(A)\).
\end{definition}

So-called normal matrices have the nicest properties: a matrix~\(A\) is termed \emph{normal} if it has a complete set of orthogonal eigenvectors \citep[p.18]{Trefethen2005}.   
That is, if real then~\(A\) is normal iff it is symmetric.  
Equivalently, a matrix~\(A\) is normal if it commutes with its adjoint: \(A^\dag A=AA^\dag\).

\begin{theorem}
Define the \(\epsilon\)-disk \(B_\epsilon:=\{z\in\CC: |z|<\epsilon\}\).
Then \(\sigma_\epsilon(A)\supseteq \sigma(A)+B_\epsilon\) for each matrix~\(A\).  Equality holds iff \(A\)~is normal.
\end{theorem}
\PSfig{heteroDiffPseudoSpectra1}{pseudo-spectra of a patch scheme applied to heterogeneous diffusion in 1D space (\cref{sec:heteroDiffPseudoSpectra1}).  
Plotted are the contours of the reciprocal-resolvent \(1\big/(zI-A)^{-1}\) for \(\epsilon=10^{-1:2}\).  The interior of these contours is the corresponding \(\epsilon\)-pseudospectrum.  The asinh-scaling of each axis deforms the circular contours.}%
That is, for a normal matrix~\(A\), the \(\epsilon\)-pseudospectrum is the union of \(\epsilon\)-disks centred on each eigenvalue.
This is the best case.
For example, \cref{fig:heteroDiffPseudoSpectra1} plots, for \(\epsilon=10^{-1:2}\), the \(\epsilon\)-pseudospectrum of a patch scheme applied to heterogeneous diffusion in 1D space: the plot is consistent with the pseudospectra being the union of circular disks (do not be misled by the nonlinear scaling of both axes).


For a non-normal matrix, the \(\epsilon\)-pseudospectrum is larger than (a superset of) the union of \(\epsilon\)-disks centred on each eigenvalue.


\section{Patch schemes}

Possibly the most significant property of the \(\epsilon\)-pseudospectra for a patch scheme is the property that \(\sigma_\epsilon(A)\) are the eigenvalues of~\(A+E\) for some~\(E\) with \(\|E\|<\epsilon\).
Let matrix~\(A\) denote the linearisation matrix of the chosen patch scheme wrapped around a given microscale system.
Then matrix~\(E\) represents `perturbation' effects of the linear system due to nonlinearity, forcing, spatial variations, etc. 
Consequently, the \(\epsilon\)-pseudospectra indicates whether a chosen patch scheme is likely to induce any artificial instability in real application where such effects are typical. 



\PSfig{heteroDiffPseudoSpectra2}{pseudo-spectrum of a patch scheme applied to heterogeneous diffusion in 2D space (\cref{sec:heteroDiffPseudoSpectra2}).
Plotted are the contours of the reciprocal-resolvent \(1\big/(zI-A)^{-1}\) for \(\epsilon=10^{-1:2}\).
\cref{fig:heteroDiffPseudoSpectra1%
,fig:heteroDiffPseudoSpectra2} use spectral, edgy interpolation: finite order and/or centre-cross interpolation appears similar. }
\cref{fig:heteroDiffPseudoSpectra1%
,fig:heteroDiffPseudoSpectra2} indicate that the patch scheme implemented in the toolbox\footcite{Maclean2020a, Roberts2019b} for heterogeneous diffusion in 1D and 2D are robust.


\paragraph{Wave-like systems}
\PSfig{waveIdealPS}{pseudo-spectrum of a patch scheme applied to ideal wave system in 1D space (\cref{sec:waveIdealPS}): \(h_t=u_x\) and \(u_t=h_x\).
Plotted are the contours of the reciprocal-resolvent \(1\big/(zI-A)^{-1}\) for \(\epsilon=0.1,1\).
Spectra are for the toolbox's staggered patches with spectral interpolation.}
Wave systems are challenging because it is excruciatingly easy for perturbations to tip the patch-wrapped system into unphysical instability.
\cref{fig:waveIdealPS} are \(\epsilon\)-pseudospectra for a staggered patch scheme in 1D ideal waves.  
The spectra indicate the patch-wrapped waves are reasonably robust to perturbations: the \(\epsilon\)-pseudospectra appear to be circles as for normal matrices; and the microscale subpatch waves appear no more sensitive to perturbations than the macroscale waves.


 

%\PSfig{stag2DPS}{pseudo-spectrum of a staggered patch scheme applied to waves in 2D space.}

\PSfig{ww2DPS0}{pseudo-spectrum of a patch scheme applied to weakly dissipative waves in 2D space:  here \(\epsilon=0.1,1,10\).  
The scheme is staggered patches, with centre-node to patch-edge interpolation.
Some sub-patch waves may be a little sensitive.}
However, waves in multiple space dimensions are trickier.
\cref{fig:ww2DPS0} plots the pseudo-spectra for an older patch scheme applied to weakly dissipative waves.
The shape of the pseudospectra appears non-circular, suggesting the matrix is non-normal.
The plot suggests that some of the sub-patch modes may be a little sensitive to perturbations in that a \(\epsilon=1\) contour (labelled~0) pokes out into instability. 


\begin{figure}
\centering
\caption{\label{fig:heteroWave2DPS}%
pseudo-spectrum of a patch scheme applied to heterogeneous waves in 2D space: left, \emph{unscaled}; right, scaled.  Here \(\epsilon=10^{-1:2}\).  
The staggered patch scheme interpolates centre-cross values to the patch edges.
These \(\epsilon\)-pseudospectra suggest that this patch scheme for waves is normal and so robust.}
\begin{tabular}{@{}cc@{}}
\raisebox{0.017\linewidth}{%
\includegraphics[width=0.49\linewidth]{heteroWave2DPSinf}
}&
\includegraphics[width=0.49\linewidth]{heteroWave2DPS0}
\end{tabular}
\end{figure}
\cref{fig:heteroWave2DPS} shows what happens when we change the inter-patch interpolation to that \emph{from the centre-cross values} in a patch to the patch-edges.
The left panel suggests the patch-wrapped wave system is normal.
The right panel expands the domain of the macroscale waves near zero to further indicate the robust properties of this patch-wrapping.


%\PSfig{heteroWave2DPSinf}{\emph{unscaled} pseudo-spectrum of a patch scheme applied to heterogeneous waves in 2D space.}
%
%\PSfig{heteroWave2DPS0}{pseudo-spectrum of a patch scheme applied to heterogeneous waves in 2D space.}

The toolbox \citep{Maclean2020a, Roberts2019b} implements this more robust centre-cross interpolation for multi-D space.

\paragraph{ToDo} explore pseudospectra of implemented polynomial interpolation for bounded domains.



\newpage\appendix
% This file was created by matlab2tikz.
%
\tikzsetnextfilename{heteroDiffPseudoSpectra1}
\definecolor{mycolor1}{rgb}{0.00000,0.44700,0.74100}%
%
\begin{tikzpicture}

\begin{axis}[%
colormap={mymap}{[1pt] rgb(0pt)=(0.19376,0.12032,0.52824); rgb(1pt)=(0.22328,0.21004,0.72516); rgb(2pt)=(0.21176,0.3224,0.7948); rgb(3pt)=(0.14128,0.43824,0.76304); rgb(4pt)=(0.0868,0.53352,0.69872); rgb(5pt)=(0.05632,0.59656,0.58064); rgb(6pt)=(0.22472,0.63712,0.42128); rgb(7pt)=(0.49832,0.62948,0.20572); rgb(8pt)=(0.73472,0.58464,0.1512); rgb(9pt)=(0.78568,0.66488,0.14416); rgb(10pt)=(0.78152,0.78712,0.0644)},
xmin=-9,
xmax=6.28176859901917,
xtick={-7.65720351546678,-6.34217825702207,-4.90131106106349,-3.58890820806613,-2.17670184360978,-1.05501457138561,0,1.05501457138561,2.17670184360978,3.58890820806613,4.90131106106349,6.34217825702207,7.65720351546678},
xticklabels={{-300},{-100},{ -30},{ -10},{  -3},{  -1},{   0},{   1},{   3},{  10},{  30},{ 100},{ 300}},
xticklabel style={rotate=40},
xlabel style={font=\color{white!15!black}},
xlabel={$\Re\lambda$},
ymin=-9,
ymax=9,
ytick={-7.89794518257253,-6.32742449980746,-5.1418917082744,-3.96578497663241,-2.4758056364204,-1.51153787230448,0,1.51153787230448,2.4758056364204,3.96578497663241,5.1418917082744,6.32742449980746,7.89794518257253},
yticklabels={{-50},{-20},{-10},{ -5},{ -2},{ -1},{  0},{  1},{  2},{  5},{ 10},{ 20},{ 50}},
ylabel style={font=\color{white!15!black}},
ylabel={$\Im\lambda$},
axis background/.style={fill=white},
title style={font=\bfseries},
title={spectrum (dots) and pseudo-spectra contours, via psa()},
xmajorgrids,
ymajorgrids,
\extraAxisOptions
]
\addplot [color=mycolor1, draw=none, mark=*, mark options={solid, mycolor1}, forget plot]
  table[row sep=crcr]{%
-8.86958955524442	0\\
-8.86667463413634	0\\
-8.86567179593032	0\\
-8.86958955524442	0\\
-8.86667463413634	0\\
3.26121984706238e-13	0\\
-7.79878467307254	0\\
-2.50230217517409	0\\
-2.50230217517408	0\\
-7.79278236636434	0\\
-1.05409230073313	0\\
-1.05409230073309	0\\
-7.77486133688067	0\\
-7.77486133688067	0\\
-7.79278236636434	0\\
};
\addplot[contour prepared, contour prepared format=matlab] table[row sep=crcr] {%
%
-4	5\\
-1.04832004640937	0\\
-1.05348032851003	-0.0068255412409924\\
-1.0586096280472	0\\
-1.05348032851003	0.0068255412409924\\
-1.04832004640937	0\\
-3	5\\
-1.01176550474976	0\\
-1.05348032851003	-0.055176489266003\\
-1.09494469461116	0\\
-1.05348032851003	0.055176489266003\\
-1.01176550474976	0\\
-2	9\\
-2.47425089414428	-0.0900000000000004\\
-2.50524834541685	-0.172752755608507\\
-2.53377410109748	-0.0900000000000004\\
-2.55789801109811	0\\
-2.53377410109748	0.0900000000000004\\
-2.50524834541685	0.172752755608507\\
-2.47425089414428	0.0900000000000004\\
-2.4502997329057	0\\
-2.47425089414428	-0.0900000000000004\\
-2	13\\
-0.980718418735136	-0.0900000000000004\\
-1.05348032851003	-0.173488410881257\\
-1.12397878631251	-0.0900000000000004\\
-1.12988917150513	-0.0558008906055807\\
-1.13922844930399	0\\
-1.12988917150513	0.0558008906055807\\
-1.12397878631251	0.0900000000000004\\
-1.05348032851003	0.173488410881257\\
-0.980718418735136	0.0900000000000004\\
-0.977071485514937	0.0705598946299685\\
-0.963664558209167	0\\
-0.977071485514937	-0.0705598946299685\\
-0.980718418735136	-0.0900000000000004\\
-2	13\\
0.103211352485611	-0.0900000000000004\\
0.0926523164164046	-0.107879347448654\\
0.0162434734213082	-0.172149718965904\\
-0.060165369573786	-0.150792180647766\\
-0.104579422366867	-0.0900000000000004\\
-0.124107403893418	0\\
-0.104579422366867	0.0900000000000004\\
-0.060165369573786	0.150792180647766\\
0.0162434734213082	0.172149718965904\\
0.0926523164164046	0.107879347448654\\
0.103211352485611	0.0900000000000004\\
0.124943907978803	0\\
0.103211352485611	-0.0900000000000004\\
0	83\\
-2.40424934068532	-1.44\\
-2.42883950242176	-1.47469573887311\\
-2.50524834541685	-1.51174815907887\\
-2.58165718841195	-1.46234621375133\\
-2.59328061742775	-1.44\\
-2.63431180244415	-1.35\\
-2.65806603140705	-1.28568191414318\\
-2.66452405519822	-1.26\\
-2.68449497823266	-1.17\\
-2.70084699412296	-1.08\\
-2.71413106116334	-0.99\\
-2.7248274575172	-0.899999999999999\\
-2.73335390226851	-0.809999999999999\\
-2.73447487440214	-0.795893386742023\\
-2.74098545748931	-0.72\\
-2.74746078113322	-0.63\\
-2.75279093035809	-0.539999999999999\\
-2.75710170540129	-0.45\\
-2.76049596631991	-0.36\\
-2.76305475453402	-0.27\\
-2.76483982011661	-0.180000000000001\\
-2.76589382208823	-0.0900000000000004\\
-2.76624234529643	0\\
-2.76589382208823	0.0900000000000004\\
-2.76483982011661	0.180000000000001\\
-2.76305475453402	0.270000000000001\\
-2.76049596631991	0.359999999999998\\
-2.75710170540129	0.449999999999999\\
-2.75279093035809	0.539999999999999\\
-2.74746078113322	0.63\\
-2.74098545748931	0.72\\
-2.73447487440214	0.795893386742025\\
-2.73335390226851	0.81\\
-2.7248274575172	0.900000000000001\\
-2.71413106116334	0.990000000000001\\
-2.70084699412296	1.08\\
-2.68449497823266	1.17\\
-2.66452405519822	1.26\\
-2.65806603140705	1.28568191414318\\
-2.63431180244415	1.35\\
-2.59328061742775	1.44\\
-2.58165718841195	1.46234621375133\\
-2.50524834541685	1.51174815907887\\
-2.42883950242176	1.47469573887311\\
-2.40424934068532	1.44\\
-2.35243065942666	1.35508190514063\\
-2.35024416678525	1.35\\
-2.3154094875112	1.26\\
-2.28748959436207	1.17\\
-2.27602181643157	1.1255811465959\\
-2.26470037920955	1.08\\
-2.24545134920217	0.990000000000001\\
-2.22962653804687	0.900000000000001\\
-2.2167156354899	0.81\\
-2.20625385323758	0.72\\
-2.19961297343647	0.649521392414169\\
-2.1974579278428	0.63\\
-2.18912871651461	0.539999999999999\\
-2.18214251014403	0.449999999999999\\
-2.17691322689006	0.359999999999998\\
-2.17287468292623	0.270000000000001\\
-2.16993870463129	0.180000000000001\\
-2.16824999452551	0.0900000000000004\\
-2.1677092242159	0\\
-2.16824999452551	-0.0900000000000004\\
-2.16993870463129	-0.180000000000001\\
-2.17287468292623	-0.27\\
-2.17691322689006	-0.36\\
-2.18214251014403	-0.45\\
-2.18912871651461	-0.539999999999999\\
-2.1974579278428	-0.63\\
-2.19961297343647	-0.649521392414169\\
-2.20625385323758	-0.72\\
-2.2167156354899	-0.809999999999999\\
-2.22962653804687	-0.899999999999999\\
-2.24545134920217	-0.99\\
-2.26470037920955	-1.08\\
-2.27602181643157	-1.1255811465959\\
-2.28748959436207	-1.17\\
-2.3154094875112	-1.26\\
-2.35024416678525	-1.35\\
-2.35243065942666	-1.35508190514063\\
-2.40424934068532	-1.44\\
0	147\\
-0.746415925396341	-1.44\\
-0.747844956529649	-1.44129556645367\\
-0.824253799524744	-1.46653175485492\\
-0.900662642519841	-1.49090630343781\\
-0.977071485514937	-1.50473049288932\\
-1.05348032851003	-1.51121031687785\\
-1.12988917150513	-1.50469178166251\\
-1.20629801450022	-1.48801748511948\\
-1.28270685749532	-1.45737836932245\\
-1.31116771740548	-1.44\\
-1.35911570049042	-1.40947823049017\\
-1.42250667742539	-1.35\\
-1.43552454348551	-1.33593545741211\\
-1.49039035582028	-1.26\\
-1.51193338648061	-1.2268736446936\\
-1.54132028361636	-1.17\\
-1.58051399124584	-1.08\\
-1.5883422294757	-1.05900907091154\\
-1.61131684549244	-0.99\\
-1.63647136425238	-0.899999999999999\\
-1.65695550419983	-0.809999999999999\\
-1.6647510724708	-0.769193222225502\\
-1.67393771993868	-0.72\\
-1.68802970890845	-0.63\\
-1.69955312235509	-0.539999999999999\\
-1.70875365674324	-0.45\\
-1.71593720494958	-0.36\\
-1.7213156913551	-0.27\\
-1.72502673998839	-0.180000000000001\\
-1.72722818094964	-0.0900000000000004\\
-1.72795631530508	0\\
-1.72722818094964	0.0900000000000004\\
-1.72502673998839	0.180000000000001\\
-1.7213156913551	0.270000000000001\\
-1.71593720494958	0.359999999999998\\
-1.70875365674324	0.449999999999999\\
-1.69955312235509	0.539999999999999\\
-1.68802970890845	0.63\\
-1.67393771993868	0.72\\
-1.6647510724708	0.769193222225503\\
-1.65695550419983	0.81\\
-1.63647136425238	0.900000000000001\\
-1.61131684549244	0.990000000000001\\
-1.5883422294757	1.05900907091155\\
-1.58051399124584	1.08\\
-1.54132028361636	1.17\\
-1.51193338648061	1.2268736446936\\
-1.49039035582028	1.26\\
-1.43552454348551	1.33593545741211\\
-1.42250667742539	1.35\\
-1.35911570049042	1.40947823049017\\
-1.31116771740548	1.44\\
-1.28270685749532	1.45737836932245\\
-1.20629801450022	1.48801748511948\\
-1.12988917150513	1.50469178166251\\
-1.05348032851003	1.51121031687785\\
-0.977071485514937	1.50473049288932\\
-0.900662642519841	1.49090630343781\\
-0.824253799524744	1.46653175485492\\
-0.747844956529649	1.44129556645367\\
-0.746415925396341	1.44\\
-0.671436113534553	1.37533413195851\\
-0.595027270539456	1.35561225633064\\
-0.518618427544362	1.37054147807861\\
-0.442209584549266	1.38779316151862\\
-0.36580074155417	1.40133671066201\\
-0.314477072657821	1.44\\
-0.289391898559074	1.46895031044461\\
-0.212983055563978	1.48316317039607\\
-0.136574212568882	1.4712146925103\\
-0.060165369573786	1.49106621231464\\
-0.0485700143761497	1.44\\
0.0162434734213082	1.37707559144765\\
0.0813282524910412	1.44\\
0.0926523164164046	1.50310433771661\\
0.169061159411501	1.49870492854968\\
0.245470002406597	1.48435631612925\\
0.321878845401693	1.46236202303491\\
0.393475136179401	1.44\\
0.398287688396789	1.43840612205815\\
0.474696531391885	1.40294741390805\\
0.55110537438698	1.35994642764983\\
0.564116119484592	1.35\\
0.627514217382075	1.30151087138999\\
0.676602853928974	1.26\\
0.703923060377171	1.23284113367579\\
0.757616859267898	1.17\\
0.780331903372268	1.1395922251759\\
0.819757792172295	1.08\\
0.856740746367364	1.01518654329719\\
0.869544685601797	0.990000000000001\\
0.891951239297519	0.900000000000001\\
0.93314958936246	0.822383225559672\\
0.942322962480226	0.81\\
0.969873651424754	0.72\\
0.991126563265445	0.63\\
1.00955843235756	0.541234761650415\\
1.00981496950199	0.539999999999999\\
1.02405238567377	0.449999999999999\\
1.03595414779352	0.359999999999998\\
1.04437593349112	0.270000000000001\\
1.0506956536166	0.180000000000001\\
1.05422271418984	0.0900000000000004\\
1.05502256001646	0\\
1.05422271418984	-0.0900000000000004\\
1.0506956536166	-0.180000000000001\\
1.04437593349112	-0.27\\
1.03595414779352	-0.36\\
1.02405238567377	-0.45\\
1.00981496950199	-0.539999999999999\\
1.00955843235756	-0.541234761650415\\
0.991126563265445	-0.63\\
0.969873651424754	-0.72\\
0.942322962480226	-0.809999999999999\\
0.93314958936246	-0.822383225559671\\
0.891951239297519	-0.899999999999999\\
0.869544685601797	-0.99\\
0.856740746367364	-1.01518654329719\\
0.819757792172295	-1.08\\
0.780331903372268	-1.1395922251759\\
0.757616859267898	-1.17\\
0.703923060377171	-1.23284113367579\\
0.676602853928974	-1.26\\
0.627514217382075	-1.30151087138999\\
0.564116119484592	-1.35\\
0.55110537438698	-1.35994642764983\\
0.474696531391885	-1.40294741390805\\
0.398287688396789	-1.43840612205815\\
0.393475136179401	-1.44\\
0.321878845401693	-1.46236202303491\\
0.245470002406597	-1.48435631612925\\
0.169061159411501	-1.49870492854968\\
0.0926523164164046	-1.50310433771661\\
0.0813282524910412	-1.44\\
0.0162434734213082	-1.37707559144765\\
-0.0485700143761497	-1.44\\
-0.060165369573786	-1.49106621231464\\
-0.136574212568882	-1.4712146925103\\
-0.212983055563978	-1.48316317039607\\
-0.289391898559074	-1.46895031044461\\
-0.314477072657821	-1.44\\
-0.36580074155417	-1.40133671066201\\
-0.442209584549266	-1.38779316151862\\
-0.518618427544362	-1.37054147807861\\
-0.595027270539456	-1.35561225633064\\
-0.671436113534553	-1.37533413195851\\
-0.746415925396341	-1.44\\
0	53\\
-7.77705422714575	-1.08\\
-7.77745851207847	-1.11205949987847\\
-7.77789322330258	-1.08\\
-7.77904851624664	-0.99\\
-7.78012438782565	-0.899999999999999\\
-7.78105446842219	-0.809999999999999\\
-7.78197328577919	-0.72\\
-7.78275649128679	-0.63\\
-7.78351500569521	-0.539999999999999\\
-7.78407105717472	-0.45\\
-7.78452599437995	-0.36\\
-7.7849819754829	-0.27\\
-7.78536145591052	-0.180000000000001\\
-7.78547222306112	-0.0900000000000004\\
-7.7854718547482	0\\
-7.78547222306112	0.0900000000000004\\
-7.78536145591052	0.180000000000001\\
-7.7849819754829	0.270000000000001\\
-7.78452599437995	0.359999999999998\\
-7.78407105717472	0.449999999999999\\
-7.78351500569521	0.539999999999999\\
-7.78275649128679	0.63\\
-7.78197328577919	0.72\\
-7.78105446842219	0.81\\
-7.78012438782565	0.900000000000001\\
-7.77904851624664	0.990000000000001\\
-7.77789322330258	1.08\\
-7.77745851207847	1.11205949987847\\
-7.77705422714575	1.08\\
-7.77597953069734	0.990000000000001\\
-7.77498071319932	0.900000000000001\\
-7.77405548516914	0.81\\
-7.77321275329023	0.72\\
-7.77245374184178	0.63\\
-7.77178469452775	0.539999999999999\\
-7.7712055018151	0.449999999999999\\
-7.77072718340868	0.359999999999998\\
-7.77034981266245	0.270000000000001\\
-7.7700791622652	0.180000000000001\\
-7.76991410580452	0.0900000000000004\\
-7.76985370509612	0\\
-7.76991410580452	-0.0900000000000004\\
-7.7700791622652	-0.180000000000001\\
-7.77034981266245	-0.27\\
-7.77072718340868	-0.36\\
-7.7712055018151	-0.45\\
-7.77178469452775	-0.539999999999999\\
-7.77245374184178	-0.63\\
-7.77321275329023	-0.72\\
-7.77405548516914	-0.809999999999999\\
-7.77498071319932	-0.899999999999999\\
-7.77597953069734	-0.99\\
-7.77705422714575	-1.08\\
1	233\\
-7.77717029025565	-5.13\\
-7.77745851207847	-5.13501217377194\\
-7.7778128948202	-5.13\\
-7.7835779312934	-5.04\\
-7.78880342440032	-4.95\\
-7.79335130478277	-4.86\\
-7.79692186282408	-4.77\\
-7.80170265158596	-4.68\\
-7.80458136763282	-4.59\\
-7.80621343895183	-4.5\\
-7.8090952688285	-4.41\\
-7.81197391136357	-4.32\\
-7.81244561633593	-4.23\\
-7.8164435725233	-4.14\\
-7.8189779961389	-4.05\\
-7.82076198263502	-3.96\\
-7.82062368693399	-3.87\\
-7.82322129868949	-3.78\\
-7.82507037736309	-3.69\\
-7.82576188799548	-3.6\\
-7.82592307199731	-3.51\\
-7.82760319623031	-3.42\\
-7.82701138559685	-3.33\\
-7.82947190893354	-3.24\\
-7.82992633552262	-3.15\\
-7.83177148124715	-3.06\\
-7.83011991262533	-2.97\\
-7.83165648370651	-2.88\\
-7.83151074307953	-2.79\\
-7.8345829344546	-2.7\\
-7.83509924973976	-2.61\\
-7.83401001138429	-2.52\\
-7.83533768238334	-2.43\\
-7.83408267958483	-2.34\\
-7.83497722881693	-2.25\\
-7.83677923907611	-2.16\\
-7.83571060378212	-2.07\\
-7.83687130112996	-1.98\\
-7.8380775374573	-1.89\\
-7.83850773973994	-1.8\\
-7.83889647806494	-1.71\\
-7.83894264528568	-1.62\\
-7.83858348791503	-1.53\\
-7.83946537037476	-1.44\\
-7.83996210779918	-1.35\\
-7.84050797559038	-1.26\\
-7.84026002879436	-1.17\\
-7.84087402737185	-1.08\\
-7.84115951471136	-0.99\\
-7.84141789565695	-0.899999999999999\\
-7.84054522441619	-0.809999999999999\\
-7.84114391593982	-0.72\\
-7.84104530458427	-0.63\\
-7.84172731529873	-0.539999999999999\\
-7.84128843332664	-0.45\\
-7.84093192968269	-0.36\\
-7.84152359186528	-0.27\\
-7.84233266799176	-0.180000000000001\\
-7.84184354228674	-0.0900000000000004\\
-7.84138627920493	0\\
-7.84184354228674	0.0900000000000004\\
-7.84233266799176	0.180000000000001\\
-7.84152359186528	0.270000000000001\\
-7.84093192968269	0.359999999999998\\
-7.84128843332664	0.449999999999999\\
-7.84172731529873	0.539999999999999\\
-7.84104530458427	0.63\\
-7.84114391593982	0.72\\
-7.84054522441619	0.81\\
-7.84141789565695	0.900000000000001\\
-7.84115951471136	0.990000000000001\\
-7.84087402737185	1.08\\
-7.84026002879436	1.17\\
-7.84050797559038	1.26\\
-7.83996210779918	1.35\\
-7.83946537037476	1.44\\
-7.83858348791503	1.53\\
-7.83894264528568	1.62\\
-7.83889647806494	1.71\\
-7.83850773973994	1.8\\
-7.8380775374573	1.89\\
-7.83687130112996	1.98\\
-7.83571060378212	2.07\\
-7.83677923907611	2.16\\
-7.83497722881693	2.25\\
-7.83408267958483	2.34\\
-7.83533768238334	2.43\\
-7.83401001138429	2.52\\
-7.83509924973976	2.61\\
-7.8345829344546	2.7\\
-7.83151074307953	2.79\\
-7.83165648370651	2.88\\
-7.83011991262533	2.97\\
-7.83177148124715	3.06\\
-7.82992633552262	3.15\\
-7.82947190893354	3.24\\
-7.82701138559685	3.33\\
-7.82760319623031	3.42\\
-7.82592307199731	3.51\\
-7.82576188799548	3.6\\
-7.82507037736309	3.69\\
-7.82322129868949	3.78\\
-7.82062368693399	3.87\\
-7.82076198263502	3.96\\
-7.8189779961389	4.05\\
-7.8164435725233	4.14\\
-7.81244561633593	4.23\\
-7.81197391136357	4.32\\
-7.8090952688285	4.41\\
-7.80621343895183	4.5\\
-7.80458136763282	4.59\\
-7.80170265158596	4.68\\
-7.79692186282408	4.77\\
-7.79335130478277	4.86\\
-7.78880342440032	4.95\\
-7.7835779312934	5.04\\
-7.7778128948202	5.13\\
-7.77745851207847	5.13501217377194\\
-7.77717029025565	5.13\\
-7.77246650062381	5.04\\
-7.76808177060205	4.95\\
-7.76454435728777	4.86\\
-7.76072795845181	4.77\\
-7.75753847227164	4.68\\
-7.75495386701214	4.59\\
-7.75316762940072	4.5\\
-7.74951338233241	4.41\\
-7.74735672642244	4.32\\
-7.74565741638767	4.23\\
-7.74342150723373	4.14\\
-7.74169408505178	4.05\\
-7.74016018941722	3.96\\
-7.73857999888415	3.87\\
-7.73715412651887	3.78\\
-7.73589528563531	3.69\\
-7.73465564963634	3.6\\
-7.73356521163905	3.51\\
-7.73252938278	3.42\\
-7.73149954873293	3.33\\
-7.73059864582933	3.24\\
-7.72974973020183	3.15\\
-7.72898121043561	3.06\\
-7.72811795881078	2.97\\
-7.72740616629886	2.88\\
-7.72670502153125	2.79\\
-7.72604141725781	2.7\\
-7.72546281530199	2.61\\
-7.72487990696729	2.52\\
-7.72430867741379	2.43\\
-7.72374254231113	2.34\\
-7.72323672466095	2.25\\
-7.72274925669526	2.16\\
-7.72233653005733	2.07\\
-7.72189223263739	1.98\\
-7.72149297556053	1.89\\
-7.72102683684344	1.8\\
-7.72064622597936	1.71\\
-7.72042859108339	1.62\\
-7.71994181716408	1.53\\
-7.71962492781888	1.44\\
-7.71933127914618	1.35\\
-7.71903414142285	1.26\\
-7.71871889965096	1.17\\
-7.71848157311039	1.08\\
-7.71820546184175	0.990000000000001\\
-7.71801144703443	0.900000000000001\\
-7.71775651372662	0.81\\
-7.71756783293242	0.72\\
-7.71739083833605	0.63\\
-7.7172504962281	0.539999999999999\\
-7.71709912747473	0.449999999999999\\
-7.7170041073184	0.359999999999998\\
-7.71692530140332	0.270000000000001\\
-7.71688246592425	0.180000000000001\\
-7.71684404433646	0.0900000000000004\\
-7.71678990599913	0\\
-7.71684404433646	-0.0900000000000004\\
-7.71688246592425	-0.180000000000001\\
-7.71692530140332	-0.27\\
-7.7170041073184	-0.36\\
-7.71709912747473	-0.45\\
-7.7172504962281	-0.539999999999999\\
-7.71739083833605	-0.63\\
-7.71756783293242	-0.72\\
-7.71775651372662	-0.809999999999999\\
-7.71801144703443	-0.899999999999999\\
-7.71820546184175	-0.99\\
-7.71848157311039	-1.08\\
-7.71871889965096	-1.17\\
-7.71903414142285	-1.26\\
-7.71933127914618	-1.35\\
-7.71962492781888	-1.44\\
-7.71994181716408	-1.53\\
-7.72042859108339	-1.62\\
-7.72064622597936	-1.71\\
-7.72102683684344	-1.8\\
-7.72149297556053	-1.89\\
-7.72189223263739	-1.98\\
-7.72233653005733	-2.07\\
-7.72274925669526	-2.16\\
-7.72323672466095	-2.25\\
-7.72374254231113	-2.34\\
-7.72430867741379	-2.43\\
-7.72487990696729	-2.52\\
-7.72546281530199	-2.61\\
-7.72604141725781	-2.7\\
-7.72670502153125	-2.79\\
-7.72740616629886	-2.88\\
-7.72811795881078	-2.97\\
-7.72898121043561	-3.06\\
-7.72974973020183	-3.15\\
-7.73059864582933	-3.24\\
-7.73149954873293	-3.33\\
-7.73252938278	-3.42\\
-7.73356521163905	-3.51\\
-7.73465564963634	-3.6\\
-7.73589528563531	-3.69\\
-7.73715412651887	-3.78\\
-7.73857999888415	-3.87\\
-7.74016018941722	-3.96\\
-7.74169408505178	-4.05\\
-7.74342150723373	-4.14\\
-7.74565741638767	-4.23\\
-7.74735672642244	-4.32\\
-7.74951338233241	-4.41\\
-7.75316762940072	-4.5\\
-7.75495386701214	-4.59\\
-7.75753847227164	-4.68\\
-7.76072795845181	-4.77\\
-7.76454435728777	-4.86\\
-7.76808177060205	-4.95\\
-7.77246650062381	-5.04\\
-7.77717029025565	-5.13\\
1	495\\
-2.64829350671834	-5.13\\
-2.65806603140705	-5.13646420819005\\
-2.70143529498722	-5.13\\
-2.73447487440214	-5.12502206929363\\
-2.81088371739724	-5.11696559693049\\
-2.88729256039233	-5.11762197076456\\
-2.96370140338743	-5.10301168535266\\
-3.04011024638252	-5.08720148371224\\
-3.11651908937762	-5.06955537077904\\
-3.19292793237272	-5.04890392677416\\
-3.21433874470747	-5.04\\
-3.26933677536781	-5.01540859217513\\
-3.34574561836291	-4.97629528136866\\
-3.38717279441171	-4.95\\
-3.422154461358	-4.92595093548039\\
-3.49767259719221	-4.86\\
-3.4985633043531	-4.85910299099822\\
-3.5749721473482	-4.77524835750951\\
-3.57869638834688	-4.77\\
-3.63829774238899	-4.68\\
-3.65138099034329	-4.65787762924974\\
-3.68767403603372	-4.59\\
-3.72778983333839	-4.50456278201117\\
-3.7296938481358	-4.5\\
-3.7607218680288	-4.41\\
-3.79099278575373	-4.32\\
-3.80419867633348	-4.27101559424185\\
-3.81444718364969	-4.23\\
-3.83324859454478	-4.14\\
-3.8532284676735	-4.05\\
-3.8689586526922	-3.96\\
-3.88060751932858	-3.87895060088013\\
-3.8818577613911	-3.87\\
-3.89472944634084	-3.78\\
-3.9049208359317	-3.69\\
-3.91251177903404	-3.6\\
-3.92138598529158	-3.51\\
-3.92851451694838	-3.42\\
-3.93565819308155	-3.33\\
-3.93620443268036	-3.24\\
-3.94683866202561	-3.15\\
-3.94107763193901	-3.06\\
-3.95405506419202	-2.97\\
-3.95701636232367	-2.91783026613366\\
-3.95913708511934	-2.88\\
-3.96217551542995	-2.79\\
-3.96516681198479	-2.7\\
-3.96626871449171	-2.61\\
-3.96982522308207	-2.52\\
-3.97216933293763	-2.43\\
-3.97385666512471	-2.34\\
-3.97544111668269	-2.25\\
-3.9771800938599	-2.16\\
-3.97870567415167	-2.07\\
-3.97998629732368	-1.98\\
-3.98075000768598	-1.89\\
-3.98179474328471	-1.8\\
-3.98251151418331	-1.71\\
-3.98289166875416	-1.62\\
-3.9835239723721	-1.53\\
-3.98436735144659	-1.44\\
-3.9854973544555	-1.35\\
-3.98601021166023	-1.26\\
-3.98618443127106	-1.17\\
-3.98675974304734	-1.08\\
-3.98446156656829	-0.99\\
-3.98688940495922	-0.899999999999999\\
-3.9870572552073	-0.809999999999999\\
-3.98754536746756	-0.72\\
-3.98771679307826	-0.63\\
-3.98810669930741	-0.539999999999999\\
-3.97885477108338	-0.45\\
-3.98850130630268	-0.36\\
-3.98866797883778	-0.27\\
-3.98804800688027	-0.180000000000001\\
-3.98787450383599	-0.0900000000000004\\
-3.98751487001921	0\\
-3.98787450383599	0.0900000000000004\\
-3.98804800688027	0.180000000000001\\
-3.98866797883778	0.270000000000001\\
-3.98850130630268	0.359999999999998\\
-3.97885477108338	0.449999999999999\\
-3.98810669930741	0.539999999999999\\
-3.98771679307826	0.63\\
-3.98754536746756	0.72\\
-3.9870572552073	0.81\\
-3.98688940495922	0.900000000000001\\
-3.98446156656829	0.990000000000001\\
-3.98675974304734	1.08\\
-3.98618443127106	1.17\\
-3.98601021166023	1.26\\
-3.9854973544555	1.35\\
-3.98436735144659	1.44\\
-3.9835239723721	1.53\\
-3.98289166875416	1.62\\
-3.98251151418331	1.71\\
-3.98179474328471	1.8\\
-3.98075000768598	1.89\\
-3.97998629732368	1.98\\
-3.97870567415167	2.07\\
-3.9771800938599	2.16\\
-3.97544111668269	2.25\\
-3.97385666512471	2.34\\
-3.97216933293763	2.43\\
-3.96982522308207	2.52\\
-3.96626871449171	2.61\\
-3.96516681198479	2.7\\
-3.96217551542995	2.79\\
-3.95913708511934	2.88\\
-3.95701636232367	2.91783026613366\\
-3.95405506419202	2.97\\
-3.94107763193901	3.06\\
-3.94683866202561	3.15\\
-3.93620443268036	3.24\\
-3.93565819308155	3.33\\
-3.92851451694838	3.42\\
-3.92138598529158	3.51\\
-3.91251177903404	3.6\\
-3.9049208359317	3.69\\
-3.89472944634084	3.78\\
-3.8818577613911	3.87\\
-3.88060751932858	3.87895060088013\\
-3.8689586526922	3.96\\
-3.8532284676735	4.05\\
-3.83324859454478	4.14\\
-3.81444718364969	4.23\\
-3.80419867633348	4.27101559424185\\
-3.79099278575373	4.32\\
-3.7607218680288	4.41\\
-3.7296938481358	4.5\\
-3.72778983333839	4.50456278201117\\
-3.68767403603372	4.59\\
-3.65138099034329	4.65787762924974\\
-3.63829774238899	4.68\\
-3.57869638834688	4.77\\
-3.5749721473482	4.77524835750951\\
-3.4985633043531	4.85910299099822\\
-3.49767259719221	4.86\\
-3.422154461358	4.92595093548039\\
-3.38717279441171	4.95\\
-3.34574561836291	4.97629528136866\\
-3.26933677536781	5.01540859217512\\
-3.21433874470747	5.04\\
-3.19292793237272	5.04890392677416\\
-3.11651908937762	5.06955537077904\\
-3.04011024638252	5.08720148371224\\
-2.96370140338743	5.10301168535266\\
-2.88729256039233	5.11762197076456\\
-2.81088371739724	5.11696559693049\\
-2.73447487440214	5.12502206929363\\
-2.70143529498722	5.13\\
-2.65806603140705	5.13646420819005\\
-2.64829350671834	5.13\\
-2.58165718841195	5.10125519208612\\
-2.50524834541685	5.1288378071726\\
-2.49395929238355	5.13\\
-2.42883950242176	5.13619292786391\\
-2.35243065942666	5.13208402368418\\
-2.33907467176259	5.13\\
-2.27602181643157	5.12189068760572\\
-2.21540070270632	5.13\\
-2.19961297343647	5.13235413517632\\
-2.17749387121004	5.13\\
-2.12320413044137	5.12409147952797\\
-2.04679528744628	5.12280632517143\\
-1.97038644445118	5.12265320586065\\
-1.89397760145609	5.12415393340121\\
-1.81756875846099	5.12862036790632\\
-1.79153511475828	5.13\\
-1.74115991546589	5.13263873184553\\
-1.6647510724708	5.13268451794396\\
-1.5883422294757	5.13610937405713\\
-1.51193338648061	5.13603249987735\\
-1.43552454348551	5.13921961411292\\
-1.35911570049042	5.1396627949217\\
-1.28270685749532	5.13734529520194\\
-1.20629801450022	5.14034022715406\\
-1.12988917150513	5.14102462666072\\
-1.05348032851003	5.14115053769972\\
-0.977071485514937	5.14101338256269\\
-0.900662642519841	5.14123684255508\\
-0.824253799524744	5.13983022855765\\
-0.747844956529649	5.14063890432855\\
-0.671436113534553	5.14036120530823\\
-0.595027270539456	5.13966926002149\\
-0.518618427544362	5.13951141855029\\
-0.442209584549266	5.13914121719193\\
-0.36580074155417	5.13886457203281\\
-0.289391898559074	5.13864459575852\\
-0.212983055563978	5.14058192411619\\
-0.136574212568882	5.13891246558808\\
-0.060165369573786	5.13507107223349\\
0.0162434734213082	5.13701696934291\\
0.0926523164164046	5.13813732440326\\
0.169061159411501	5.13428075601426\\
0.245470002406597	5.13005566708485\\
0.321878845401693	5.13248893587719\\
0.398287688396789	5.13017136365525\\
0.474696531391885	5.13127497775174\\
0.533956438029938	5.13\\
0.55110537438698	5.12961928058491\\
0.627514217382075	5.12797834183109\\
0.703923060377171	5.12468129813818\\
0.780331903372268	5.1201658893968\\
0.856740746367364	5.11565253924537\\
0.919854843050414	5.13\\
0.93314958936246	5.1328532672158\\
0.952754056858207	5.13\\
1.00955843235756	5.12057136179307\\
1.08596727535265	5.11216668371572\\
1.16237611834775	5.11107820899753\\
1.23878496134284	5.1170716384062\\
1.31519380433794	5.09884911187128\\
1.39160264733303	5.0920683755805\\
1.46801149032813	5.098824837649\\
1.54442033332323	5.0996507355693\\
1.62082917631832	5.09716528993099\\
1.69723801931342	5.09058553282236\\
1.77364686230851	5.08972661657798\\
1.85005570530361	5.07074963981603\\
1.9264645482987	5.07535080345637\\
2.0028733912938	5.05510744354461\\
2.03638376694845	5.04\\
2.0792822342889	5.01575441888775\\
2.13156371909761	5.04\\
2.15569107728399	5.04808495710299\\
2.19894021174444	5.04\\
2.23209992027909	5.02378877056353\\
2.30850876327418	4.99973140689269\\
2.37727902114351	4.95\\
2.38491760626928	4.94425138041303\\
2.46014327478489	4.95\\
2.46132644926438	4.95016661031122\\
2.46159584460695	4.95\\
2.53773529225947	4.91258606818569\\
2.61414413525457	4.89571272262143\\
2.63832951122164	4.86\\
2.69055297824966	4.84475220006038\\
2.7637649980696	4.77\\
2.76696182124476	4.76761817357663\\
2.84337066423985	4.7430902679297\\
2.89433198294949	4.77\\
2.91977950723495	4.78176086254853\\
2.92874035597715	4.77\\
2.99618835023005	4.70462730961733\\
3.04983657919685	4.68\\
3.05283469163474	4.59\\
3.07259719322514	4.57114610537693\\
3.14900603622024	4.53362679445037\\
3.17897178984511	4.5\\
3.14900603622024	4.43948874108388\\
3.13303922012578	4.41\\
3.14900603622024	4.39384643231771\\
3.22541487921533	4.34250757363087\\
3.24210765340459	4.32\\
3.30182372221043	4.23170795446115\\
3.30326758583801	4.23\\
3.3357411751356	4.14\\
3.36774858789737	4.05\\
3.37823256520553	4.03523988415826\\
3.40666610657522	3.96\\
3.38980213057332	3.87\\
3.41108770605934	3.78\\
3.37823256520553	3.72352850285427\\
3.36278693132767	3.69\\
3.37823256520553	3.67242232742487\\
3.4188039451247	3.6\\
3.43898592765329	3.51\\
3.44491655459452	3.42\\
3.45464140820062	3.39878605453533\\
3.48394083553759	3.33\\
3.45464140820062	3.25867381179632\\
3.44882204515515	3.24\\
3.45464140820062	3.18627616162963\\
3.49278558747383	3.15\\
3.45464140820062	3.13505507728379\\
3.43108304645254	3.06\\
3.45464140820062	3.02283258497866\\
3.50087763611353	2.97\\
3.52378933731521	2.88\\
3.52178360811142	2.79\\
3.51588783671307	2.7\\
3.49113157528946	2.61\\
3.52210016656762	2.52\\
3.50097133460823	2.43\\
3.49620715619392	2.34\\
3.53105025119572	2.26984365419576\\
3.54026950765731	2.25\\
3.53105025119572	2.22347946773073\\
3.50865251366341	2.16\\
3.53105025119572	2.09201228949785\\
3.54684730401077	2.07\\
3.55846154380202	1.98\\
3.54710915837609	1.89\\
3.53105025119572	1.85207874676337\\
3.50738095791379	1.8\\
3.53105025119572	1.74464987251579\\
3.54756763312885	1.71\\
3.55086750347467	1.62\\
3.5478094335181	1.53\\
3.54032995035871	1.44\\
3.53105025119572	1.36199575795037\\
3.52929345016759	1.35\\
3.53105025119572	1.34341607026988\\
3.5430102752172	1.26\\
3.55133356742326	1.17\\
3.53105025119572	1.14470306210721\\
3.50697799596437	1.08\\
3.50273771516571	0.990000000000001\\
3.53105025119572	0.923799086630539\\
3.54234606800543	0.900000000000001\\
3.55572227369121	0.81\\
3.53174150766906	0.72\\
3.57031826579242	0.63\\
3.56092170519229	0.539999999999999\\
3.56140908941894	0.449999999999999\\
3.57872836060374	0.359999999999998\\
3.53105025119572	0.307731485518274\\
3.51406076234931	0.270000000000001\\
3.53105025119572	0.233925398655912\\
3.5741646421715	0.180000000000001\\
3.55890425486816	0.0900000000000004\\
3.53105025119572	0.0156977857734062\\
3.52453951316401	0\\
3.53105025119572	-0.0156977857734062\\
3.55890425486816	-0.0900000000000004\\
3.5741646421715	-0.180000000000001\\
3.53105025119572	-0.233925398655911\\
3.51406076234931	-0.27\\
3.53105025119572	-0.307731485518273\\
3.57872836060374	-0.36\\
3.56140908941894	-0.45\\
3.56092170519229	-0.539999999999999\\
3.57031826579242	-0.63\\
3.53174150766906	-0.72\\
3.55572227369121	-0.809999999999999\\
3.54234606800543	-0.899999999999999\\
3.53105025119572	-0.923799086630537\\
3.50273771516571	-0.99\\
3.50697799596437	-1.08\\
3.53105025119572	-1.14470306210721\\
3.55133356742326	-1.17\\
3.5430102752172	-1.26\\
3.53105025119572	-1.34341607026988\\
3.52929345016759	-1.35\\
3.53105025119572	-1.36199575795037\\
3.54032995035871	-1.44\\
3.5478094335181	-1.53\\
3.55086750347467	-1.62\\
3.54756763312885	-1.71\\
3.53105025119572	-1.74464987251579\\
3.50738095791379	-1.8\\
3.53105025119572	-1.85207874676337\\
3.54710915837609	-1.89\\
3.55846154380202	-1.98\\
3.54684730401077	-2.07\\
3.53105025119572	-2.09201228949785\\
3.50865251366341	-2.16\\
3.53105025119572	-2.22347946773073\\
3.54026950765731	-2.25\\
3.53105025119572	-2.26984365419576\\
3.49620715619392	-2.34\\
3.50097133460823	-2.43\\
3.52210016656762	-2.52\\
3.49113157528946	-2.61\\
3.51588783671307	-2.7\\
3.52178360811142	-2.79\\
3.52378933731521	-2.88\\
3.50087763611353	-2.97\\
3.45464140820062	-3.02283258497866\\
3.43108304645254	-3.06\\
3.45464140820062	-3.13505507728379\\
3.49278558747383	-3.15\\
3.45464140820062	-3.18627616162963\\
3.44882204515515	-3.24\\
3.45464140820062	-3.25867381179632\\
3.48394083553759	-3.33\\
3.45464140820062	-3.39878605453533\\
3.44491655459452	-3.42\\
3.43898592765329	-3.51\\
3.4188039451247	-3.6\\
3.37823256520553	-3.67242232742487\\
3.36278693132767	-3.69\\
3.37823256520553	-3.72352850285427\\
3.41108770605934	-3.78\\
3.38980213057332	-3.87\\
3.40666610657522	-3.96\\
3.37823256520553	-4.03523988415826\\
3.36774858789737	-4.05\\
3.3357411751356	-4.14\\
3.30326758583801	-4.23\\
3.30182372221043	-4.23170795446115\\
3.24210765340459	-4.32\\
3.22541487921533	-4.34250757363087\\
3.14900603622024	-4.39384643231771\\
3.13303922012578	-4.41\\
3.14900603622024	-4.43948874108387\\
3.17897178984511	-4.5\\
3.14900603622024	-4.53362679445037\\
3.07259719322514	-4.57114610537693\\
3.05283469163474	-4.59\\
3.04983657919685	-4.68\\
2.99618835023005	-4.70462730961733\\
2.92874035597715	-4.77\\
2.91977950723495	-4.78176086254853\\
2.89433198294949	-4.77\\
2.84337066423985	-4.7430902679297\\
2.76696182124476	-4.76761817357663\\
2.7637649980696	-4.77\\
2.69055297824966	-4.84475220006038\\
2.63832951122164	-4.86\\
2.61414413525457	-4.89571272262143\\
2.53773529225947	-4.91258606818569\\
2.46159584460695	-4.95\\
2.46132644926438	-4.95016661031122\\
2.46014327478489	-4.95\\
2.38491760626928	-4.94425138041303\\
2.37727902114351	-4.95\\
2.30850876327418	-4.99973140689269\\
2.23209992027909	-5.02378877056353\\
2.19894021174444	-5.04\\
2.15569107728399	-5.04808495710299\\
2.13156371909761	-5.04\\
2.0792822342889	-5.01575441888776\\
2.03638376694845	-5.04\\
2.0028733912938	-5.05510744354462\\
1.9264645482987	-5.07535080345638\\
1.85005570530361	-5.07074963981604\\
1.77364686230851	-5.08972661657798\\
1.69723801931342	-5.09058553282236\\
1.62082917631832	-5.09716528993099\\
1.54442033332323	-5.0996507355693\\
1.46801149032813	-5.098824837649\\
1.39160264733303	-5.0920683755805\\
1.31519380433794	-5.09884911187128\\
1.23878496134284	-5.1170716384062\\
1.16237611834775	-5.11107820899753\\
1.08596727535265	-5.11216668371572\\
1.00955843235756	-5.12057136179307\\
0.952754056858207	-5.13\\
0.93314958936246	-5.1328532672158\\
0.919854843050414	-5.13\\
0.856740746367364	-5.11565253924538\\
0.780331903372268	-5.1201658893968\\
0.703923060377171	-5.12468129813818\\
0.627514217382075	-5.12797834183109\\
0.55110537438698	-5.12961928058491\\
0.533956438029938	-5.13\\
0.474696531391885	-5.13127497775174\\
0.398287688396789	-5.13017136365525\\
0.321878845401693	-5.13248893587719\\
0.245470002406597	-5.13005566708485\\
0.169061159411501	-5.13428075601426\\
0.0926523164164046	-5.13813732440326\\
0.0162434734213082	-5.13701696934291\\
-0.060165369573786	-5.13507107223349\\
-0.136574212568882	-5.13891246558808\\
-0.212983055563978	-5.14058192411619\\
-0.289391898559074	-5.13864459575852\\
-0.36580074155417	-5.13886457203281\\
-0.442209584549266	-5.13914121719193\\
-0.518618427544362	-5.13951141855029\\
-0.595027270539456	-5.13966926002149\\
-0.671436113534553	-5.14036120530823\\
-0.747844956529649	-5.14063890432855\\
-0.824253799524744	-5.13983022855765\\
-0.900662642519841	-5.14123684255508\\
-0.977071485514937	-5.14101338256269\\
-1.05348032851003	-5.14115053769972\\
-1.12988917150513	-5.14102462666072\\
-1.20629801450022	-5.14034022715406\\
-1.28270685749532	-5.13734529520194\\
-1.35911570049042	-5.1396627949217\\
-1.43552454348551	-5.13921961411292\\
-1.51193338648061	-5.13603249987735\\
-1.5883422294757	-5.13610937405713\\
-1.6647510724708	-5.13268451794396\\
-1.74115991546589	-5.13263873184553\\
-1.79153511475828	-5.13\\
-1.81756875846099	-5.12862036790632\\
-1.89397760145609	-5.12415393340121\\
-1.97038644445118	-5.12265320586065\\
-2.04679528744628	-5.12280632517143\\
-2.12320413044137	-5.12409147952797\\
-2.17749387121004	-5.13\\
-2.19961297343647	-5.13235413517632\\
-2.21540070270632	-5.13\\
-2.27602181643157	-5.12189068760572\\
-2.33907467176259	-5.13\\
-2.35243065942666	-5.13208402368418\\
-2.42883950242176	-5.13619292786391\\
-2.49395929238355	-5.13\\
-2.50524834541685	-5.1288378071726\\
-2.58165718841195	-5.10125519208612\\
-2.64829350671834	-5.13\\
2	28\\
-8.89177309298091	-9\\
-8.92359115700491	-8.94532440973957\\
-8.92836052993452	-8.91\\
-8.9376112923391	-8.82\\
-8.9467743871481	-8.73\\
-8.95426492148047	-8.64\\
-8.96108324855106	-8.55\\
-8.9668171103512	-8.46\\
-8.97117250554961	-8.37\\
-8.97530597895563	-8.28\\
-8.97950407733043	-8.19\\
-8.98214700724065	-8.1\\
-8.98571776513872	-8.01\\
-8.98807520672688	-7.92\\
-8.98986509020299	-7.83\\
-8.99265509325082	-7.74\\
-8.99264894821631	-7.65\\
-8.99427772092819	-7.56\\
-8.9958102218444	-7.47\\
-8.9967663791572	-7.38\\
-8.99792090719315	-7.29\\
-8.99741948236214	-7.2\\
-8.99884368753359	-7.11\\
-8.9993068783906	-7.02\\
-8.99951320986859	-6.93\\
-8.99961299492066	-6.84\\
-8.99984591603747	-6.75\\
-9	-6.73764911943134\\
2	207\\
-7.88521213576294	-9\\
-7.93027619806866	-8.91193636927365\\
-7.9309058844232	-8.91\\
-7.95962812694168	-8.82\\
-7.98207301751827	-8.73\\
-8.00042853216388	-8.64\\
-8.00668504106375	-8.60756589513709\\
-8.01586276766799	-8.55\\
-8.02287477071349	-8.46\\
-8.03450637711273	-8.37\\
-8.04400905486429	-8.28\\
-8.04965501017376	-8.19\\
-8.05706102260233	-8.1\\
-8.06247342674484	-8.01\\
-8.06806431424868	-7.92\\
-8.07093908613559	-7.83\\
-8.07744290704365	-7.74\\
-8.07763992678242	-7.65\\
-8.08193956757526	-7.56\\
-8.08309388405885	-7.51602164546219\\
-8.08437567538907	-7.47\\
-8.08531293466097	-7.38\\
-8.08936942125938	-7.29\\
-8.08852705670537	-7.2\\
-8.09265871628703	-7.11\\
-8.0939534323316	-7.02\\
-8.09469791861148	-6.93\\
-8.09528070260181	-6.84\\
-8.09826921690878	-6.75\\
-8.09817688567518	-6.66\\
-8.10058877784846	-6.57\\
-8.09896098207741	-6.48\\
-8.10193740235511	-6.39\\
-8.10084685650898	-6.3\\
-8.10445362211958	-6.21\\
-8.10321645545206	-6.12\\
-8.10279878911078	-6.03\\
-8.10375546758717	-5.94\\
-8.10393220214989	-5.85\\
-8.10499278644423	-5.76\\
-8.1043230151827	-5.67\\
-8.10366450821207	-5.58\\
-8.10411126375969	-5.49\\
-8.10575233847876	-5.4\\
-8.1049556577738	-5.31\\
-8.10407509904725	-5.22\\
-8.10709890642556	-5.13\\
-8.10724466226367	-5.04\\
-8.10522933312041	-4.95\\
-8.10662897624477	-4.86\\
-8.10720389223139	-4.77\\
-8.10838316438896	-4.68\\
-8.10527879113807	-4.59\\
-8.10725878770458	-4.5\\
-8.10470063313061	-4.41\\
-8.10822980608972	-4.32\\
-8.10595505779905	-4.23\\
-8.10484975998955	-4.14\\
-8.10764672777039	-4.05\\
-8.1080870296097	-3.96\\
-8.10717937336863	-3.87\\
-8.10641960161942	-3.78\\
-8.10866303435182	-3.69\\
-8.10617352374751	-3.6\\
-8.10520146518515	-3.51\\
-8.10516637913637	-3.42\\
-8.1065393667062	-3.33\\
-8.10637681554273	-3.24\\
-8.10630751644906	-3.15\\
-8.10774897819736	-3.06\\
-8.10654510566172	-2.97\\
-8.10643244764618	-2.88\\
-8.10647353664855	-2.79\\
-8.10666197835882	-2.7\\
-8.10739244111362	-2.61\\
-8.1074390598054	-2.52\\
-8.10648213560915	-2.43\\
-8.10764943623845	-2.34\\
-8.10431684944137	-2.25\\
-8.10509940495974	-2.16\\
-8.10685825233372	-2.07\\
-8.10582918867276	-1.98\\
-8.10740716776762	-1.89\\
-8.10641346696913	-1.8\\
-8.10346206724909	-1.71\\
-8.10534664260622	-1.62\\
-8.1056726919945	-1.53\\
-8.10545998309995	-1.44\\
-8.10552301931032	-1.35\\
-8.10786427095884	-1.26\\
-8.10463120534043	-1.17\\
-8.10866739135679	-1.08\\
-8.10772467671148	-0.99\\
-8.10612731395517	-0.899999999999999\\
-8.10997821454274	-0.809999999999999\\
-8.10662239927914	-0.72\\
-8.10704550082627	-0.63\\
-8.1055359572499	-0.539999999999999\\
-8.10702288197029	-0.45\\
-8.10916801012395	-0.36\\
-8.10583798151782	-0.27\\
-8.10653426309158	-0.180000000000001\\
-8.10580989085261	-0.0900000000000004\\
-8.10778853461329	0\\
-8.10580989085261	0.0900000000000004\\
-8.10653426309158	0.180000000000001\\
-8.10583798151782	0.270000000000001\\
-8.10916801012395	0.359999999999998\\
-8.10702288197029	0.449999999999999\\
-8.1055359572499	0.539999999999999\\
-8.10704550082627	0.63\\
-8.10662239927914	0.72\\
-8.10997821454274	0.81\\
-8.10612731395517	0.900000000000001\\
-8.10772467671148	0.990000000000001\\
-8.10866739135679	1.08\\
-8.10463120534043	1.17\\
-8.10786427095884	1.26\\
-8.10552301931032	1.35\\
-8.10545998309995	1.44\\
-8.1056726919945	1.53\\
-8.10534664260622	1.62\\
-8.10346206724909	1.71\\
-8.10641346696913	1.8\\
-8.10740716776762	1.89\\
-8.10582918867276	1.98\\
-8.10685825233372	2.07\\
-8.10509940495974	2.16\\
-8.10431684944137	2.25\\
-8.10764943623845	2.34\\
-8.10648213560915	2.43\\
-8.1074390598054	2.52\\
-8.10739244111362	2.61\\
-8.10666197835882	2.7\\
-8.10647353664855	2.79\\
-8.10643244764618	2.88\\
-8.10654510566172	2.97\\
-8.10774897819736	3.06\\
-8.10630751644906	3.15\\
-8.10637681554273	3.24\\
-8.1065393667062	3.33\\
-8.10516637913637	3.42\\
-8.10520146518515	3.51\\
-8.10617352374751	3.6\\
-8.10866303435182	3.69\\
-8.10641960161942	3.78\\
-8.10717937336863	3.87\\
-8.1080870296097	3.96\\
-8.10764672777039	4.05\\
-8.10484975998955	4.14\\
-8.10595505779905	4.23\\
-8.10822980608972	4.32\\
-8.10470063313061	4.41\\
-8.10725878770458	4.5\\
-8.10527879113807	4.59\\
-8.10838316438896	4.68\\
-8.10720389223139	4.77\\
-8.10662897624477	4.86\\
-8.10522933312041	4.95\\
-8.10724466226367	5.04\\
-8.10709890642556	5.13\\
-8.10407509904725	5.22\\
-8.1049556577738	5.31\\
-8.10575233847876	5.4\\
-8.10411126375969	5.49\\
-8.10366450821207	5.58\\
-8.1043230151827	5.67\\
-8.10499278644423	5.76\\
-8.10393220214989	5.85\\
-8.10375546758717	5.94\\
-8.10279878911078	6.03\\
-8.10321645545206	6.12\\
-8.10445362211958	6.21\\
-8.10084685650898	6.3\\
-8.10193740235511	6.39\\
-8.09896098207741	6.48\\
-8.10058877784846	6.57\\
-8.09817688567518	6.66\\
-8.09826921690878	6.75\\
-8.09528070260181	6.84\\
-8.09469791861148	6.93\\
-8.0939534323316	7.02\\
-8.09265871628703	7.11\\
-8.08852705670537	7.2\\
-8.08936942125938	7.29\\
-8.08531293466097	7.38\\
-8.08437567538907	7.47\\
-8.08309388405885	7.51602164546219\\
-8.08193956757526	7.56\\
-8.07763992678242	7.65\\
-8.07744290704365	7.74\\
-8.07093908613559	7.83\\
-8.06806431424868	7.92\\
-8.06247342674484	8.01\\
-8.05706102260233	8.1\\
-8.04965501017376	8.19\\
-8.04400905486429	8.28\\
-8.03450637711273	8.37\\
-8.02287477071349	8.46\\
-8.01586276766799	8.55\\
-8.00668504106375	8.60756589513709\\
-8.00042853216388	8.64\\
-7.98207301751827	8.73\\
-7.95962812694168	8.82\\
-7.9309058844232	8.91\\
-7.93027619806866	8.91193636927365\\
-7.88521213576294	9\\
2	245\\
-5.0120605793886	-9\\
-5.02674016425502	-8.99708753295565\\
-5.10314900725011	-8.98110063745369\\
-5.17955785024521	-8.97014544710919\\
-5.2559666932403	-8.94912899750232\\
-5.3323755362354	-8.92174582193893\\
-5.37980744096959	-8.91\\
-5.40878437923049	-8.90306316532083\\
-5.48519322222559	-8.87737717852137\\
-5.56160206522069	-8.83625932588613\\
-5.59169435852877	-8.82\\
-5.63801090821578	-8.79272140239271\\
-5.71441975121088	-8.73423514086678\\
-5.72382000601758	-8.73\\
-5.79082859420597	-8.69351136457862\\
-5.86316366363377	-8.64\\
-5.86723743720107	-8.63588032009316\\
-5.91900988649119	-8.55\\
-5.94364628019617	-8.51615249472953\\
-5.97897496639782	-8.46\\
-6.02005512319126	-8.37513383545536\\
-6.02246573261501	-8.37\\
-6.09192977290566	-8.28\\
-6.09646396618636	-8.24610355068079\\
-6.10380683028996	-8.19\\
-6.14240082587843	-8.1\\
-6.17287280918145	-8.02762026306519\\
-6.17863716218935	-8.01\\
-6.1995861211145	-7.92\\
-6.22211742416305	-7.83\\
-6.22469127163645	-7.74\\
-6.23877835522292	-7.65\\
-6.24928165217655	-7.6177453739461\\
-6.26757583336104	-7.56\\
-6.25820515337296	-7.47\\
-6.28230397353483	-7.38\\
-6.29886743364299	-7.29\\
-6.30224253770435	-7.2\\
-6.31231411557341	-7.11\\
-6.32569049517164	-7.03277313715051\\
-6.32766408303747	-7.02\\
-6.32569049517164	-6.95851867637609\\
-6.32454624110021	-6.93\\
-6.32569049517164	-6.90575665147595\\
-6.32852838813453	-6.84\\
-6.32569049517164	-6.8235428007192\\
-6.31360642082644	-6.75\\
-6.32569049517164	-6.70340489845874\\
-6.33767876095879	-6.66\\
-6.34323153712115	-6.57\\
-6.32972740133784	-6.48\\
-6.35622576173588	-6.39\\
-6.33511431326685	-6.3\\
-6.34167206924632	-6.21\\
-6.35296740373965	-6.12\\
-6.34435527458252	-6.03\\
-6.34073963863541	-5.94\\
-6.35766990660787	-5.85\\
-6.36614176995265	-5.76\\
-6.35832190178961	-5.67\\
-6.35518609181261	-5.58\\
-6.35210780682119	-5.49\\
-6.37118313615816	-5.4\\
-6.36952732570226	-5.31\\
-6.36755729483494	-5.22\\
-6.36782054114924	-5.13\\
-6.37545052126343	-5.04\\
-6.35699769932359	-4.95\\
-6.36537980678454	-4.86\\
-6.36052362604711	-4.77\\
-6.36075572663355	-4.68\\
-6.36648222271316	-4.59\\
-6.35761439624437	-4.5\\
-6.35995979085283	-4.41\\
-6.37369940748277	-4.32\\
-6.36520767531067	-4.23\\
-6.37699090591027	-4.14\\
-6.37096139764781	-4.05\\
-6.36495116781764	-3.96\\
-6.35915840589698	-3.87\\
-6.37437305603268	-3.78\\
-6.37034327032684	-3.69\\
-6.37224165779393	-3.6\\
-6.36282920521954	-3.51\\
-6.37994680139053	-3.42\\
-6.36537221130809	-3.33\\
-6.36287689285313	-3.24\\
-6.37538839076078	-3.15\\
-6.35971402772241	-3.06\\
-6.37692911204429	-2.97\\
-6.38413811522655	-2.88\\
-6.37422803297451	-2.79\\
-6.37357458110751	-2.7\\
-6.36824985196144	-2.61\\
-6.36974101223604	-2.52\\
-6.36855427858256	-2.43\\
-6.37047917089679	-2.34\\
-6.37723564998988	-2.25\\
-6.36977297719927	-2.16\\
-6.37372004176794	-2.07\\
-6.36502528441179	-1.98\\
-6.36515978807276	-1.89\\
-6.3668138965306	-1.8\\
-6.36455152621406	-1.71\\
-6.37814076529681	-1.62\\
-6.36838709662252	-1.53\\
-6.36239372060553	-1.44\\
-6.36905732542223	-1.35\\
-6.3517765675883	-1.26\\
-6.35786594301304	-1.17\\
-6.36403296683484	-1.08\\
-6.36103799579432	-0.99\\
-6.36975293831412	-0.899999999999999\\
-6.38305650611118	-0.809999999999999\\
-6.35753391856678	-0.72\\
-6.36374204468608	-0.63\\
-6.35860156045991	-0.539999999999999\\
-6.35705333510069	-0.45\\
-6.36769098142276	-0.36\\
-6.36517692197779	-0.27\\
-6.36179361109354	-0.180000000000001\\
-6.36794572836066	-0.0900000000000004\\
-6.36585882228355	0\\
-6.36794572836066	0.0900000000000004\\
-6.36179361109354	0.180000000000001\\
-6.36517692197779	0.270000000000001\\
-6.36769098142276	0.359999999999998\\
-6.35705333510069	0.449999999999999\\
-6.35860156045991	0.539999999999999\\
-6.36374204468608	0.63\\
-6.35753391856678	0.72\\
-6.38305650611118	0.81\\
-6.36975293831412	0.900000000000001\\
-6.36103799579432	0.990000000000001\\
-6.36403296683484	1.08\\
-6.35786594301304	1.17\\
-6.3517765675883	1.26\\
-6.36905732542223	1.35\\
-6.36239372060553	1.44\\
-6.36838709662252	1.53\\
-6.37814076529681	1.62\\
-6.36455152621406	1.71\\
-6.3668138965306	1.8\\
-6.36515978807276	1.89\\
-6.36502528441179	1.98\\
-6.37372004176794	2.07\\
-6.36977297719927	2.16\\
-6.37723564998988	2.25\\
-6.37047917089679	2.34\\
-6.36855427858256	2.43\\
-6.36974101223604	2.52\\
-6.36824985196144	2.61\\
-6.37357458110751	2.7\\
-6.37422803297451	2.79\\
-6.38413811522655	2.88\\
-6.37692911204429	2.97\\
-6.35971402772241	3.06\\
-6.37538839076078	3.15\\
-6.36287689285313	3.24\\
-6.36537221130809	3.33\\
-6.37994680139053	3.42\\
-6.36282920521954	3.51\\
-6.37224165779393	3.6\\
-6.37034327032684	3.69\\
-6.37437305603268	3.78\\
-6.35915840589698	3.87\\
-6.36495116781764	3.96\\
-6.37096139764781	4.05\\
-6.37699090591027	4.14\\
-6.36520767531067	4.23\\
-6.37369940748277	4.32\\
-6.35995979085283	4.41\\
-6.35761439624437	4.5\\
-6.36648222271316	4.59\\
-6.36075572663355	4.68\\
-6.36052362604711	4.77\\
-6.36537980678454	4.86\\
-6.35699769932359	4.95\\
-6.37545052126343	5.04\\
-6.36782054114924	5.13\\
-6.36755729483494	5.22\\
-6.36952732570226	5.31\\
-6.37118313615816	5.4\\
-6.35210780682119	5.49\\
-6.35518609181261	5.58\\
-6.35832190178961	5.67\\
-6.36614176995265	5.76\\
-6.35766990660787	5.85\\
-6.34073963863541	5.94\\
-6.34435527458252	6.03\\
-6.35296740373965	6.12\\
-6.34167206924632	6.21\\
-6.33511431326685	6.3\\
-6.35622576173588	6.39\\
-6.32972740133784	6.48\\
-6.34323153712115	6.57\\
-6.33767876095879	6.66\\
-6.32569049517164	6.70340489845874\\
-6.31360642082644	6.75\\
-6.32569049517164	6.8235428007192\\
-6.32852838813453	6.84\\
-6.32569049517164	6.90575665147595\\
-6.32454624110021	6.93\\
-6.32569049517164	6.95851867637609\\
-6.32766408303747	7.02\\
-6.32569049517164	7.03277313715051\\
-6.31231411557341	7.11\\
-6.30224253770435	7.2\\
-6.29886743364299	7.29\\
-6.28230397353483	7.38\\
-6.25820515337296	7.47\\
-6.26757583336104	7.56\\
-6.24928165217655	7.6177453739461\\
-6.23877835522292	7.65\\
-6.22469127163645	7.74\\
-6.22211742416305	7.83\\
-6.1995861211145	7.92\\
-6.17863716218935	8.01\\
-6.17287280918145	8.02762026306519\\
-6.14240082587843	8.1\\
-6.10380683028996	8.19\\
-6.09646396618636	8.24610355068079\\
-6.09192977290566	8.28\\
-6.02246573261501	8.37\\
-6.02005512319126	8.37513383545536\\
-5.97897496639782	8.46\\
-5.94364628019617	8.51615249472953\\
-5.91900988649119	8.55\\
-5.86723743720107	8.63588032009316\\
-5.86316366363377	8.64\\
-5.79082859420597	8.69351136457862\\
-5.72382000601758	8.73\\
-5.71441975121088	8.73423514086678\\
-5.63801090821578	8.79272140239271\\
-5.59169435852877	8.82\\
-5.56160206522069	8.83625932588613\\
-5.48519322222559	8.87737717852137\\
-5.40878437923049	8.90306316532083\\
-5.37980744096959	8.91\\
-5.3323755362354	8.92174582193893\\
-5.2559666932403	8.94912899750232\\
-5.17955785024521	8.97014544710919\\
-5.10314900725011	8.98110063745369\\
-5.02674016425502	8.99708753295565\\
-5.0120605793886	9\\
2	3\\
4.86897045653931	-9\\
4.83000058211235	-8.99880544764354\\
4.82426493991513	-9\\
2	44\\
6.28176859901917	-6.70138904936238\\
6.27746128056697	-6.75\\
6.27019195518326	-6.84\\
6.27009972308914	-6.93\\
6.26054836650623	-7.02\\
6.24489045421782	-7.11\\
6.24600778749966	-7.2\\
6.24752016749408	-7.29\\
6.23313906673205	-7.38\\
6.22503101454883	-7.47\\
6.20901753864773	-7.56\\
6.20535975602407	-7.57985641137035\\
6.194937506706	-7.65\\
6.18058979160897	-7.74\\
6.16166378173301	-7.83\\
6.14374934795796	-7.92\\
6.12895091302898	-7.96244782775443\\
6.11255539129354	-8.01\\
6.0882286492363	-8.1\\
6.05965311260525	-8.19\\
6.05254207003388	-8.20233330380839\\
6.01411732780688	-8.28\\
5.98106691794973	-8.37\\
5.97613322703879	-8.3779405257131\\
5.92006296505097	-8.46\\
5.89972438404369	-8.49125555004638\\
5.85315330527102	-8.55\\
5.82331554104859	-8.5789936100198\\
5.76621254274647	-8.64\\
5.7469066980535	-8.66152896571738\\
5.6704978550584	-8.7264067893415\\
5.66619355570424	-8.73\\
5.59408901206331	-8.77276521304005\\
5.51768016906821	-8.81044087861943\\
5.50278218245989	-8.82\\
5.44127132607311	-8.854830545963\\
5.36486248307802	-8.87800904378039\\
5.2936464846197	-8.91\\
5.28845364008292	-8.91217176235376\\
5.21204479708783	-8.93396771545706\\
5.13563595409273	-8.94999228401567\\
5.05922711109764	-8.96337582728465\\
4.98281826810254	-8.98169968870573\\
4.9115965110588	-9\\
2	3\\
6.28176859901917	-6.38306780165292\\
6.28080167091171	-6.39\\
6.28176859901917	-6.39797334650047\\
2	3\\
6.28176859901917	6.39797334650047\\
6.28080167091171	6.39\\
6.28176859901917	6.38306780165292\\
2	28\\
-9	6.73764911943134\\
-8.99984591603747	6.75\\
-8.99961299492066	6.84\\
-8.99951320986859	6.93\\
-8.9993068783906	7.02\\
-8.99884368753359	7.11\\
-8.99741948236214	7.2\\
-8.99792090719315	7.29\\
-8.9967663791572	7.38\\
-8.9958102218444	7.47\\
-8.99427772092819	7.56\\
-8.99264894821631	7.65\\
-8.99265509325082	7.74\\
-8.98986509020299	7.83\\
-8.98807520672688	7.92\\
-8.98571776513872	8.01\\
-8.98214700724065	8.1\\
-8.97950407733043	8.19\\
-8.97530597895563	8.28\\
-8.97117250554961	8.37\\
-8.9668171103512	8.46\\
-8.96108324855106	8.55\\
-8.95426492148047	8.64\\
-8.9467743871481	8.73\\
-8.9376112923391	8.82\\
-8.92836052993452	8.91\\
-8.92359115700491	8.94532440973957\\
-8.89177309298091	9\\
2	203\\
-8.82970530312222	9\\
-8.81058671985328	8.91\\
-8.79433721130047	8.82\\
-8.78166073290398	8.73\\
-8.77077347101471	8.64245998117584\\
-8.77054148243246	8.64\\
-8.76368507932219	8.55\\
-8.75764969702012	8.46\\
-8.75254708265505	8.37\\
-8.74727089467898	8.28\\
-8.74255973177398	8.19\\
-8.73824734905225	8.1\\
-8.73518029248	8.01\\
-8.73219722977638	7.92\\
-8.72905752787689	7.83\\
-8.72752719006162	7.74\\
-8.72505575388453	7.65\\
-8.72453495302625	7.56\\
-8.72173671432673	7.47\\
-8.72099864085235	7.38\\
-8.71977569788869	7.29\\
-8.71860612134861	7.2\\
-8.71774962992946	7.11\\
-8.71644662069289	7.02\\
-8.71530043401281	6.93\\
-8.71514865573871	6.84\\
-8.71431962922212	6.75\\
-8.71400172664282	6.66\\
-8.71229322596017	6.57\\
-8.7128324493588	6.48\\
-8.71183559809847	6.39\\
-8.71169985994209	6.3\\
-8.71186330472356	6.21\\
-8.7104014802729	6.12\\
-8.71067438659123	6.03\\
-8.71107907351115	5.94\\
-8.71039017716438	5.85\\
-8.70958328775868	5.76\\
-8.710152799712	5.67\\
-8.71000677448955	5.58\\
-8.70946644594797	5.49\\
-8.710099815554	5.4\\
-8.71037769844505	5.31\\
-8.7100446513961	5.22\\
-8.70931748742469	5.13\\
-8.7086148155892	5.04\\
-8.70959506151885	4.95\\
-8.7088727365622	4.86\\
-8.70882585513242	4.77\\
-8.70878863084069	4.68\\
-8.70919381025445	4.59\\
-8.70926227243122	4.5\\
-8.7096835843271	4.41\\
-8.70940865199095	4.32\\
-8.70897805812243	4.23\\
-8.7094481584779	4.14\\
-8.70937657643245	4.05\\
-8.70873661984156	3.96\\
-8.70845556513762	3.87\\
-8.70885418255048	3.78\\
-8.70875848092121	3.69\\
-8.7082010885471	3.6\\
-8.70888379727632	3.51\\
-8.70838175135277	3.42\\
-8.70810035675421	3.33\\
-8.70860372458164	3.24\\
-8.70863396605788	3.15\\
-8.70902089208991	3.06\\
-8.70944543321849	2.97\\
-8.70843301874628	2.88\\
-8.70797939769949	2.79\\
-8.70806492222606	2.7\\
-8.70873981667449	2.61\\
-8.7092165780064	2.52\\
-8.70972392756517	2.43\\
-8.70866469455515	2.34\\
-8.70939035201222	2.25\\
-8.70912878966012	2.16\\
-8.70857077575643	2.07\\
-8.70874950767434	1.98\\
-8.7085156474531	1.89\\
-8.70877837381691	1.8\\
-8.7084627299057	1.71\\
-8.70828508009257	1.62\\
-8.70861189046517	1.53\\
-8.70777691799056	1.44\\
-8.70933756395545	1.35\\
-8.70934855722968	1.26\\
-8.70951619673345	1.17\\
-8.70870971784107	1.08\\
-8.70903977243626	0.990000000000001\\
-8.70926411569806	0.900000000000001\\
-8.70921211081159	0.81\\
-8.70881126257178	0.72\\
-8.70829949428466	0.63\\
-8.70825845791962	0.539999999999999\\
-8.70880603173936	0.449999999999999\\
-8.70916818561871	0.359999999999998\\
-8.70946973365588	0.270000000000001\\
-8.70827610307757	0.180000000000001\\
-8.70808152094816	0.0900000000000004\\
-8.70844635556463	0\\
-8.70808152094816	-0.0900000000000004\\
-8.70827610307757	-0.180000000000001\\
-8.70946973365588	-0.27\\
-8.70916818561871	-0.36\\
-8.70880603173936	-0.45\\
-8.70825845791962	-0.539999999999999\\
-8.70829949428466	-0.63\\
-8.70881126257178	-0.72\\
-8.70921211081159	-0.809999999999999\\
-8.70926411569806	-0.899999999999999\\
-8.70903977243626	-0.99\\
-8.70870971784107	-1.08\\
-8.70951619673345	-1.17\\
-8.70934855722968	-1.26\\
-8.70933756395545	-1.35\\
-8.70777691799056	-1.44\\
-8.70861189046517	-1.53\\
-8.70828508009257	-1.62\\
-8.7084627299057	-1.71\\
-8.70877837381691	-1.8\\
-8.7085156474531	-1.89\\
-8.70874950767434	-1.98\\
-8.70857077575643	-2.07\\
-8.70912878966012	-2.16\\
-8.70939035201222	-2.25\\
-8.70866469455515	-2.34\\
-8.70972392756517	-2.43\\
-8.7092165780064	-2.52\\
-8.70873981667449	-2.61\\
-8.70806492222606	-2.7\\
-8.70797939769949	-2.79\\
-8.70843301874628	-2.88\\
-8.70944543321849	-2.97\\
-8.70902089208991	-3.06\\
-8.70863396605788	-3.15\\
-8.70860372458164	-3.24\\
-8.70810035675421	-3.33\\
-8.70838175135277	-3.42\\
-8.70888379727632	-3.51\\
-8.7082010885471	-3.6\\
-8.70875848092121	-3.69\\
-8.70885418255048	-3.78\\
-8.70845556513762	-3.87\\
-8.70873661984156	-3.96\\
-8.70937657643245	-4.05\\
-8.7094481584779	-4.14\\
-8.70897805812243	-4.23\\
-8.70940865199095	-4.32\\
-8.7096835843271	-4.41\\
-8.70926227243122	-4.5\\
-8.70919381025445	-4.59\\
-8.70878863084069	-4.68\\
-8.70882585513242	-4.77\\
-8.7088727365622	-4.86\\
-8.70959506151885	-4.95\\
-8.7086148155892	-5.04\\
-8.70931748742469	-5.13\\
-8.7100446513961	-5.22\\
-8.71037769844505	-5.31\\
-8.710099815554	-5.4\\
-8.70946644594797	-5.49\\
-8.71000677448955	-5.58\\
-8.710152799712	-5.67\\
-8.70958328775868	-5.76\\
-8.71039017716438	-5.85\\
-8.71107907351115	-5.94\\
-8.71067438659123	-6.03\\
-8.7104014802729	-6.12\\
-8.71186330472356	-6.21\\
-8.71169985994209	-6.3\\
-8.71183559809847	-6.39\\
-8.7128324493588	-6.48\\
-8.71229322596017	-6.57\\
-8.71400172664282	-6.66\\
-8.71431962922212	-6.75\\
-8.71514865573871	-6.84\\
-8.71530043401281	-6.93\\
-8.71644662069289	-7.02\\
-8.71774962992946	-7.11\\
-8.71860612134861	-7.2\\
-8.71977569788869	-7.29\\
-8.72099864085235	-7.38\\
-8.72173671432673	-7.47\\
-8.72453495302625	-7.56\\
-8.72505575388453	-7.65\\
-8.72752719006162	-7.74\\
-8.72905752787689	-7.83\\
-8.73219722977638	-7.92\\
-8.73518029248	-8.01\\
-8.73824734905225	-8.1\\
-8.74255973177398	-8.19\\
-8.74727089467898	-8.28\\
-8.75254708265505	-8.37\\
-8.75764969702012	-8.46\\
-8.76368507932219	-8.55\\
-8.77054148243246	-8.64\\
-8.77077347101471	-8.64245998117584\\
-8.78166073290398	-8.73\\
-8.79433721130047	-8.82\\
-8.81058671985328	-8.91\\
-8.82970530312222	-9\\
2	209\\
-7.67553558986003	9\\
-7.62464082608827	8.91986098969417\\
-7.61900743654856	8.91\\
-7.58444411753657	8.82\\
-7.56254397182998	8.73\\
-7.54823198309318	8.69378006313019\\
-7.52965591996071	8.64\\
-7.51573434589937	8.55\\
-7.48940279889197	8.46\\
-7.48099763833178	8.37\\
-7.47258427032942	8.28\\
-7.47182314009808	8.27637362708879\\
-7.45171146440833	8.19\\
-7.44240645732753	8.1\\
-7.42863285376732	8.01\\
-7.42702327403572	7.92\\
-7.41546701207684	7.83\\
-7.41139852514668	7.74\\
-7.40854388817841	7.65\\
-7.40635596703571	7.56\\
-7.39812537836639	7.47\\
-7.39541429710299	7.44719497160555\\
-7.38499127299253	7.38\\
-7.38226160055221	7.29\\
-7.39127925587153	7.2\\
-7.38273746992671	7.11\\
-7.38812156761629	7.02\\
-7.37209550825976	6.93\\
-7.36807686762895	6.84\\
-7.37521344886706	6.75\\
-7.37129240573687	6.66\\
-7.37371487213982	6.57\\
-7.36709137460877	6.48\\
-7.36699198941904	6.39\\
-7.36142650273749	6.3\\
-7.3585353394418	6.21\\
-7.36723601464467	6.12\\
-7.36179796190186	6.03\\
-7.35950311102766	5.94\\
-7.35511138243209	5.85\\
-7.36034032116746	5.76\\
-7.36468784484031	5.67\\
-7.36271712328512	5.58\\
-7.3575606164878	5.49\\
-7.35894673603088	5.4\\
-7.36723681504915	5.31\\
-7.35048992511661	5.22\\
-7.35567024498281	5.13\\
-7.35517944677655	5.04\\
-7.35938242045564	4.95\\
-7.36213072223567	4.86\\
-7.35112922465367	4.77\\
-7.35311490752528	4.68\\
-7.35132241832837	4.59\\
-7.36071385669226	4.5\\
-7.35045774474939	4.41\\
-7.35210391621615	4.32\\
-7.35307187810086	4.23\\
-7.35992720389436	4.14\\
-7.35564905988725	4.05\\
-7.35068529204585	3.96\\
-7.35719740421219	3.87\\
-7.35787487063914	3.78\\
-7.36317291533063	3.69\\
-7.35636042219786	3.6\\
-7.35682785023139	3.51\\
-7.35419486536294	3.42\\
-7.35345672315595	3.33\\
-7.35197285772541	3.24\\
-7.35586916126956	3.15\\
-7.3547886497566	3.06\\
-7.35278899468669	2.97\\
-7.34827398480398	2.88\\
-7.35423152848458	2.79\\
-7.3570050238566	2.7\\
-7.35643255103681	2.61\\
-7.35628212954657	2.52\\
-7.35626542787491	2.43\\
-7.35022612837648	2.34\\
-7.35525039176984	2.25\\
-7.35684711670065	2.16\\
-7.35397256658497	2.07\\
-7.36607729056358	1.98\\
-7.36158172923075	1.89\\
-7.34730227532781	1.8\\
-7.3519992535108	1.71\\
-7.35954479379235	1.62\\
-7.36605552620923	1.53\\
-7.35113069358905	1.44\\
-7.35554442164807	1.35\\
-7.34939948429302	1.26\\
-7.35355144847499	1.17\\
-7.36344995029677	1.08\\
-7.35604338322582	0.990000000000001\\
-7.35647924088924	0.900000000000001\\
-7.34737709228761	0.81\\
-7.35609319949138	0.72\\
-7.35623652925431	0.63\\
-7.35328942019428	0.539999999999999\\
-7.35232610371031	0.449999999999999\\
-7.35353573929215	0.359999999999998\\
-7.35709096318621	0.270000000000001\\
-7.35165512719088	0.180000000000001\\
-7.36406208587991	0.0900000000000004\\
-7.350944283243	0\\
-7.36406208587991	-0.0900000000000004\\
-7.35165512719088	-0.180000000000001\\
-7.35709096318621	-0.27\\
-7.35353573929215	-0.36\\
-7.35232610371031	-0.45\\
-7.35328942019428	-0.539999999999999\\
-7.35623652925431	-0.63\\
-7.35609319949138	-0.72\\
-7.34737709228761	-0.809999999999999\\
-7.35647924088924	-0.899999999999999\\
-7.35604338322582	-0.99\\
-7.36344995029677	-1.08\\
-7.35355144847499	-1.17\\
-7.34939948429302	-1.26\\
-7.35554442164807	-1.35\\
-7.35113069358905	-1.44\\
-7.36605552620923	-1.53\\
-7.35954479379235	-1.62\\
-7.3519992535108	-1.71\\
-7.34730227532781	-1.8\\
-7.36158172923075	-1.89\\
-7.36607729056358	-1.98\\
-7.35397256658497	-2.07\\
-7.35684711670065	-2.16\\
-7.35525039176984	-2.25\\
-7.35022612837648	-2.34\\
-7.35626542787491	-2.43\\
-7.35628212954657	-2.52\\
-7.35643255103681	-2.61\\
-7.3570050238566	-2.7\\
-7.35423152848458	-2.79\\
-7.34827398480398	-2.88\\
-7.35278899468669	-2.97\\
-7.3547886497566	-3.06\\
-7.35586916126956	-3.15\\
-7.35197285772541	-3.24\\
-7.35345672315595	-3.33\\
-7.35419486536294	-3.42\\
-7.35682785023139	-3.51\\
-7.35636042219786	-3.6\\
-7.36317291533063	-3.69\\
-7.35787487063914	-3.78\\
-7.35719740421219	-3.87\\
-7.35068529204585	-3.96\\
-7.35564905988725	-4.05\\
-7.35992720389436	-4.14\\
-7.35307187810086	-4.23\\
-7.35210391621615	-4.32\\
-7.35045774474939	-4.41\\
-7.36071385669226	-4.5\\
-7.35132241832837	-4.59\\
-7.35311490752528	-4.68\\
-7.35112922465367	-4.77\\
-7.36213072223567	-4.86\\
-7.35938242045564	-4.95\\
-7.35517944677655	-5.04\\
-7.35567024498281	-5.13\\
-7.35048992511661	-5.22\\
-7.36723681504915	-5.31\\
-7.35894673603088	-5.4\\
-7.3575606164878	-5.49\\
-7.36271712328512	-5.58\\
-7.36468784484031	-5.67\\
-7.36034032116746	-5.76\\
-7.35511138243209	-5.85\\
-7.35950311102766	-5.94\\
-7.36179796190186	-6.03\\
-7.36723601464467	-6.12\\
-7.3585353394418	-6.21\\
-7.36142650273749	-6.3\\
-7.36699198941904	-6.39\\
-7.36709137460877	-6.48\\
-7.37371487213982	-6.57\\
-7.37129240573687	-6.66\\
-7.37521344886706	-6.75\\
-7.36807686762895	-6.84\\
-7.37209550825976	-6.93\\
-7.38812156761629	-7.02\\
-7.38273746992671	-7.11\\
-7.39127925587153	-7.2\\
-7.38226160055221	-7.29\\
-7.38499127299253	-7.38\\
-7.39541429710299	-7.44719497160555\\
-7.39812537836639	-7.47\\
-7.40635596703571	-7.56\\
-7.40854388817841	-7.65\\
-7.41139852514668	-7.74\\
-7.41546701207684	-7.83\\
-7.42702327403572	-7.92\\
-7.42863285376732	-8.01\\
-7.44240645732753	-8.1\\
-7.45171146440833	-8.19\\
-7.47182314009808	-8.27637362708879\\
-7.47258427032942	-8.28\\
-7.48099763833178	-8.37\\
-7.48940279889197	-8.46\\
-7.51573434589937	-8.55\\
-7.52965591996071	-8.64\\
-7.54823198309318	-8.69378006313019\\
-7.56254397182998	-8.73\\
-7.58444411753657	-8.82\\
-7.61900743654856	-8.91\\
-7.62464082608827	-8.91986098969417\\
-7.67553558986003	-9\\
2	3\\
4.82426493991513	9\\
4.83000058211235	8.99880544764354\\
4.86897045653931	9\\
2	44\\
4.9115965110588	9\\
4.98281826810254	8.98169968870573\\
5.05922711109764	8.96337582728465\\
5.13563595409273	8.94999228401567\\
5.21204479708783	8.93396771545706\\
5.28845364008292	8.91217176235376\\
5.2936464846197	8.91\\
5.36486248307802	8.87800904378039\\
5.44127132607311	8.854830545963\\
5.50278218245989	8.82\\
5.51768016906821	8.81044087861943\\
5.59408901206331	8.77276521304005\\
5.66619355570424	8.73\\
5.6704978550584	8.7264067893415\\
5.7469066980535	8.66152896571738\\
5.76621254274647	8.64\\
5.82331554104859	8.5789936100198\\
5.85315330527102	8.55\\
5.89972438404369	8.49125555004638\\
5.92006296505097	8.46\\
5.97613322703879	8.3779405257131\\
5.98106691794973	8.37\\
6.01411732780688	8.28\\
6.05254207003388	8.20233330380839\\
6.05965311260525	8.19\\
6.0882286492363	8.1\\
6.11255539129354	8.01\\
6.12895091302898	7.96244782775443\\
6.14374934795796	7.92\\
6.16166378173301	7.83\\
6.18058979160897	7.74\\
6.194937506706	7.65\\
6.20535975602407	7.57985641137035\\
6.20901753864773	7.56\\
6.22503101454883	7.47\\
6.23313906673205	7.38\\
6.24752016749408	7.29\\
6.24600778749966	7.2\\
6.24489045421782	7.11\\
6.26054836650623	7.02\\
6.27009972308914	6.93\\
6.27019195518326	6.84\\
6.27746128056697	6.75\\
6.28176859901917	6.70138904936238\\
};
\end{axis}
\end{tikzpicture}%
% This file was created by matlab2tikz.
%
\tikzsetnextfilename{heteroDiffPseudoSpectra2}
\definecolor{mycolor1}{rgb}{0.00000,0.44700,0.74100}%
%
\begin{tikzpicture}

\begin{axis}[%
colormap={mymap}{[1pt] rgb(0pt)=(0.19376,0.12032,0.52824); rgb(1pt)=(0.22328,0.21004,0.72516); rgb(2pt)=(0.21176,0.3224,0.7948); rgb(3pt)=(0.14128,0.43824,0.76304); rgb(4pt)=(0.0868,0.53352,0.69872); rgb(5pt)=(0.05632,0.59656,0.58064); rgb(6pt)=(0.22472,0.63712,0.42128); rgb(7pt)=(0.49832,0.62948,0.20572); rgb(8pt)=(0.73472,0.58464,0.1512); rgb(9pt)=(0.78568,0.66488,0.14416); rgb(10pt)=(0.78152,0.78712,0.0644)},
xmin=-9,
xmax=6.81801636637699,
xtick={-8.83036431126592,-7.71522800386549,-6.4931487918959,-5.37803481445811,-4.15620950005667,-3.04331926771432,-1.84579774033744,-0.894630340670458,0,0.894630340670458,1.84579774033744,3.04331926771432,4.15620950005667,5.37803481445811,6.4931487918959,7.71522800386549,8.83036431126592},
xticklabels={{-3000},{-1000},{ -300},{ -100},{  -30},{  -10},{   -3},{   -1},{    0},{    1},{    3},{   10},{   30},{  100},{  300},{ 1000},{ 3000}},
xticklabel style={rotate=40},
xlabel style={font=\color{white!15!black}},
xlabel={$\Re\lambda$},
ymin=-9,
ymax=9,
ytick={-8.57116439544667,-7.09917822562273,-5.48632967280292,-4.01727891773653,-2.43651214229402,-1.1809407067048,0,1.1809407067048,2.43651214229402,4.01727891773653,5.48632967280292,7.09917822562273,8.57116439544667},
yticklabels={{-300},{-100},{ -30},{ -10},{  -3},{  -1},{   0},{   1},{   3},{  10},{  30},{ 100},{ 300}},
ylabel style={font=\color{white!15!black}},
ylabel={$\Im\lambda$},
axis background/.style={fill=white},
title style={font=\bfseries},
title={spectrum (dots) and pseudo-spectra contours, via psa()},
xmajorgrids,
ymajorgrids,
\extraAxisOptions
]
\addplot [color=mycolor1, draw=none, mark=*, mark options={solid, mycolor1}, forget plot]
  table[row sep=crcr]{%
-8.87152871961259	0\\
-8.87156400025211	0\\
-8.87166756981568	0\\
-8.87156400025211	0\\
-8.87166756981568	0\\
-8.87218336409861	0\\
-8.87424461887465	0\\
-8.87420767767065	0\\
-8.87424461887465	0\\
-8.87420767767065	0\\
-8.87413206774059	0\\
-8.8740872209193	0\\
-8.8740872209193	0\\
-8.87413206774059	0\\
-8.87411297484805	0\\
-8.87411297484805	0\\
-8.87233200056156	0\\
-8.87218336409861	0\\
-8.8722136051962	0\\
-8.87222369647079	0\\
-8.87221360519621	0\\
-8.87222369647079	0\\
-8.87231247575125	0\\
-8.87231247575125	0\\
-8.87233200056156	0\\
1.15396803192811e-12	0\\
-8.36789369818123	0\\
-8.3645388391031	0\\
-8.36461558628039	0\\
-8.36468188942724	0\\
-8.36616191391527	0\\
-8.36615280009774	0\\
-8.36615280009773	0\\
-3.27462168080938	0\\
-3.27462168080953	0\\
-3.14920245459328	0\\
-3.14920245459336	0\\
-0.836014534515648	0\\
-0.836014534516337	0\\
-8.29386482485004	0\\
-8.30804087702598	0\\
-8.36789369818122	0\\
-8.29386482485004	0\\
-8.36616191391526	0\\
-8.29637710119613	0\\
-8.29637710119614	0\\
-8.36468188942724	0\\
-8.36461558628039	0\\
-8.3645388391031	0\\
-8.36363691210822	0\\
-8.36363691210822	0\\
-8.36355990598333	0\\
-8.36350431311972	0\\
-8.29838198946269	0\\
-8.29838198946269	0\\
-8.30202194183385	0\\
-8.36341382935075	0\\
-8.36314785476571	0\\
-8.36314066860937	0\\
-8.30753511876111	0\\
-8.30753511876111	0\\
-8.30016229697758	0\\
-8.30016229697758	0\\
-2.98643623754532	0\\
-2.9864362375452	0\\
-1.48919870259133	0\\
-1.48919870259086	0\\
-2.5435182549147	0\\
-2.54351825491457	0\\
-8.30202194183385	0\\
-8.29969248131193	0\\
-2.90475433100759	0\\
-2.9047543310075	0\\
-2.8385436536364	0\\
-2.83854365363635	0\\
-2.41631089591293	0\\
-2.41631089591308	0\\
-1.7765929258807	0\\
-1.77659292588087	0\\
-1.89469629322192	0\\
-1.89469629322183	0\\
-2.04094887844735	0\\
-2.04094887844684	0\\
-8.30416889292272	0\\
-8.29969248131193	0\\
-8.30416889292272	0\\
-8.36313805531427	0\\
-8.36313805531426	0\\
-8.30604568125471	0\\
-8.30608204191762	0\\
-8.30608204191761	0\\
-8.30604568125471	0\\
-8.30556436588888	0\\
-8.3055449250309	0\\
-8.30556436588888	0\\
-8.30554492503089	0\\
-8.36350431311972	0\\
-8.36341382935076	0\\
-7.90082575726871	0\\
-7.92402784756169	0\\
-7.92385640813185	0\\
-7.92354743709201	0\\
-7.92311539346452	0\\
-8.36355990598332	0\\
-7.92258962640959	0\\
-8.3631478547657	0\\
-7.92258962640959	0\\
-7.90139062432076	0\\
-7.90124637137994	0\\
-6.47631651270725	0\\
-6.47717863596875	0\\
-7.92402784756169	0\\
-7.90082575726871	0\\
-7.90124637137995	0\\
-7.92385640813185	0\\
-7.923547437092	0\\
-7.92311539346452	0\\
-6.51049933217707	0\\
-6.47631651270725	0\\
-6.50376194103333	0\\
-6.47717863596875	0\\
-6.51198125679972	0\\
-6.48372658772653	0\\
-6.49616916722592	0\\
-6.51198125679972	0\\
-6.48603042961819	0\\
-6.48603042961819	0\\
-6.48372658772654	0\\
-6.49118925443831	0\\
-6.49118925443832	0\\
-7.7823918965551	0\\
-7.90814727136408	0\\
-7.90775630455697	0\\
-7.78281193681874	0\\
-7.78249920589591	0\\
-7.90723901030544	0\\
-7.90723901030544	0\\
-6.51049933217707	0\\
-6.4935636245496	0\\
-6.4935636245496	0\\
-7.90775630455697	0\\
-7.90814727136408	0\\
-7.90806638906126	0\\
-7.90806638906128	0\\
-7.9079983635274	0\\
-7.9079983635274	0\\
-6.50017866987781	0\\
-6.4992517213398	0\\
-6.50017866987781	0\\
-6.4992517213398	0\\
-6.49689770745089	0\\
-6.49689770745089	0\\
-7.7824992058959	0\\
-6.50376194103333	0\\
-7.78281193681874	0\\
-7.77336989421052	0\\
-7.77320939169815	0\\
-7.77343744389863	0\\
-7.77343744389864	0\\
-7.7731190736042	0\\
-7.77317482444542	0\\
-7.77317482444543	0\\
-7.7731190736042	0\\
-7.75024091154033	0\\
-7.77320939169815	0\\
-7.48433355671552	0\\
-7.77336989421052	0\\
-7.48496647088547	0\\
-7.4940971995629	0\\
-7.75024091154033	0\\
-7.75018507332153	0\\
-7.4940971995629	0\\
-6.99217694411382	0\\
-7.49192482198036	0\\
-6.99217694411382	0\\
-6.99039572535003	0\\
-6.98346378141207	0\\
-6.98346378141207	0\\
-6.98035897032818	0\\
-6.98035897032818	0\\
-6.98156518076348	0\\
-6.98156518076348	0\\
-7.48496647088547	0\\
-6.99020203191929	0\\
-6.98121156213578	0\\
-6.98048999061891	0\\
-7.75013972184693	0\\
-7.75009412112849	0\\
-7.75011012173042	0\\
-6.99085144591996	0\\
-7.4889965622271	0\\
-7.49192482198037	0\\
-7.49183747401546	0\\
-7.49183747401546	0\\
-7.75013972184692	0\\
-7.48792599895005	0\\
-7.48601749974936	0\\
-7.48636310788257	0\\
-7.49070793884637	0\\
-7.49067646129276	0\\
-6.98121156213578	0\\
-6.98048999061891	0\\
-6.99085144591997	0\\
-6.98789399659895	0\\
-6.98808222508056	0\\
-6.98918591634851	0\\
-6.9891859163485	0\\
-6.9886265393914	0\\
-6.9886265393914	0\\
-7.48692025406585	0\\
-7.75011012173042	0\\
-7.49067646129275	0\\
-7.48681143384292	0\\
-7.48792599895004	0\\
-7.48601749974936	0\\
-6.98808222508056	0\\
-7.75018507332153	0\\
-7.75009412112849	0\\
-7.49070793884638	0\\
-7.48636310788257	0\\
-7.48692025406585	0\\
-7.48681143384293	0\\
-6.99020203191929	0\\
-6.98789399659894	0\\
-7.4889965622271	0\\
};
\addplot[contour prepared, contour prepared format=matlab] table[row sep=crcr] {%
%
-3	5\\
-2.01917679134759	0\\
-2.04007279879412	-0.0332092433004376\\
-2.05953964331473	0\\
-2.04007279879412	0.0332092433004372\\
-2.01917679134759	0\\
-3	5\\
-1.47515492986085	0\\
-1.48644222597093	-0.0180478489581887\\
-1.49772866366399	0\\
-1.48644222597093	0.0180478489581884\\
-1.47515492986085	0\\
-2	9\\
-3.13589438506225	-0.0900000000000001\\
-3.14733394444051	-0.137924720402257\\
-3.1604454870283	-0.0900000000000001\\
-3.1847216341552	0\\
-3.1604454870283	0.0899999999999989\\
-3.14733394444051	0.137924720402256\\
-3.13589438506225	0.0899999999999989\\
-3.11334067225805	0\\
-3.13589438506225	-0.0900000000000001\\
-2	13\\
-2.82719189705199	-0.0900000000000001\\
-2.83097361711297	-0.10771910315713\\
-2.91006369894486	-0.123047155836766\\
-2.98915378077674	-0.135995344043966\\
-2.9999592741493	-0.0900000000000001\\
-3.02032311270084	0\\
-2.9999592741493	0.0899999999999989\\
-2.98915378077674	0.135995344043965\\
-2.91006369894486	0.123047155836766\\
-2.83097361711297	0.107719103157129\\
-2.82719189705199	0.0899999999999989\\
-2.81474012956908	0\\
-2.82719189705199	-0.0900000000000001\\
-2	9\\
-2.01485142223295	-0.0900000000000001\\
-2.04007279879412	-0.141317493704926\\
-2.06051664652915	-0.0900000000000001\\
-2.09974416250223	0\\
-2.06051664652915	0.0899999999999989\\
-2.04007279879412	0.141317493704926\\
-2.01485142223295	0.0899999999999989\\
-1.97602064801696	0\\
-2.01485142223295	-0.0900000000000001\\
-2	11\\
-1.81053994923738	-0.0900000000000001\\
-1.88189263513035	-0.126620143951792\\
-1.89912806828931	-0.0900000000000001\\
-1.92324654540344	0\\
-1.89912806828931	0.0899999999999989\\
-1.88189263513035	0.126620143951791\\
-1.81053994923738	0.0899999999999989\\
-1.80280255329847	0.0817995284188126\\
-1.77249929387527	0\\
-1.80280255329847	-0.0817995284188137\\
-1.81053994923738	-0.0900000000000001\\
-2	9\\
-1.45640138881697	-0.0900000000000001\\
-1.48644222597093	-0.141091386062956\\
-1.51647658033697	-0.0900000000000001\\
-1.55198421806742	0\\
-1.51647658033697	0.0899999999999989\\
-1.48644222597093	0.141091386062955\\
-1.45640138881697	0.0899999999999989\\
-1.42089524892144	0\\
-1.45640138881697	-0.0900000000000001\\
-2	11\\
-0.78444713412583	-0.0900000000000001\\
-0.853721571315847	-0.136678792410153\\
-0.889712875136201	-0.0900000000000001\\
-0.919413917176874	0\\
-0.889712875136201	0.0899999999999989\\
-0.853721571315847	0.136678792410152\\
-0.78444713412583	0.0899999999999989\\
-0.774631489483964	0.0736961571059024\\
-0.753320226404285	0\\
-0.774631489483964	-0.0736961571059034\\
-0.78444713412583	-0.0900000000000001\\
-2	13\\
0.0730458593375333	-0.0900000000000001\\
0.0162693288348863	-0.139288681768641\\
-0.0628207529969992	-0.106724312025638\\
-0.0761991287933236	-0.0900000000000001\\
-0.10917744760987	0\\
-0.0761991287933236	0.0899999999999989\\
-0.0628207529969992	0.106724312025637\\
0.0162693288348863	0.13928868176864\\
0.0730458593375333	0.0899999999999989\\
0.0953594106667719	0.0266129291186206\\
0.103298711056133	0\\
0.0953594106667719	-0.0266129291186209\\
0.0730458593375333	-0.0900000000000001\\
0	251\\
-3.14411609368809	-1.17\\
-3.14733394444051	-1.18039598974141\\
-3.15144048959826	-1.17\\
-3.19198431593825	-1.08\\
-3.20626174833033	-0.99\\
-3.2264240262724	-0.932229389047429\\
-3.25814715938866	-0.99\\
-3.28859006149852	-1.08\\
-3.30551410810428	-1.1043121838327\\
-3.30878340628184	-1.08\\
-3.31953733858196	-0.99\\
-3.32845565989792	-0.900000000000001\\
-3.33582443087105	-0.810000000000001\\
-3.34215818970132	-0.719999999999999\\
-3.34728408597572	-0.629999999999999\\
-3.35149258817404	-0.54\\
-3.35490973706206	-0.45\\
-3.35765800431854	-0.359999999999999\\
-3.35971924938596	-0.27\\
-3.36125466143585	-0.18\\
-3.36213619893382	-0.0900000000000001\\
-3.36244490353075	0\\
-3.36213619893382	0.0899999999999989\\
-3.36125466143585	0.18\\
-3.35971924938596	0.27\\
-3.35765800431854	0.36\\
-3.35490973706206	0.449999999999999\\
-3.35149258817404	0.54\\
-3.34728408597572	0.630000000000001\\
-3.34215818970132	0.719999999999999\\
-3.33582443087105	0.81\\
-3.32845565989792	0.9\\
-3.31953733858196	0.990000000000001\\
-3.30878340628184	1.08\\
-3.30551410810428	1.1043121838327\\
-3.28859006149852	1.08\\
-3.25814715938866	0.990000000000001\\
-3.2264240262724	0.932229389047429\\
-3.20626174833033	0.990000000000001\\
-3.19198431593825	1.08\\
-3.15144048959826	1.17\\
-3.14733394444051	1.18039598974141\\
-3.14411609368809	1.17\\
-3.11813293620525	1.08\\
-3.10062541942349	0.990000000000001\\
-3.08537344829298	0.9\\
-3.07546046297051	0.81\\
-3.06824386260863	0.726849309383919\\
-3.06100308769863	0.81\\
-3.05082845697594	0.9\\
-3.03573243007745	0.990000000000001\\
-3.01598214953387	1.08\\
-2.9892885090232	1.17\\
-2.98915378077674	1.1704127561989\\
-2.98848026623542	1.17\\
-2.93068143061623	1.08\\
-2.91006369894486	1.06701899590524\\
-2.89012616990893	1.08\\
-2.83097361711297	1.16163150784791\\
-2.80274260914895	1.08\\
-2.76956571152681	0.990000000000001\\
-2.75188353528109	0.914100064959989\\
-2.74783611551632	0.9\\
-2.7085604751215	0.81\\
-2.68721351078727	0.719999999999999\\
-2.6727934534492	0.663461971128397\\
-2.66800548042267	0.719999999999999\\
-2.66134672721785	0.81\\
-2.63769884144362	0.9\\
-2.62973646372694	0.990000000000001\\
-2.60305430265278	1.08\\
-2.59370337161732	1.12001202002505\\
-2.51461328978543	1.10414476403201\\
-2.43552320795355	1.16757916881331\\
-2.35643312612166	1.10120340507587\\
-2.34841638279855	1.08\\
-2.31200622194288	0.990000000000001\\
-2.28926853044952	0.9\\
-2.27734304428978	0.855886795728711\\
-2.22120225215049	0.9\\
-2.19825296245789	0.919236153712455\\
-2.17420051439906	0.990000000000001\\
-2.12901175113978	1.08\\
-2.11916288062601	1.11239725776705\\
-2.04007279879412	1.16134558136919\\
-1.96098271696224	1.11916442882531\\
-1.88189263513035	1.16673969775908\\
-1.80280255329847	1.14201866002302\\
-1.72371247146658	1.12557756233757\\
-1.6446223896347	1.10456433433251\\
-1.56553230780281	1.14258235147816\\
-1.50009215970488	1.17\\
-1.48644222597093	1.17442163578439\\
-1.46640442317794	1.17\\
-1.40735214413904	1.15321449873098\\
-1.32826206230716	1.10302783702982\\
-1.30344484617582	1.08\\
-1.24917198047527	1.01700667797268\\
-1.17008189864339	1.02965812021749\\
-1.10568669251094	1.08\\
-1.0909918168115	1.09298077076797\\
-1.01190173497962	1.13849109825566\\
-0.933796353992691	1.17\\
-0.932811653147733	1.17041401261469\\
-0.853721571315847	1.17760968143549\\
-0.822082596563553	1.17\\
-0.774631489483964	1.16010276806928\\
-0.695541407652078	1.15685439258413\\
-0.616451325820192	1.13917734786163\\
-0.547977451280629	1.08\\
-0.537361243988309	1.07321270433123\\
-0.458271162156423	1.07379152170141\\
-0.379181080324538	1.05310647273969\\
-0.300090998492654	1.04403647188006\\
-0.253706426888418	1.08\\
-0.221000916660768	1.10608274906562\\
-0.145223801529114	1.08\\
-0.141910834828883	1.07856909661322\\
-0.13962122135171	1.08\\
-0.0628207529969992	1.13186140801541\\
0.0162693288348863	1.13788906661853\\
0.0953594106667719	1.1626333868314\\
0.174449492498656	1.14606145396106\\
0.253539574330541	1.1415455572144\\
0.332629656162427	1.11837667522077\\
0.411719737994312	1.09013732482053\\
0.431109140892563	1.08\\
0.490809819826198	1.04370519185709\\
0.565530537919929	0.990000000000001\\
0.559344102684484	0.9\\
0.569899901658081	0.888757127083075\\
0.63539651971477	0.9\\
0.648989983489967	0.905744715197587\\
0.653518790392703	0.9\\
0.714804997714541	0.81\\
0.728080065321853	0.787597818454253\\
0.76346671804337	0.719999999999999\\
0.798205162665318	0.630000000000001\\
0.807170147153736	0.603072629344487\\
0.82772194953684	0.54\\
0.850602804379511	0.449999999999999\\
0.867026297286707	0.36\\
0.879311594222017	0.27\\
0.886260228985622	0.183150699261056\\
0.886515042231716	0.18\\
0.89257191853358	0.0899999999999989\\
0.894360802242554	0\\
0.89257191853358	-0.0900000000000001\\
0.886515042231716	-0.18\\
0.886260228985622	-0.183150699261056\\
0.879311594222017	-0.27\\
0.867026297286707	-0.359999999999999\\
0.850602804379511	-0.45\\
0.82772194953684	-0.54\\
0.807170147153736	-0.603072629344487\\
0.798205162665318	-0.629999999999999\\
0.76346671804337	-0.719999999999999\\
0.728080065321853	-0.787597818454254\\
0.714804997714541	-0.810000000000001\\
0.653518790392703	-0.900000000000001\\
0.648989983489967	-0.905744715197588\\
0.63539651971477	-0.900000000000001\\
0.569899901658081	-0.888757127083076\\
0.559344102684484	-0.900000000000001\\
0.565530537919929	-0.99\\
0.490809819826198	-1.04370519185709\\
0.431109140892563	-1.08\\
0.411719737994312	-1.09013732482053\\
0.332629656162427	-1.11837667522077\\
0.253539574330541	-1.1415455572144\\
0.174449492498656	-1.14606145396106\\
0.0953594106667719	-1.1626333868314\\
0.0162693288348863	-1.13788906661853\\
-0.0628207529969992	-1.13186140801541\\
-0.13962122135171	-1.08\\
-0.141910834828883	-1.07856909661322\\
-0.145223801529114	-1.08\\
-0.221000916660768	-1.10608274906562\\
-0.253706426888418	-1.08\\
-0.300090998492654	-1.04403647188006\\
-0.379181080324538	-1.05310647273969\\
-0.458271162156423	-1.07379152170141\\
-0.537361243988309	-1.07321270433123\\
-0.547977451280629	-1.08\\
-0.616451325820192	-1.13917734786163\\
-0.695541407652078	-1.15685439258413\\
-0.774631489483964	-1.16010276806928\\
-0.822082596563553	-1.17\\
-0.853721571315847	-1.17760968143549\\
-0.932811653147733	-1.17041401261469\\
-0.933796353992691	-1.17\\
-1.01190173497962	-1.13849109825566\\
-1.0909918168115	-1.09298077076797\\
-1.10568669251094	-1.08\\
-1.17008189864339	-1.02965812021749\\
-1.24917198047527	-1.01700667797268\\
-1.30344484617582	-1.08\\
-1.32826206230716	-1.10302783702982\\
-1.40735214413904	-1.15321449873098\\
-1.46640442317794	-1.17\\
-1.48644222597093	-1.17442163578439\\
-1.50009215970488	-1.17\\
-1.56553230780281	-1.14258235147816\\
-1.6446223896347	-1.10456433433251\\
-1.72371247146658	-1.12557756233757\\
-1.80280255329847	-1.14201866002302\\
-1.88189263513035	-1.16673969775908\\
-1.96098271696224	-1.11916442882531\\
-2.04007279879412	-1.16134558136919\\
-2.11916288062601	-1.11239725776705\\
-2.12901175113978	-1.08\\
-2.17420051439906	-0.99\\
-2.19825296245789	-0.919236153712456\\
-2.22120225215049	-0.900000000000001\\
-2.27734304428978	-0.855886795728712\\
-2.28926853044952	-0.900000000000001\\
-2.31200622194288	-0.99\\
-2.34841638279855	-1.08\\
-2.35643312612166	-1.10120340507587\\
-2.43552320795355	-1.16757916881331\\
-2.51461328978543	-1.10414476403201\\
-2.59370337161732	-1.12001202002505\\
-2.60305430265278	-1.08\\
-2.62973646372694	-0.99\\
-2.63769884144362	-0.900000000000001\\
-2.66134672721785	-0.810000000000001\\
-2.66800548042267	-0.719999999999999\\
-2.6727934534492	-0.663461971128396\\
-2.68721351078727	-0.719999999999999\\
-2.7085604751215	-0.810000000000001\\
-2.74783611551632	-0.900000000000001\\
-2.75188353528109	-0.91410006495999\\
-2.76956571152681	-0.99\\
-2.80274260914895	-1.08\\
-2.83097361711297	-1.16163150784791\\
-2.89012616990893	-1.08\\
-2.91006369894486	-1.06701899590524\\
-2.93068143061623	-1.08\\
-2.98848026623542	-1.17\\
-2.98915378077674	-1.1704127561989\\
-2.9892885090232	-1.17\\
-3.01598214953387	-1.08\\
-3.03573243007745	-0.99\\
-3.05082845697594	-0.900000000000001\\
-3.06100308769863	-0.810000000000001\\
-3.06824386260863	-0.726849309383919\\
-3.07546046297051	-0.810000000000001\\
-3.08537344829298	-0.900000000000001\\
-3.10062541942349	-0.99\\
-3.11813293620525	-1.08\\
-3.14411609368809	-1.17\\
1	191\\
-8.36475609675931	-3.96\\
-8.36727934534492	-3.97436631497979\\
-8.36759031501945	-3.96\\
-8.36920395046607	-3.87\\
-8.37132612591022	-3.78\\
-8.37281240155737	-3.69\\
-8.37446899560384	-3.6\\
-8.37605058294963	-3.51\\
-8.37735326069034	-3.42\\
-8.37860549576792	-3.33\\
-8.37964303311079	-3.24\\
-8.38136968819006	-3.15\\
-8.38238990066618	-3.06\\
-8.38372960398397	-2.97\\
-8.38466386792576	-2.88\\
-8.38542545108938	-2.79\\
-8.38697534232359	-2.7\\
-8.3878983030384	-2.61\\
-8.38913427149764	-2.52\\
-8.38935544104336	-2.43\\
-8.39033080658045	-2.34\\
-8.39155384550242	-2.25\\
-8.3919016418731	-2.16\\
-8.39309389856842	-2.07\\
-8.39344269978792	-1.98\\
-8.39490475779067	-1.89\\
-8.39522082100797	-1.8\\
-8.39623167223142	-1.71\\
-8.39668061728465	-1.62\\
-8.39730512226163	-1.53\\
-8.39794861913193	-1.44\\
-8.39843138405771	-1.35\\
-8.3989461351262	-1.26\\
-8.39932328619896	-1.17\\
-8.39970974656963	-1.08\\
-8.40000162106305	-0.99\\
-8.40042378103961	-0.900000000000001\\
-8.40067775782444	-0.810000000000001\\
-8.40095716382818	-0.719999999999999\\
-8.40119425184875	-0.629999999999999\\
-8.40142739031591	-0.54\\
-8.40163931472911	-0.45\\
-8.40166481472042	-0.359999999999999\\
-8.40188030649074	-0.27\\
-8.40188696200765	-0.18\\
-8.40192067581554	-0.0900000000000001\\
-8.40193110895903	0\\
-8.40192067581554	0.0899999999999989\\
-8.40188696200765	0.18\\
-8.40188030649074	0.27\\
-8.40166481472042	0.36\\
-8.40163931472911	0.449999999999999\\
-8.40142739031591	0.54\\
-8.40119425184875	0.630000000000001\\
-8.40095716382818	0.719999999999999\\
-8.40067775782444	0.81\\
-8.40042378103961	0.9\\
-8.40000162106305	0.990000000000001\\
-8.39970974656963	1.08\\
-8.39932328619896	1.17\\
-8.3989461351262	1.26\\
-8.39843138405771	1.35\\
-8.39794861913193	1.44\\
-8.39730512226163	1.53\\
-8.39668061728465	1.62\\
-8.39623167223142	1.71\\
-8.39522082100797	1.8\\
-8.39490475779067	1.89\\
-8.39344269978792	1.98\\
-8.39309389856842	2.07\\
-8.3919016418731	2.16\\
-8.39155384550242	2.25\\
-8.39033080658045	2.34\\
-8.38935544104336	2.43\\
-8.38913427149764	2.52\\
-8.3878983030384	2.61\\
-8.38697534232359	2.7\\
-8.38542545108938	2.79\\
-8.38466386792576	2.88\\
-8.38372960398397	2.97\\
-8.38238990066618	3.06\\
-8.38136968819006	3.15\\
-8.37964303311079	3.24\\
-8.37860549576792	3.33\\
-8.37735326069034	3.42\\
-8.37605058294963	3.51\\
-8.37446899560384	3.6\\
-8.37281240155737	3.69\\
-8.37132612591022	3.78\\
-8.36920395046607	3.87\\
-8.36759031501945	3.96\\
-8.36727934534492	3.97436631497979\\
-8.36475609675931	3.96\\
-8.35239248020731	3.87\\
-8.33895613623852	3.78\\
-8.33103752824661	3.69\\
-8.32251679759601	3.6\\
-8.31760653259053	3.51\\
-8.31143448140554	3.42\\
-8.31167039971413	3.33\\
-8.30459845531044	3.24\\
-8.30284731624337	3.15\\
-8.29939856561767	3.06\\
-8.29711840015303	2.97\\
-8.29514868578017	2.88\\
-8.29426335614822	2.79\\
-8.29337150236143	2.7\\
-8.29271185485211	2.61\\
-8.29123248087557	2.52\\
-8.29040018332926	2.43\\
-8.29007948209943	2.34\\
-8.28961449489765	2.25\\
-8.28937609275474	2.16\\
-8.28977847035904	2.07\\
-8.28950288801261	1.98\\
-8.28857174457552	1.89\\
-8.28861897651586	1.8\\
-8.28828550791464	1.71\\
-8.28827049892505	1.62\\
-8.28893538950753	1.53\\
-8.28818926351304	1.45005793593193\\
-8.28813048814321	1.44\\
-8.28814718544705	1.35\\
-8.28805932428609	1.26\\
-8.28807089875608	1.17\\
-8.28818926351304	1.11486403540824\\
-8.28829547293991	1.08\\
-8.28818926351304	1.05813225449734\\
-8.2879587750648	0.990000000000001\\
-8.28803354385989	0.9\\
-8.28818926351304	0.856076601541088\\
-8.2884099597719	0.81\\
-8.28818926351304	0.727234090923837\\
-8.28817478102685	0.719999999999999\\
-8.28787158416636	0.630000000000001\\
-8.28793246239889	0.54\\
-8.28783454789057	0.449999999999999\\
-8.28804686703236	0.36\\
-8.28807841141841	0.27\\
-8.28804874823815	0.18\\
-8.28805034073168	0.0899999999999989\\
-8.28786737543056	0\\
-8.28805034073168	-0.0900000000000001\\
-8.28804874823815	-0.18\\
-8.28807841141841	-0.27\\
-8.28804686703236	-0.359999999999999\\
-8.28783454789057	-0.45\\
-8.28793246239889	-0.54\\
-8.28787158416636	-0.629999999999999\\
-8.28817478102685	-0.719999999999999\\
-8.28818926351304	-0.727234090923837\\
-8.2884099597719	-0.810000000000001\\
-8.28818926351304	-0.856076601541089\\
-8.28803354385989	-0.900000000000001\\
-8.2879587750648	-0.99\\
-8.28818926351304	-1.05813225449734\\
-8.28829547293991	-1.08\\
-8.28818926351304	-1.11486403540824\\
-8.28807089875608	-1.17\\
-8.28805932428609	-1.26\\
-8.28814718544705	-1.35\\
-8.28813048814321	-1.44\\
-8.28818926351304	-1.45005793593193\\
-8.28893538950753	-1.53\\
-8.28827049892505	-1.62\\
-8.28828550791464	-1.71\\
-8.28861897651586	-1.8\\
-8.28857174457552	-1.89\\
-8.28950288801261	-1.98\\
-8.28977847035904	-2.07\\
-8.28937609275474	-2.16\\
-8.28961449489765	-2.25\\
-8.29007948209943	-2.34\\
-8.29040018332926	-2.43\\
-8.29123248087557	-2.52\\
-8.29271185485211	-2.61\\
-8.29337150236143	-2.7\\
-8.29426335614822	-2.79\\
-8.29514868578017	-2.88\\
-8.29711840015303	-2.97\\
-8.29939856561767	-3.06\\
-8.30284731624337	-3.15\\
-8.30459845531044	-3.24\\
-8.31167039971413	-3.33\\
-8.31143448140554	-3.42\\
-8.31760653259053	-3.51\\
-8.32251679759601	-3.6\\
-8.33103752824661	-3.69\\
-8.33895613623852	-3.78\\
-8.35239248020731	-3.87\\
-8.36475609675931	-3.96\\
1	411\\
0.298043992333027	-3.96\\
0.253539574330541	-3.96990477390342\\
0.174449492498656	-3.97158937453651\\
0.0953594106667719	-3.98838920530233\\
0.0162693288348863	-3.9786986416135\\
-0.0628207529969992	-3.96558799130525\\
-0.141910834828883	-3.97351266204298\\
-0.221000916660768	-3.96743856772533\\
-0.300090998492654	-3.98539148217601\\
-0.379181080324538	-3.98321004197989\\
-0.458271162156423	-3.98330863053281\\
-0.537361243988309	-3.97065081190581\\
-0.616451325820192	-3.98813377884763\\
-0.695541407652078	-3.98142689308296\\
-0.774631489483964	-3.98954761221078\\
-0.853721571315847	-3.98359487528685\\
-0.932811653147733	-3.98932988354717\\
-1.01190173497962	-3.98866103797587\\
-1.0909918168115	-3.98415808789946\\
-1.17008189864339	-3.98920137958283\\
-1.24917198047527	-3.99787377345786\\
-1.32826206230716	-3.99632980595241\\
-1.40735214413904	-4.00071347625305\\
-1.48644222597093	-4.00216312396937\\
-1.56553230780281	-3.9918791641841\\
-1.6446223896347	-3.99501349451247\\
-1.72371247146658	-4.00032765669902\\
-1.80280255329847	-3.99917671619973\\
-1.88189263513035	-3.99403622299482\\
-1.96098271696224	-3.98929350884645\\
-2.04007279879412	-3.9952542263199\\
-2.11916288062601	-4.00157826875563\\
-2.19825296245789	-3.99695717471539\\
-2.27734304428978	-3.98953899533495\\
-2.35643312612166	-3.98369073958715\\
-2.43552320795355	-3.98719932279105\\
-2.51461328978543	-3.9846516310145\\
-2.59370337161732	-3.9912525899205\\
-2.6727934534492	-3.99349146388984\\
-2.75188353528109	-3.99295154681419\\
-2.83097361711297	-3.99306120910786\\
-2.91006369894486	-3.97977023373185\\
-2.98915378077674	-3.99013180699309\\
-3.06824386260863	-3.98874359582033\\
-3.14733394444051	-3.98469961097189\\
-3.2264240262724	-3.980488048163\\
-3.30551410810428	-3.96021385106501\\
-3.30617783807002	-3.96\\
-3.38460418993617	-3.94170068086449\\
-3.46369427176805	-3.87754738111054\\
-3.46824751002662	-3.87\\
-3.54278435359994	-3.81723132213602\\
-3.60066918071399	-3.78\\
-3.62003424519601	-3.69\\
-3.62187443543182	-3.68247257793295\\
-3.64491547638885	-3.6\\
-3.70096451726371	-3.54193725838954\\
-3.72009282884111	-3.51\\
-3.70096451726371	-3.47878808392567\\
-3.6786913296872	-3.42\\
-3.70096451726371	-3.38381670610117\\
-3.7506457358342	-3.33\\
-3.75196675876277	-3.24\\
-3.78005459909559	-3.15479255378355\\
-3.78140589463293	-3.15\\
-3.79019212617906	-3.06\\
-3.80188888761016	-2.97\\
-3.79257511007438	-2.88\\
-3.81459182592362	-2.79\\
-3.8230981529908	-2.7\\
-3.81572429103944	-2.61\\
-3.8144627266748	-2.52\\
-3.8404963448767	-2.43\\
-3.82694242474268	-2.34\\
-3.81339421947465	-2.25\\
-3.8196868521399	-2.16\\
-3.83750808190979	-2.07\\
-3.84720610141309	-1.98\\
-3.83473768949521	-1.89\\
-3.84528708767312	-1.8\\
-3.85148993624867	-1.71\\
-3.84883416931969	-1.62\\
-3.83132872546418	-1.53\\
-3.83985201234588	-1.44\\
-3.84529829853732	-1.35\\
-3.84640974344797	-1.26\\
-3.83483307283148	-1.17\\
-3.85784439177534	-1.08\\
-3.84573019968691	-0.99\\
-3.83731070514457	-0.900000000000001\\
-3.85357837569662	-0.810000000000001\\
-3.84854283735398	-0.719999999999999\\
-3.83783657396719	-0.629999999999999\\
-3.85914468092748	-0.543992137756958\\
-3.860028864826	-0.54\\
-3.85914468092748	-0.502126739919338\\
-3.85774670179969	-0.45\\
-3.84089745485624	-0.359999999999999\\
-3.82567274341279	-0.27\\
-3.85550938134303	-0.18\\
-3.85914468092748	-0.1178350525452\\
-3.86106994031621	-0.0900000000000001\\
-3.85914468092748	-0.0724324015475545\\
-3.85172415998822	0\\
-3.85914468092748	0.0724324015475536\\
-3.86106994031621	0.0899999999999989\\
-3.85914468092748	0.117835052545199\\
-3.85550938134303	0.18\\
-3.82567274341279	0.27\\
-3.84089745485624	0.36\\
-3.85774670179969	0.449999999999999\\
-3.85914468092748	0.502126739919338\\
-3.860028864826	0.54\\
-3.85914468092748	0.543992137756958\\
-3.83783657396719	0.630000000000001\\
-3.84854283735398	0.719999999999999\\
-3.85357837569662	0.81\\
-3.83731070514457	0.9\\
-3.84573019968691	0.990000000000001\\
-3.85784439177534	1.08\\
-3.83483307283148	1.17\\
-3.84640974344797	1.26\\
-3.84529829853732	1.35\\
-3.83985201234588	1.44\\
-3.83132872546418	1.53\\
-3.84883416931969	1.62\\
-3.85148993624867	1.71\\
-3.84528708767312	1.8\\
-3.83473768949521	1.89\\
-3.84720610141309	1.98\\
-3.83750808190979	2.07\\
-3.8196868521399	2.16\\
-3.81339421947465	2.25\\
-3.82694242474268	2.34\\
-3.8404963448767	2.43\\
-3.8144627266748	2.52\\
-3.81572429103944	2.61\\
-3.8230981529908	2.7\\
-3.81459182592362	2.79\\
-3.79257511007438	2.88\\
-3.80188888761016	2.97\\
-3.79019212617906	3.06\\
-3.78140589463293	3.15\\
-3.78005459909559	3.15479255378355\\
-3.75196675876277	3.24\\
-3.7506457358342	3.33\\
-3.70096451726371	3.38381670610117\\
-3.6786913296872	3.42\\
-3.70096451726371	3.47878808392567\\
-3.72009282884111	3.51\\
-3.70096451726371	3.54193725838954\\
-3.64491547638885	3.6\\
-3.62187443543182	3.68247257793295\\
-3.62003424519601	3.69\\
-3.60066918071399	3.78\\
-3.54278435359994	3.81723132213603\\
-3.46824751002662	3.87\\
-3.46369427176805	3.87754738111054\\
-3.38460418993617	3.94170068086449\\
-3.30617783807002	3.96\\
-3.30551410810428	3.96021385106501\\
-3.2264240262724	3.980488048163\\
-3.14733394444051	3.98469961097189\\
-3.06824386260863	3.98874359582033\\
-2.98915378077674	3.99013180699309\\
-2.91006369894486	3.97977023373185\\
-2.83097361711297	3.99306120910786\\
-2.75188353528109	3.99295154681419\\
-2.6727934534492	3.99349146388983\\
-2.59370337161732	3.9912525899205\\
-2.51461328978543	3.9846516310145\\
-2.43552320795355	3.98719932279105\\
-2.35643312612166	3.98369073958715\\
-2.27734304428978	3.98953899533495\\
-2.19825296245789	3.99695717471539\\
-2.11916288062601	4.00157826875563\\
-2.04007279879412	3.99525422631989\\
-1.96098271696224	3.98929350884645\\
-1.88189263513035	3.99403622299482\\
-1.80280255329847	3.99917671619973\\
-1.72371247146658	4.00032765669902\\
-1.6446223896347	3.99501349451247\\
-1.56553230780281	3.9918791641841\\
-1.48644222597093	4.00216312396937\\
-1.40735214413904	4.00071347625305\\
-1.32826206230716	3.99632980595241\\
-1.24917198047527	3.99787377345786\\
-1.17008189864339	3.98920137958283\\
-1.0909918168115	3.98415808789946\\
-1.01190173497962	3.98866103797587\\
-0.932811653147733	3.98932988354717\\
-0.853721571315847	3.98359487528685\\
-0.774631489483964	3.98954761221078\\
-0.695541407652078	3.98142689308296\\
-0.616451325820192	3.98813377884763\\
-0.537361243988309	3.97065081190581\\
-0.458271162156423	3.98330863053281\\
-0.379181080324538	3.98321004197989\\
-0.300090998492654	3.98539148217601\\
-0.221000916660768	3.96743856772533\\
-0.141910834828883	3.97351266204298\\
-0.0628207529969992	3.96558799130525\\
0.0162693288348863	3.9786986416135\\
0.0953594106667719	3.98838920530233\\
0.174449492498656	3.97158937453651\\
0.253539574330541	3.96990477390342\\
0.298043992333027	3.96\\
0.332629656162427	3.95317837321856\\
0.358482223583142	3.96\\
0.411719737994312	3.97332432959273\\
0.479609080551699	3.96\\
0.490809819826198	3.95744076584037\\
0.569899901658081	3.94333786860785\\
0.648989983489967	3.94524829408371\\
0.728080065321853	3.94728791539735\\
0.807170147153736	3.93153516723554\\
0.886260228985622	3.94782709077515\\
0.965350310817507	3.95216141979915\\
1.04444039264939	3.92452691368295\\
1.12353047448128	3.92316494575543\\
1.20262055631316	3.93855656515031\\
1.28171063814505	3.91982235693917\\
1.36080071997693	3.90198631228037\\
1.43989080180882	3.93053053753755\\
1.51417820504815	3.87\\
1.5189808836407	3.86766689659781\\
1.52160876089184	3.87\\
1.59807096547259	3.91668435158777\\
1.67716104730447	3.88627609652793\\
1.70222009978377	3.87\\
1.75625112913636	3.83719454309924\\
1.83534121096824	3.85486829755666\\
1.91443129280013	3.82052403979703\\
1.99352137463201	3.80487678341221\\
2.0726114564639	3.78725644127537\\
2.15170153829578	3.79193227306668\\
2.23079162012767	3.79602995087007\\
2.24983870067159	3.78\\
2.23079162012767	3.75003962076414\\
2.2017566834734	3.69\\
2.23079162012767	3.66179991318584\\
2.30988170195955	3.68271291792536\\
2.38897178379144	3.6538774660014\\
2.41427026114359	3.6\\
2.43484446329886	3.51\\
2.46806186562332	3.46134977838234\\
2.53006442003914	3.51\\
2.54715194745521	3.52154901859424\\
2.62624202928709	3.51819195219326\\
2.63738412947958	3.51\\
2.62624202928709	3.49447526209353\\
2.59201566892776	3.42\\
2.62624202928709	3.3463939090465\\
2.6427876527317	3.33\\
2.70533211111898	3.2885188028927\\
2.78014422277552	3.24\\
2.75822879541857	3.15\\
2.78442219295086	3.11706461347465\\
2.81283439490558	3.06\\
2.86351227478275	2.98060770030529\\
2.87055838879578	2.97\\
2.86351227478275	2.93921061382865\\
2.84754653576932	2.88\\
2.81994102300374	2.79\\
2.86351227478275	2.73329400216209\\
2.89908058459018	2.7\\
2.94089151858075	2.61\\
2.88659599371809	2.52\\
2.92252102461294	2.43\\
2.86351227478275	2.35313207879093\\
2.78442219295086	2.35255306762552\\
2.77677218196656	2.34\\
2.78442219295086	2.32602726642777\\
2.86351227478275	2.32488831798255\\
2.9058996670786	2.25\\
2.94260235661463	2.1795692476872\\
2.96206515750178	2.16\\
2.94260235661463	2.12962249526268\\
2.92255266523344	2.07\\
2.94260235661463	1.99994897369609\\
2.95026766927498	1.98\\
2.94260235661463	1.96606463426658\\
2.9108116414162	1.89\\
2.93531952070178	1.8\\
2.94260235661463	1.76183417526072\\
2.95448523514937	1.71\\
2.94260235661463	1.68363814890168\\
2.91349883172785	1.62\\
2.93901484450696	1.53\\
2.92430873343296	1.44\\
2.94260235661463	1.40364461154673\\
2.96633668461963	1.35\\
2.95189358096413	1.26\\
2.96592524588723	1.17\\
2.94994205387048	1.08\\
2.97254559015981	0.990000000000001\\
2.97347166671456	0.9\\
2.97655887329308	0.81\\
2.96571037320486	0.719999999999999\\
2.9832791769763	0.630000000000001\\
2.94260235661463	0.562348027267863\\
2.93207258956268	0.54\\
2.93961047479424	0.449999999999999\\
2.94260235661463	0.429911173587565\\
2.9560096787989	0.36\\
2.97464846744448	0.27\\
2.98505461331555	0.18\\
2.97182012397556	0.0899999999999989\\
3.0022116924074	0\\
2.97182012397556	-0.0900000000000001\\
2.98505461331555	-0.18\\
2.97464846744448	-0.27\\
2.9560096787989	-0.359999999999999\\
2.94260235661463	-0.429911173587565\\
2.93961047479424	-0.45\\
2.93207258956268	-0.54\\
2.94260235661463	-0.562348027267863\\
2.9832791769763	-0.629999999999999\\
2.96571037320486	-0.719999999999999\\
2.97655887329308	-0.810000000000001\\
2.97347166671456	-0.900000000000001\\
2.97254559015981	-0.99\\
2.94994205387048	-1.08\\
2.96592524588723	-1.17\\
2.95189358096413	-1.26\\
2.96633668461963	-1.35\\
2.94260235661463	-1.40364461154673\\
2.92430873343296	-1.44\\
2.93901484450696	-1.53\\
2.91349883172785	-1.62\\
2.94260235661463	-1.68363814890168\\
2.95448523514937	-1.71\\
2.94260235661463	-1.76183417526072\\
2.93531952070178	-1.8\\
2.9108116414162	-1.89\\
2.94260235661463	-1.96606463426658\\
2.95026766927498	-1.98\\
2.94260235661463	-1.99994897369609\\
2.92255266523344	-2.07\\
2.94260235661463	-2.12962249526268\\
2.96206515750178	-2.16\\
2.94260235661463	-2.1795692476872\\
2.9058996670786	-2.25\\
2.86351227478275	-2.32488831798255\\
2.78442219295086	-2.32602726642777\\
2.77677218196656	-2.34\\
2.78442219295086	-2.35255306762552\\
2.86351227478275	-2.35313207879093\\
2.92252102461294	-2.43\\
2.88659599371809	-2.52\\
2.94089151858075	-2.61\\
2.89908058459018	-2.7\\
2.86351227478275	-2.73329400216209\\
2.81994102300374	-2.79\\
2.84754653576932	-2.88\\
2.86351227478275	-2.93921061382865\\
2.87055838879578	-2.97\\
2.86351227478275	-2.98060770030529\\
2.81283439490558	-3.06\\
2.78442219295086	-3.11706461347465\\
2.75822879541857	-3.15\\
2.78014422277552	-3.24\\
2.70533211111898	-3.2885188028927\\
2.6427876527317	-3.33\\
2.62624202928709	-3.3463939090465\\
2.59201566892776	-3.42\\
2.62624202928709	-3.49447526209352\\
2.63738412947958	-3.51\\
2.62624202928709	-3.51819195219326\\
2.54715194745521	-3.52154901859424\\
2.53006442003914	-3.51\\
2.46806186562332	-3.46134977838234\\
2.43484446329886	-3.51\\
2.41427026114359	-3.6\\
2.38897178379144	-3.6538774660014\\
2.30988170195955	-3.68271291792536\\
2.23079162012767	-3.66179991318584\\
2.2017566834734	-3.69\\
2.23079162012767	-3.75003962076414\\
2.24983870067159	-3.78\\
2.23079162012767	-3.79602995087007\\
2.15170153829578	-3.79193227306668\\
2.0726114564639	-3.78725644127537\\
1.99352137463201	-3.80487678341221\\
1.91443129280013	-3.82052403979703\\
1.83534121096824	-3.85486829755666\\
1.75625112913636	-3.83719454309924\\
1.70222009978377	-3.87\\
1.67716104730447	-3.88627609652793\\
1.59807096547259	-3.91668435158777\\
1.52160876089184	-3.87\\
1.5189808836407	-3.86766689659781\\
1.51417820504815	-3.87\\
1.43989080180882	-3.93053053753755\\
1.36080071997693	-3.90198631228037\\
1.28171063814505	-3.91982235693917\\
1.20262055631316	-3.93855656515031\\
1.12353047448128	-3.92316494575543\\
1.04444039264939	-3.92452691368295\\
0.965350310817507	-3.95216141979915\\
0.886260228985622	-3.94782709077515\\
0.807170147153736	-3.93153516723553\\
0.728080065321853	-3.94728791539735\\
0.648989983489967	-3.94524829408371\\
0.569899901658081	-3.94333786860785\\
0.490809819826198	-3.95744076584037\\
0.479609080551699	-3.96\\
0.411719737994312	-3.97332432959273\\
0.358482223583142	-3.96\\
0.332629656162427	-3.95317837321856\\
0.298043992333027	-3.96\\
1	177\\
-7.49726611454154	-3.87\\
-7.49728844519419	-3.87300628405933\\
-7.49730752542087	-3.87\\
-7.49970659799671	-3.78\\
-7.5027429501784	-3.69\\
-7.50438721063597	-3.6\\
-7.5069192545529	-3.51\\
-7.50918575835978	-3.42\\
-7.51027872458542	-3.33\\
-7.51278863395411	-3.24\\
-7.51371638132891	-3.15\\
-7.51499270712352	-3.06\\
-7.51516885037107	-2.97\\
-7.51795712178273	-2.88\\
-7.51839099160606	-2.79\\
-7.51996277112269	-2.7\\
-7.52094958312277	-2.61\\
-7.5215083677371	-2.52\\
-7.52263343239258	-2.43\\
-7.52307515416468	-2.34\\
-7.52368430390075	-2.25\\
-7.52466868150382	-2.16\\
-7.5250103052252	-2.07\\
-7.52583221845083	-1.98\\
-7.52619885750186	-1.89\\
-7.52665707020831	-1.8\\
-7.52699134455622	-1.71\\
-7.52743936628293	-1.62\\
-7.52780170199007	-1.53\\
-7.52787960808941	-1.44\\
-7.5283514595176	-1.35\\
-7.52850137047252	-1.26\\
-7.5287642906788	-1.17\\
-7.52869480978589	-1.08\\
-7.52915247886826	-0.99\\
-7.52920213719621	-0.900000000000001\\
-7.52936158740143	-0.810000000000001\\
-7.52947727120807	-0.719999999999999\\
-7.52941376734996	-0.629999999999999\\
-7.52962603332097	-0.54\\
-7.52963195399283	-0.45\\
-7.52967330652548	-0.359999999999999\\
-7.52980874968955	-0.27\\
-7.52976649737683	-0.18\\
-7.52971661567168	-0.0900000000000001\\
-7.52981859692511	0\\
-7.52971661567168	0.0899999999999989\\
-7.52976649737683	0.18\\
-7.52980874968955	0.27\\
-7.52967330652548	0.36\\
-7.52963195399283	0.449999999999999\\
-7.52962603332097	0.54\\
-7.52941376734996	0.630000000000001\\
-7.52947727120807	0.719999999999999\\
-7.52936158740143	0.81\\
-7.52920213719621	0.9\\
-7.52915247886826	0.990000000000001\\
-7.52869480978589	1.08\\
-7.5287642906788	1.17\\
-7.52850137047252	1.26\\
-7.5283514595176	1.35\\
-7.52787960808941	1.44\\
-7.52780170199007	1.53\\
-7.52743936628293	1.62\\
-7.52699134455622	1.71\\
-7.52665707020831	1.8\\
-7.52619885750186	1.89\\
-7.52583221845083	1.98\\
-7.5250103052252	2.07\\
-7.52466868150382	2.16\\
-7.52368430390075	2.25\\
-7.52307515416468	2.34\\
-7.52263343239258	2.43\\
-7.5215083677371	2.52\\
-7.52094958312277	2.61\\
-7.51996277112269	2.7\\
-7.51839099160606	2.79\\
-7.51795712178273	2.88\\
-7.51516885037107	2.97\\
-7.51499270712352	3.06\\
-7.51371638132891	3.15\\
-7.51278863395411	3.24\\
-7.51027872458542	3.33\\
-7.50918575835978	3.42\\
-7.5069192545529	3.51\\
-7.50438721063597	3.6\\
-7.5027429501784	3.69\\
-7.49970659799671	3.78\\
-7.49730752542087	3.87\\
-7.49728844519419	3.87300628405933\\
-7.49726611454154	3.87\\
-7.49444064737191	3.78\\
-7.49094153649877	3.69\\
-7.48906088646376	3.6\\
-7.48624298374334	3.51\\
-7.48354597115463	3.42\\
-7.48244393343176	3.33\\
-7.47969770335683	3.24\\
-7.47858882947818	3.15\\
-7.47707141058497	3.06\\
-7.47710662564155	2.97\\
-7.47391502324541	2.88\\
-7.47361136981611	2.79\\
-7.47177485128355	2.7\\
-7.47072934694472	2.61\\
-7.46994103275787	2.52\\
-7.46903005939927	2.43\\
-7.46836139854494	2.34\\
-7.46778473285311	2.25\\
-7.4668211111393	2.16\\
-7.46634183361892	2.07\\
-7.46562226220151	1.98\\
-7.46524521174932	1.89\\
-7.46462671019868	1.8\\
-7.46437396262813	1.71\\
-7.46413272699878	1.62\\
-7.46361628505533	1.53\\
-7.46349950660243	1.44\\
-7.46303486892169	1.35\\
-7.4628514744052	1.26\\
-7.46253531809686	1.17\\
-7.46247413496036	1.08\\
-7.46242106121838	0.990000000000001\\
-7.46210040778424	0.9\\
-7.46202995678954	0.81\\
-7.4617401564142	0.719999999999999\\
-7.46177688419517	0.630000000000001\\
-7.46175977387595	0.54\\
-7.46171229833783	0.449999999999999\\
-7.46152243156028	0.36\\
-7.46149963864522	0.27\\
-7.46144681614344	0.18\\
-7.46153717604629	0.0899999999999989\\
-7.46153492204328	0\\
-7.46153717604629	-0.0900000000000001\\
-7.46144681614344	-0.18\\
-7.46149963864522	-0.27\\
-7.46152243156028	-0.359999999999999\\
-7.46171229833783	-0.45\\
-7.46175977387595	-0.54\\
-7.46177688419517	-0.629999999999999\\
-7.4617401564142	-0.719999999999999\\
-7.46202995678954	-0.810000000000001\\
-7.46210040778424	-0.900000000000001\\
-7.46242106121838	-0.99\\
-7.46247413496036	-1.08\\
-7.46253531809686	-1.17\\
-7.4628514744052	-1.26\\
-7.46303486892169	-1.35\\
-7.46349950660243	-1.44\\
-7.46361628505533	-1.53\\
-7.46413272699878	-1.62\\
-7.46437396262813	-1.71\\
-7.46462671019868	-1.8\\
-7.46524521174932	-1.89\\
-7.46562226220151	-1.98\\
-7.46634183361892	-2.07\\
-7.4668211111393	-2.16\\
-7.46778473285311	-2.25\\
-7.46836139854494	-2.34\\
-7.46903005939927	-2.43\\
-7.46994103275787	-2.52\\
-7.47072934694472	-2.61\\
-7.47177485128355	-2.7\\
-7.47361136981611	-2.79\\
-7.47391502324541	-2.88\\
-7.47710662564155	-2.97\\
-7.47707141058497	-3.06\\
-7.47858882947818	-3.15\\
-7.47969770335683	-3.24\\
-7.48244393343176	-3.33\\
-7.48354597115463	-3.42\\
-7.48624298374334	-3.51\\
-7.48906088646376	-3.6\\
-7.49094153649877	-3.69\\
-7.49444064737191	-3.78\\
-7.49726611454154	-3.87\\
1	177\\
-6.46574096775569	-3.87\\
-6.46911738137968	-3.94959624100803\\
-6.47655493499565	-3.87\\
-6.48886021419141	-3.78\\
-6.49651990763726	-3.69\\
-6.50298420012745	-3.6\\
-6.51010762339226	-3.51\\
-6.5139143724293	-3.42\\
-6.51884441935589	-3.33\\
-6.51915074690409	-3.24\\
-6.5235902364861	-3.15\\
-6.52715037566223	-3.06\\
-6.52802915278332	-2.97\\
-6.53035692106067	-2.88\\
-6.53211962063798	-2.79\\
-6.53431876275758	-2.7\\
-6.53501974891571	-2.61\\
-6.53498962013876	-2.52\\
-6.53693411399418	-2.43\\
-6.53725316126349	-2.34\\
-6.5370247302314	-2.25\\
-6.53755794582538	-2.16\\
-6.5396465861336	-2.07\\
-6.5400500998245	-1.98\\
-6.53881151759734	-1.89\\
-6.54014018676342	-1.8\\
-6.53908119305805	-1.71\\
-6.53980917319647	-1.62\\
-6.54143452799678	-1.53\\
-6.54057831210858	-1.44\\
-6.54046535632301	-1.35\\
-6.5414798923868	-1.26\\
-6.54140506395745	-1.17\\
-6.541905934785	-1.08\\
-6.54130525647481	-0.99\\
-6.54174865027276	-0.900000000000001\\
-6.54174338115283	-0.810000000000001\\
-6.54252123968215	-0.719999999999999\\
-6.5418604526331	-0.629999999999999\\
-6.54081085212934	-0.54\\
-6.54209844512622	-0.45\\
-6.54268082419037	-0.359999999999999\\
-6.54197413182461	-0.27\\
-6.54076651965716	-0.18\\
-6.54240983124214	-0.0900000000000001\\
-6.54133808995154	0\\
-6.54240983124214	0.0899999999999989\\
-6.54076651965716	0.18\\
-6.54197413182461	0.27\\
-6.54268082419037	0.36\\
-6.54209844512622	0.449999999999999\\
-6.54081085212934	0.54\\
-6.5418604526331	0.630000000000001\\
-6.54252123968215	0.719999999999999\\
-6.54174338115283	0.81\\
-6.54174865027276	0.9\\
-6.54130525647481	0.990000000000001\\
-6.541905934785	1.08\\
-6.54140506395745	1.17\\
-6.5414798923868	1.26\\
-6.54046535632301	1.35\\
-6.54057831210858	1.44\\
-6.54143452799678	1.53\\
-6.53980917319647	1.62\\
-6.53908119305805	1.71\\
-6.54014018676342	1.8\\
-6.53881151759734	1.89\\
-6.5400500998245	1.98\\
-6.5396465861336	2.07\\
-6.53755794582538	2.16\\
-6.5370247302314	2.25\\
-6.53725316126349	2.34\\
-6.53693411399418	2.43\\
-6.53498962013876	2.52\\
-6.53501974891571	2.61\\
-6.53431876275758	2.7\\
-6.53211962063798	2.79\\
-6.53035692106067	2.88\\
-6.52802915278332	2.97\\
-6.52715037566223	3.06\\
-6.5235902364861	3.15\\
-6.51915074690409	3.24\\
-6.51884441935589	3.33\\
-6.5139143724293	3.42\\
-6.51010762339226	3.51\\
-6.50298420012745	3.6\\
-6.49651990763726	3.69\\
-6.48886021419141	3.78\\
-6.47655493499565	3.87\\
-6.46911738137968	3.94959624100803\\
-6.46574096775569	3.87\\
-6.45988783256324	3.78\\
-6.45602495406238	3.69\\
-6.45307965511539	3.6\\
-6.44863991868972	3.51\\
-6.44623775960625	3.42\\
-6.44474780431967	3.33\\
-6.4421898094561	3.24\\
-6.43937435204016	3.15\\
-6.4379390787596	3.06\\
-6.43573258079328	2.97\\
-6.43432958142153	2.88\\
-6.43237830386164	2.79\\
-6.43156232675637	2.7\\
-6.42978776420382	2.61\\
-6.42928144688884	2.52\\
-6.42747050893566	2.43\\
-6.42713163448327	2.34\\
-6.42581774305397	2.25\\
-6.42539577554598	2.16\\
-6.42425299030899	2.07\\
-6.42414089175174	1.98\\
-6.423296848044	1.89\\
-6.42280719565946	1.8\\
-6.42241739303624	1.71\\
-6.42309147350013	1.62\\
-6.42159304015453	1.53\\
-6.42149805940267	1.44\\
-6.42098241808726	1.35\\
-6.42113121578302	1.26\\
-6.42018474228142	1.17\\
-6.42017830651303	1.08\\
-6.41993776255674	0.990000000000001\\
-6.42056806395635	0.9\\
-6.41979503404652	0.81\\
-6.41986314514139	0.719999999999999\\
-6.4199106433384	0.630000000000001\\
-6.41943238720401	0.54\\
-6.4193484876467	0.449999999999999\\
-6.4204914219873	0.36\\
-6.42026749848525	0.27\\
-6.41980254590397	0.18\\
-6.41908730242406	0.0899999999999989\\
-6.42018243622672	0\\
-6.41908730242406	-0.0900000000000001\\
-6.41980254590397	-0.18\\
-6.42026749848525	-0.27\\
-6.4204914219873	-0.359999999999999\\
-6.4193484876467	-0.45\\
-6.41943238720401	-0.54\\
-6.4199106433384	-0.629999999999999\\
-6.41986314514139	-0.719999999999999\\
-6.41979503404652	-0.810000000000001\\
-6.42056806395635	-0.900000000000001\\
-6.41993776255674	-0.99\\
-6.42017830651303	-1.08\\
-6.42018474228142	-1.17\\
-6.42113121578302	-1.26\\
-6.42098241808726	-1.35\\
-6.42149805940267	-1.44\\
-6.42159304015453	-1.53\\
-6.42309147350013	-1.62\\
-6.42241739303624	-1.71\\
-6.42280719565946	-1.8\\
-6.423296848044	-1.89\\
-6.42414089175174	-1.98\\
-6.42425299030899	-2.07\\
-6.42539577554598	-2.16\\
-6.42581774305397	-2.25\\
-6.42713163448327	-2.34\\
-6.42747050893566	-2.43\\
-6.42928144688884	-2.52\\
-6.42978776420382	-2.61\\
-6.43156232675637	-2.7\\
-6.43237830386164	-2.79\\
-6.43432958142153	-2.88\\
-6.43573258079328	-2.97\\
-6.4379390787596	-3.06\\
-6.43937435204016	-3.15\\
-6.4421898094561	-3.24\\
-6.44474780431967	-3.33\\
-6.44623775960625	-3.42\\
-6.44863991868972	-3.51\\
-6.45307965511539	-3.6\\
-6.45602495406238	-3.69\\
-6.45988783256324	-3.78\\
-6.46574096775569	-3.87\\
1	5\\
-7.89264083973768	-2.16\\
-7.89273885435361	-2.17534914264048\\
-7.89280526265462	-2.16\\
-7.89273885435361	-2.14813308417089\\
-7.89264083973768	-2.16\\
1	5\\
2.85860251483738	-2.16\\
2.86351227478275	-2.16647674703965\\
2.88279039174738	-2.16\\
2.86351227478275	-2.15524608507789\\
2.85860251483738	-2.16\\
1	17\\
-7.89215875049686	-1.98\\
-7.89273885435361	-2.02272846270422\\
-7.89313240400678	-1.98\\
-7.89328106905907	-1.89\\
-7.8933649493065	-1.8\\
-7.89347771705044	-1.71\\
-7.89356244067385	-1.62\\
-7.89336854143646	-1.53\\
-7.89356754668747	-1.44\\
-7.89273885435361	-1.37628464436382\\
-7.89153064705521	-1.44\\
-7.8918207149614	-1.53\\
-7.8915466298255	-1.62\\
-7.89164744008729	-1.71\\
-7.89182289210834	-1.8\\
-7.89196205164735	-1.89\\
-7.89215875049686	-1.98\\
1	9\\
-7.89133004216515	-1.26\\
-7.89273885435361	-1.32647312738872\\
-7.89370358522777	-1.26\\
-7.89363529229881	-1.17\\
-7.89344820107195	-1.08\\
-7.89273885435361	-1.0081951091329\\
-7.891688460405	-1.08\\
-7.89141294194534	-1.17\\
-7.89133004216515	-1.26\\
1	45\\
-7.89174312860727	-0.900000000000001\\
-7.89273885435361	-0.971217608180991\\
-7.89342320361798	-0.900000000000001\\
-7.8936099276903	-0.810000000000001\\
-7.89409879305502	-0.719999999999999\\
-7.89379307834898	-0.629999999999999\\
-7.89333526532712	-0.54\\
-7.89410230080991	-0.45\\
-7.89371451938262	-0.359999999999999\\
-7.89354901731094	-0.27\\
-7.8927650944714	-0.18\\
-7.89322925172317	-0.0900000000000001\\
-7.89416867586823	0\\
-7.89322925172317	0.0899999999999989\\
-7.8927650944714	0.18\\
-7.89354901731094	0.27\\
-7.89371451938262	0.36\\
-7.89410230080991	0.449999999999999\\
-7.89333526532712	0.54\\
-7.89379307834898	0.630000000000001\\
-7.89409879305502	0.719999999999999\\
-7.8936099276903	0.81\\
-7.89342320361798	0.9\\
-7.89273885435361	0.971217608180992\\
-7.89174312860727	0.9\\
-7.89147449272909	0.81\\
-7.89077569437109	0.719999999999999\\
-7.89120400485389	0.630000000000001\\
-7.89186301767239	0.54\\
-7.8907785785809	0.449999999999999\\
-7.89133954632314	0.36\\
-7.8915668877787	0.27\\
-7.89270014938872	0.18\\
-7.89202068616981	0.0899999999999989\\
-7.89066465756441	0\\
-7.89202068616981	-0.0900000000000001\\
-7.89270014938872	-0.18\\
-7.8915668877787	-0.27\\
-7.89133954632314	-0.359999999999999\\
-7.8907785785809	-0.45\\
-7.89186301767239	-0.54\\
-7.89120400485389	-0.629999999999999\\
-7.89077569437109	-0.719999999999999\\
-7.89147449272909	-0.810000000000001\\
-7.89174312860727	-0.900000000000001\\
1	5\\
2.85991031173239	-0.810000000000001\\
2.86351227478275	-0.820884949230298\\
2.88441561583573	-0.810000000000001\\
2.86351227478275	-0.801662410551757\\
2.85991031173239	-0.810000000000001\\
1	5\\
2.85991031173239	0.81\\
2.86351227478275	0.801662410551756\\
2.88441561583573	0.81\\
2.86351227478275	0.820884949230297\\
2.85991031173239	0.81\\
1	9\\
-7.891688460405	1.08\\
-7.89273885435361	1.0081951091329\\
-7.89344820107195	1.08\\
-7.89363529229881	1.17\\
-7.89370358522777	1.26\\
-7.89273885435361	1.32647312738872\\
-7.89133004216515	1.26\\
-7.89141294194534	1.17\\
-7.891688460405	1.08\\
1	17\\
-7.89153064705521	1.44\\
-7.89273885435361	1.37628464436382\\
-7.89356754668747	1.44\\
-7.89336854143646	1.53\\
-7.89356244067385	1.62\\
-7.89347771705044	1.71\\
-7.8933649493065	1.8\\
-7.89328106905907	1.89\\
-7.89313240400678	1.98\\
-7.89273885435361	2.02272846270422\\
-7.89215875049686	1.98\\
-7.89196205164735	1.89\\
-7.89182289210834	1.8\\
-7.89164744008729	1.71\\
-7.8915466298255	1.62\\
-7.8918207149614	1.53\\
-7.89153064705521	1.44\\
1	5\\
-7.89264083973768	2.16\\
-7.89273885435361	2.14813308417089\\
-7.89280526265462	2.16\\
-7.89273885435361	2.17534914264048\\
-7.89264083973768	2.16\\
1	5\\
2.85860251483738	2.16\\
2.86351227478275	2.15524608507789\\
2.88279039174738	2.16\\
2.86351227478275	2.16647674703965\\
2.85860251483738	2.16\\
2	319\\
-8.28472721352799	-7.02\\
-8.28818926351304	-7.05909331255301\\
-8.36727934534492	-7.09716499668476\\
-8.3739840661569	-7.02\\
-8.38094908774891	-6.93\\
-8.38681898779081	-6.84\\
-8.39198519659105	-6.75\\
-8.39640720990174	-6.66\\
-8.40025336872455	-6.57\\
-8.40376862237429	-6.48\\
-8.40667841347481	-6.39\\
-8.40943865653098	-6.3\\
-8.41168548543777	-6.21\\
-8.41380327299917	-6.12\\
-8.415495209884	-6.03\\
-8.41734397989397	-5.94\\
-8.41883785622502	-5.85\\
-8.42010671022932	-5.76\\
-8.42146589074534	-5.67\\
-8.42239496279964	-5.58\\
-8.42347864347687	-5.49\\
-8.4244551366826	-5.4\\
-8.42534949411129	-5.31\\
-8.42627041662315	-5.22\\
-8.42699867636016	-5.13\\
-8.42764498002874	-5.04\\
-8.42832117843417	-4.95\\
-8.4286843987696	-4.86\\
-8.42939008812822	-4.77\\
-8.42969976970857	-4.68\\
-8.43054050530897	-4.59\\
-8.43096434329868	-4.5\\
-8.4312919915708	-4.41\\
-8.43172172812646	-4.32\\
-8.43210560594673	-4.23\\
-8.43259358047883	-4.14\\
-8.43261426697646	-4.05\\
-8.43331955087427	-3.96\\
-8.43348171658094	-3.87\\
-8.43376387016587	-3.78\\
-8.43398817936229	-3.69\\
-8.43433992775675	-3.6\\
-8.43454343755856	-3.51\\
-8.43491018907192	-3.42\\
-8.43496178751257	-3.33\\
-8.43523090393639	-3.24\\
-8.43550728674206	-3.15\\
-8.43559916904689	-3.06\\
-8.4357717293342	-2.97\\
-8.43600914443414	-2.88\\
-8.43615856231862	-2.79\\
-8.4365583562226	-2.7\\
-8.43650780315063	-2.61\\
-8.43681948937902	-2.52\\
-8.43676548075053	-2.43\\
-8.43706746129842	-2.34\\
-8.4372159937965	-2.25\\
-8.43726395788696	-2.16\\
-8.43755594736519	-2.07\\
-8.43751565164353	-1.98\\
-8.43779873425051	-1.89\\
-8.43789045321787	-1.8\\
-8.43800397254069	-1.71\\
-8.43800660472593	-1.62\\
-8.43817651313521	-1.53\\
-8.43836168760786	-1.44\\
-8.43840070355791	-1.35\\
-8.43857640404032	-1.26\\
-8.43852405663549	-1.17\\
-8.4385931704697	-1.08\\
-8.43866237681287	-0.99\\
-8.43869122995276	-0.900000000000001\\
-8.43864400265745	-0.810000000000001\\
-8.43873945410668	-0.719999999999999\\
-8.4387681764015	-0.629999999999999\\
-8.43887378033437	-0.54\\
-8.43898464524379	-0.45\\
-8.43886388160032	-0.359999999999999\\
-8.43900472949328	-0.27\\
-8.43888591467013	-0.18\\
-8.43886290127951	-0.0900000000000001\\
-8.43892507951572	0\\
-8.43886290127951	0.0899999999999989\\
-8.43888591467013	0.18\\
-8.43900472949328	0.27\\
-8.43886388160032	0.36\\
-8.43898464524379	0.449999999999999\\
-8.43887378033437	0.54\\
-8.4387681764015	0.630000000000001\\
-8.43873945410668	0.719999999999999\\
-8.43864400265745	0.81\\
-8.43869122995276	0.9\\
-8.43866237681287	0.990000000000001\\
-8.4385931704697	1.08\\
-8.43852405663549	1.17\\
-8.43857640404032	1.26\\
-8.43840070355791	1.35\\
-8.43836168760786	1.44\\
-8.43817651313521	1.53\\
-8.43800660472593	1.62\\
-8.43800397254069	1.71\\
-8.43789045321787	1.8\\
-8.43779873425051	1.89\\
-8.43751565164353	1.98\\
-8.43755594736519	2.07\\
-8.43726395788696	2.16\\
-8.4372159937965	2.25\\
-8.43706746129842	2.34\\
-8.43676548075053	2.43\\
-8.43681948937902	2.52\\
-8.43650780315063	2.61\\
-8.4365583562226	2.7\\
-8.43615856231862	2.79\\
-8.43600914443414	2.88\\
-8.4357717293342	2.97\\
-8.43559916904689	3.06\\
-8.43550728674206	3.15\\
-8.43523090393639	3.24\\
-8.43496178751257	3.33\\
-8.43491018907192	3.42\\
-8.43454343755856	3.51\\
-8.43433992775675	3.6\\
-8.43398817936229	3.69\\
-8.43376387016587	3.78\\
-8.43348171658094	3.87\\
-8.43331955087427	3.96\\
-8.43261426697646	4.05\\
-8.43259358047883	4.14\\
-8.43210560594673	4.23\\
-8.43172172812646	4.32\\
-8.4312919915708	4.41\\
-8.43096434329868	4.5\\
-8.43054050530897	4.59\\
-8.42969976970857	4.68\\
-8.42939008812822	4.77\\
-8.4286843987696	4.86\\
-8.42832117843417	4.95\\
-8.42764498002874	5.04\\
-8.42699867636016	5.13\\
-8.42627041662315	5.22\\
-8.42534949411129	5.31\\
-8.4244551366826	5.4\\
-8.42347864347687	5.49\\
-8.42239496279964	5.58\\
-8.42146589074534	5.67\\
-8.42010671022932	5.76\\
-8.41883785622502	5.85\\
-8.41734397989397	5.94\\
-8.415495209884	6.03\\
-8.41380327299917	6.12\\
-8.41168548543777	6.21\\
-8.40943865653098	6.3\\
-8.40667841347481	6.39\\
-8.40376862237429	6.48\\
-8.40025336872455	6.57\\
-8.39640720990174	6.66\\
-8.39198519659105	6.75\\
-8.38681898779081	6.84\\
-8.38094908774891	6.93\\
-8.3739840661569	7.02\\
-8.36727934534492	7.09716499668476\\
-8.28818926351304	7.05909331255301\\
-8.28472721352799	7.02\\
-8.27800563033416	6.93\\
-8.27102982202965	6.84\\
-8.26499398366129	6.75\\
-8.26134836031249	6.66\\
-8.25691609381744	6.57\\
-8.25285368511667	6.48\\
-8.24983885295521	6.39\\
-8.24655463339375	6.3\\
-8.24436623357079	6.21\\
-8.24136871419682	6.12\\
-8.24071033464807	6.03\\
-8.23846981556356	5.94\\
-8.23608923600346	5.85\\
-8.23445382015965	5.76\\
-8.23371812120598	5.67\\
-8.23245154505567	5.58\\
-8.23158102022826	5.49\\
-8.23136385783122	5.4\\
-8.23018281937614	5.31\\
-8.22910313274415	5.22\\
-8.22834424393072	5.13\\
-8.22786509708426	5.04\\
-8.22684260252509	4.95\\
-8.22661430204902	4.86\\
-8.22598319315139	4.77\\
-8.22469614332621	4.68\\
-8.22506402588731	4.59\\
-8.22477180143969	4.5\\
-8.22467620550811	4.41\\
-8.22411085244845	4.32\\
-8.22381850477814	4.23\\
-8.22321834739263	4.14\\
-8.22354948199712	4.05\\
-8.2236989967417	3.96\\
-8.22319493515192	3.87\\
-8.22243872935485	3.78\\
-8.22262233682294	3.69\\
-8.22208946736276	3.6\\
-8.22252347898992	3.51\\
-8.22298189625116	3.42\\
-8.22255177368424	3.33\\
-8.22178520368959	3.24\\
-8.22145760201982	3.15\\
-8.22217160279308	3.06\\
-8.2211195245588	2.97\\
-8.2219613184479	2.88\\
-8.22185489236872	2.79\\
-8.22182743252803	2.7\\
-8.22166126258908	2.61\\
-8.22138873392654	2.52\\
-8.2215183002312	2.43\\
-8.2215266535262	2.34\\
-8.22121369599537	2.25\\
-8.22132108950909	2.16\\
-8.22139648855772	2.07\\
-8.22134051591094	1.98\\
-8.22094340825203	1.89\\
-8.22157416882255	1.8\\
-8.22136270139823	1.71\\
-8.22106075704666	1.62\\
-8.22156111215897	1.53\\
-8.2213606455349	1.44\\
-8.22056115093264	1.35\\
-8.22146986228787	1.26\\
-8.22133605881779	1.17\\
-8.22187202302815	1.08\\
-8.22199766583731	0.990000000000001\\
-8.22167249551807	0.9\\
-8.22094365734584	0.81\\
-8.22086224460145	0.719999999999999\\
-8.22138787450697	0.630000000000001\\
-8.22117582836749	0.54\\
-8.22138286479658	0.449999999999999\\
-8.22062148392457	0.36\\
-8.22149417836209	0.27\\
-8.22144079937169	0.18\\
-8.22126662234723	0.0899999999999989\\
-8.22133839931971	0\\
-8.22126662234723	-0.0900000000000001\\
-8.22144079937169	-0.18\\
-8.22149417836209	-0.27\\
-8.22062148392457	-0.359999999999999\\
-8.22138286479658	-0.45\\
-8.22117582836749	-0.54\\
-8.22138787450697	-0.629999999999999\\
-8.22086224460145	-0.719999999999999\\
-8.22094365734584	-0.810000000000001\\
-8.22167249551807	-0.900000000000001\\
-8.22199766583731	-0.99\\
-8.22187202302815	-1.08\\
-8.22133605881779	-1.17\\
-8.22146986228787	-1.26\\
-8.22056115093264	-1.35\\
-8.2213606455349	-1.44\\
-8.22156111215897	-1.53\\
-8.22106075704666	-1.62\\
-8.22136270139823	-1.71\\
-8.22157416882255	-1.8\\
-8.22094340825203	-1.89\\
-8.22134051591094	-1.98\\
-8.22139648855772	-2.07\\
-8.22132108950909	-2.16\\
-8.22121369599537	-2.25\\
-8.2215266535262	-2.34\\
-8.2215183002312	-2.43\\
-8.22138873392654	-2.52\\
-8.22166126258908	-2.61\\
-8.22182743252803	-2.7\\
-8.22185489236872	-2.79\\
-8.2219613184479	-2.88\\
-8.2211195245588	-2.97\\
-8.22217160279308	-3.06\\
-8.22145760201982	-3.15\\
-8.22178520368959	-3.24\\
-8.22255177368424	-3.33\\
-8.22298189625116	-3.42\\
-8.22252347898992	-3.51\\
-8.22208946736276	-3.6\\
-8.22262233682294	-3.69\\
-8.22243872935485	-3.78\\
-8.22319493515192	-3.87\\
-8.2236989967417	-3.96\\
-8.22354948199712	-4.05\\
-8.22321834739263	-4.14\\
-8.22381850477814	-4.23\\
-8.22411085244845	-4.32\\
-8.22467620550811	-4.41\\
-8.22477180143969	-4.5\\
-8.22506402588731	-4.59\\
-8.22469614332621	-4.68\\
-8.22598319315139	-4.77\\
-8.22661430204902	-4.86\\
-8.22684260252509	-4.95\\
-8.22786509708426	-5.04\\
-8.22834424393072	-5.13\\
-8.22910313274415	-5.22\\
-8.23018281937614	-5.31\\
-8.23136385783122	-5.4\\
-8.23158102022826	-5.49\\
-8.23245154505567	-5.58\\
-8.23371812120598	-5.67\\
-8.23445382015965	-5.76\\
-8.23608923600346	-5.85\\
-8.23846981556356	-5.94\\
-8.24071033464807	-6.03\\
-8.24136871419682	-6.12\\
-8.24436623357079	-6.21\\
-8.24655463339375	-6.3\\
-8.24983885295521	-6.39\\
-8.25285368511667	-6.48\\
-8.25691609381744	-6.57\\
-8.26134836031249	-6.66\\
-8.26499398366129	-6.75\\
-8.27102982202965	-6.84\\
-8.27800563033416	-6.93\\
-8.28472721352799	-7.02\\
2	399\\
-7.85918183433593	-7.02\\
-7.89273885435361	-7.06285045687274\\
-7.90835953117391	-7.02\\
-7.93059144027816	-6.93\\
-7.95511585071876	-6.84\\
-7.96752191980072	-6.75\\
-7.9718289361855	-6.72426873304781\\
-7.97572251109529	-6.66\\
-7.97925887262948	-6.57\\
-7.98567614211265	-6.48\\
-7.98808401592367	-6.39\\
-7.99091016203429	-6.3\\
-7.99371045065937	-6.21\\
-7.99499719161741	-6.12\\
-7.99724552125536	-6.03\\
-7.99974447744278	-5.94\\
-8.00092813356837	-5.85\\
-8.00241004149523	-5.76\\
-8.00377726094079	-5.67\\
-8.00318306564655	-5.58\\
-8.00464882147578	-5.49\\
-8.00555706367507	-5.4\\
-8.00668966037317	-5.31\\
-8.00694928667718	-5.22\\
-8.00672558750481	-5.13\\
-8.00773182309601	-5.04\\
-8.00840564607987	-4.95\\
-8.00808425519762	-4.86\\
-8.00872278894135	-4.77\\
-8.00886875634467	-4.68\\
-8.00887692394545	-4.59\\
-8.00842869567762	-4.5\\
-8.00928722154944	-4.41\\
-8.0093762238144	-4.32\\
-8.00977855682882	-4.23\\
-8.00973973399987	-4.14\\
-8.01028607855836	-4.05\\
-8.00940952503341	-3.96\\
-8.0093951767906	-3.87\\
-8.01070585568981	-3.78\\
-8.01076879933262	-3.69\\
-8.01020767278618	-3.6\\
-8.01028407161566	-3.51\\
-8.01012254836456	-3.42\\
-8.01031966754739	-3.33\\
-8.00980482696614	-3.24\\
-8.01033408476618	-3.15\\
-8.01090941330675	-3.06\\
-8.01051435914792	-2.97\\
-8.00966053146962	-2.88\\
-8.01071591955054	-2.79\\
-8.01086815550833	-2.7\\
-8.01030552191592	-2.61\\
-8.0106712030964	-2.52\\
-8.01026643227108	-2.43\\
-8.01059457027535	-2.34\\
-8.01066145104685	-2.25\\
-8.01062032930551	-2.16\\
-8.01071031473896	-2.07\\
-8.00907982664382	-1.98\\
-8.01040762535214	-1.89\\
-8.01107839109051	-1.8\\
-8.01068009154804	-1.71\\
-8.01109528492614	-1.62\\
-8.01075624184351	-1.53\\
-8.01039701003315	-1.44\\
-8.00982895744134	-1.35\\
-8.00972485212651	-1.26\\
-8.01068725242115	-1.17\\
-8.00991303791641	-1.08\\
-8.01072382802414	-0.99\\
-8.01088374430843	-0.900000000000001\\
-8.01109740502429	-0.810000000000001\\
-8.01073139282319	-0.719999999999999\\
-8.00983925660476	-0.629999999999999\\
-8.01007710958119	-0.54\\
-8.01064590076974	-0.45\\
-8.01005032583905	-0.359999999999999\\
-8.0099565713228	-0.27\\
-8.01016022320927	-0.18\\
-8.01027278878876	-0.0900000000000001\\
-8.01067667156035	0\\
-8.01027278878876	0.0899999999999989\\
-8.01016022320927	0.18\\
-8.0099565713228	0.27\\
-8.01005032583905	0.36\\
-8.01064590076974	0.449999999999999\\
-8.01007710958119	0.54\\
-8.00983925660476	0.630000000000001\\
-8.01073139282319	0.719999999999999\\
-8.01109740502429	0.81\\
-8.01088374430843	0.9\\
-8.01072382802414	0.990000000000001\\
-8.00991303791641	1.08\\
-8.01068725242115	1.17\\
-8.00972485212651	1.26\\
-8.00982895744134	1.35\\
-8.01039701003315	1.44\\
-8.01075624184351	1.53\\
-8.01109528492614	1.62\\
-8.01068009154804	1.71\\
-8.01107839109051	1.8\\
-8.01040762535214	1.89\\
-8.00907982664382	1.98\\
-8.01071031473896	2.07\\
-8.01062032930551	2.16\\
-8.01066145104685	2.25\\
-8.01059457027535	2.34\\
-8.01026643227108	2.43\\
-8.0106712030964	2.52\\
-8.01030552191592	2.61\\
-8.01086815550833	2.7\\
-8.01071591955054	2.79\\
-8.00966053146962	2.88\\
-8.01051435914792	2.97\\
-8.01090941330675	3.06\\
-8.01033408476618	3.15\\
-8.00980482696614	3.24\\
-8.01031966754739	3.33\\
-8.01012254836456	3.42\\
-8.01028407161566	3.51\\
-8.01020767278618	3.6\\
-8.01076879933262	3.69\\
-8.01070585568981	3.78\\
-8.0093951767906	3.87\\
-8.00940952503341	3.96\\
-8.01028607855836	4.05\\
-8.00973973399987	4.14\\
-8.00977855682882	4.23\\
-8.0093762238144	4.32\\
-8.00928722154944	4.41\\
-8.00842869567762	4.5\\
-8.00887692394545	4.59\\
-8.00886875634467	4.68\\
-8.00872278894135	4.77\\
-8.00808425519762	4.86\\
-8.00840564607987	4.95\\
-8.00773182309601	5.04\\
-8.00672558750481	5.13\\
-8.00694928667718	5.22\\
-8.00668966037317	5.31\\
-8.00555706367507	5.4\\
-8.00464882147578	5.49\\
-8.00318306564655	5.58\\
-8.00377726094079	5.67\\
-8.00241004149523	5.76\\
-8.00092813356837	5.85\\
-7.99974447744278	5.94\\
-7.99724552125536	6.03\\
-7.99499719161741	6.12\\
-7.99371045065937	6.21\\
-7.99091016203429	6.3\\
-7.98808401592367	6.39\\
-7.98567614211265	6.48\\
-7.97925887262948	6.57\\
-7.97572251109529	6.66\\
-7.9718289361855	6.72426873304781\\
-7.96752191980072	6.75\\
-7.95511585071876	6.84\\
-7.93059144027816	6.93\\
-7.90835953117391	7.02\\
-7.89273885435361	7.06285045687274\\
-7.85918183433593	7.02\\
-7.81364877252173	6.96708702828217\\
-7.74876658210108	7.02\\
-7.73455869068984	7.03603413822565\\
-7.73214630204473	7.02\\
-7.71640693025908	6.93\\
-7.70597799089157	6.84\\
-7.69525142250571	6.75\\
-7.68773228070135	6.66\\
-7.68029362999896	6.57\\
-7.67523896054613	6.48\\
-7.67004882369435	6.39\\
-7.66887297138313	6.3\\
-7.66376526441964	6.21\\
-7.6616395675011	6.12\\
-7.6602748267009	6.03\\
-7.65946629327709	5.94\\
-7.65683829941116	5.85\\
-7.6568877531349	5.76\\
-7.65626793497638	5.67\\
-7.65546860885796	5.63604262170181\\
-7.65223193802773	5.67\\
-7.64990291265439	5.76\\
-7.6500762490909	5.85\\
-7.64076044916533	5.94\\
-7.63837820410396	6.03\\
-7.63324858820676	6.12\\
-7.6254157723974	6.21\\
-7.61489722627427	6.3\\
-7.61208920775307	6.39\\
-7.59509854223791	6.48\\
-7.5836557943082	6.57\\
-7.57637852702607	6.61954943347901\\
-7.57332154752728	6.66\\
-7.56349937370165	6.75\\
-7.55073884971885	6.84\\
-7.53575090170674	6.93\\
-7.5172604907845	7.02\\
-7.49728844519419	7.09583101209715\\
-7.46516511383146	7.02\\
-7.4357772366568	6.93\\
-7.4181983633623	6.87200727683181\\
-7.41360522986562	6.84\\
-7.40424389929734	6.75\\
-7.39589552972877	6.66\\
-7.38877173019567	6.57\\
-7.38225807671506	6.48\\
-7.37757923370332	6.39\\
-7.37278117346035	6.3\\
-7.37051087567103	6.21\\
-7.36736501094513	6.12\\
-7.36445548420599	6.03\\
-7.36140682938779	5.94\\
-7.36063902735306	5.85\\
-7.35856622705996	5.76\\
-7.35734397083433	5.67\\
-7.35620116441151	5.58\\
-7.35524753837725	5.49\\
-7.35339842527976	5.4\\
-7.3530867442091	5.31\\
-7.35346880591459	5.22\\
-7.35277063565523	5.13\\
-7.35148740445859	5.04\\
-7.35121820384984	4.95\\
-7.35092335161102	4.86\\
-7.35066051809411	4.77\\
-7.35008899088564	4.68\\
-7.34990680394613	4.59\\
-7.34997019078778	4.5\\
-7.34982575105341	4.41\\
-7.35005507844556	4.32\\
-7.34893488218752	4.23\\
-7.3493519543006	4.14\\
-7.34982848790641	4.05\\
-7.34883306321192	3.96\\
-7.3497087336037	3.87\\
-7.34957188193043	3.78\\
-7.34912230707951	3.69\\
-7.34826926892921	3.6\\
-7.34922837246028	3.51\\
-7.34868458508058	3.42\\
-7.34870994200573	3.33\\
-7.34906651838499	3.24\\
-7.34935789734061	3.15\\
-7.34858765159908	3.06\\
-7.34958377207011	2.97\\
-7.34841831474121	2.88\\
-7.34853403981157	2.79\\
-7.34873356122085	2.7\\
-7.3490386671018	2.61\\
-7.34881618892451	2.52\\
-7.34876793181573	2.43\\
-7.34802628632643	2.34\\
-7.34920931983222	2.25\\
-7.34820273702571	2.16\\
-7.34916963283846	2.07\\
-7.34929108192708	1.98\\
-7.34873806154993	1.89\\
-7.34848766570674	1.8\\
-7.34798486460908	1.71\\
-7.34882546838577	1.62\\
-7.34913091458249	1.53\\
-7.34879707725242	1.44\\
-7.34864895221268	1.35\\
-7.3486551414167	1.26\\
-7.34845357862383	1.17\\
-7.34774857341723	1.08\\
-7.34837398518043	0.990000000000001\\
-7.34862606515531	0.9\\
-7.3489407033091	0.81\\
-7.34857501742721	0.719999999999999\\
-7.34923037431775	0.630000000000001\\
-7.34868419059745	0.54\\
-7.349066809759	0.449999999999999\\
-7.34819274847761	0.36\\
-7.34878669132961	0.27\\
-7.34820108124059	0.18\\
-7.34868377965771	0.0899999999999989\\
-7.34915816698733	0\\
-7.34868377965771	-0.0900000000000001\\
-7.34820108124059	-0.18\\
-7.34878669132961	-0.27\\
-7.34819274847761	-0.359999999999999\\
-7.349066809759	-0.45\\
-7.34868419059745	-0.54\\
-7.34923037431775	-0.629999999999999\\
-7.34857501742721	-0.719999999999999\\
-7.3489407033091	-0.810000000000001\\
-7.34862606515531	-0.900000000000001\\
-7.34837398518043	-0.99\\
-7.34774857341723	-1.08\\
-7.34845357862383	-1.17\\
-7.3486551414167	-1.26\\
-7.34864895221268	-1.35\\
-7.34879707725242	-1.44\\
-7.34913091458249	-1.53\\
-7.34882546838577	-1.62\\
-7.34798486460908	-1.71\\
-7.34848766570674	-1.8\\
-7.34873806154993	-1.89\\
-7.34929108192708	-1.98\\
-7.34916963283846	-2.07\\
-7.34820273702571	-2.16\\
-7.34920931983222	-2.25\\
-7.34802628632643	-2.34\\
-7.34876793181573	-2.43\\
-7.34881618892451	-2.52\\
-7.3490386671018	-2.61\\
-7.34873356122085	-2.7\\
-7.34853403981157	-2.79\\
-7.34841831474121	-2.88\\
-7.34958377207011	-2.97\\
-7.34858765159908	-3.06\\
-7.34935789734061	-3.15\\
-7.34906651838499	-3.24\\
-7.34870994200573	-3.33\\
-7.34868458508058	-3.42\\
-7.34922837246028	-3.51\\
-7.34826926892921	-3.6\\
-7.34912230707951	-3.69\\
-7.34957188193043	-3.78\\
-7.3497087336037	-3.87\\
-7.34883306321192	-3.96\\
-7.34982848790641	-4.05\\
-7.3493519543006	-4.14\\
-7.34893488218752	-4.23\\
-7.35005507844556	-4.32\\
-7.34982575105341	-4.41\\
-7.34997019078778	-4.5\\
-7.34990680394613	-4.59\\
-7.35008899088564	-4.68\\
-7.35066051809411	-4.77\\
-7.35092335161102	-4.86\\
-7.35121820384984	-4.95\\
-7.35148740445859	-5.04\\
-7.35277063565523	-5.13\\
-7.35346880591459	-5.22\\
-7.3530867442091	-5.31\\
-7.35339842527976	-5.4\\
-7.35524753837725	-5.49\\
-7.35620116441151	-5.58\\
-7.35734397083433	-5.67\\
-7.35856622705996	-5.76\\
-7.36063902735306	-5.85\\
-7.36140682938779	-5.94\\
-7.36445548420599	-6.03\\
-7.36736501094513	-6.12\\
-7.37051087567103	-6.21\\
-7.37278117346035	-6.3\\
-7.37757923370332	-6.39\\
-7.38225807671506	-6.48\\
-7.38877173019567	-6.57\\
-7.39589552972877	-6.66\\
-7.40424389929734	-6.75\\
-7.41360522986562	-6.84\\
-7.4181983633623	-6.87200727683181\\
-7.4357772366568	-6.93\\
-7.46516511383146	-7.02\\
-7.49728844519419	-7.09583101209715\\
-7.5172604907845	-7.02\\
-7.53575090170674	-6.93\\
-7.55073884971885	-6.84\\
-7.56349937370165	-6.75\\
-7.57332154752728	-6.66\\
-7.57637852702607	-6.61954943347902\\
-7.5836557943082	-6.57\\
-7.59509854223791	-6.48\\
-7.61208920775307	-6.39\\
-7.61489722627427	-6.3\\
-7.6254157723974	-6.21\\
-7.63324858820676	-6.12\\
-7.63837820410396	-6.03\\
-7.64076044916533	-5.94\\
-7.6500762490909	-5.85\\
-7.64990291265439	-5.76\\
-7.65223193802773	-5.67\\
-7.65546860885796	-5.63604262170181\\
-7.65626793497638	-5.67\\
-7.6568877531349	-5.76\\
-7.65683829941116	-5.85\\
-7.65946629327709	-5.94\\
-7.6602748267009	-6.03\\
-7.6616395675011	-6.12\\
-7.66376526441964	-6.21\\
-7.66887297138313	-6.3\\
-7.67004882369435	-6.39\\
-7.67523896054613	-6.48\\
-7.68029362999896	-6.57\\
-7.68773228070135	-6.66\\
-7.69525142250571	-6.75\\
-7.70597799089157	-6.84\\
-7.71640693025908	-6.93\\
-7.73214630204473	-7.02\\
-7.73455869068984	-7.03603413822565\\
-7.74876658210108	-7.02\\
-7.81364877252173	-6.96708702828217\\
-7.85918183433593	-7.02\\
2	393\\
-6.91973239069251	-7.02\\
-6.94365787237099	-7.07189254164727\\
-7.02274795420288	-7.07696717378646\\
-7.04548811979912	-7.02\\
-7.07473541311415	-6.93\\
-7.09650823536508	-6.84\\
-7.10183803603476	-6.81508579442844\\
-7.11072049374624	-6.75\\
-7.12225288874396	-6.66\\
-7.13007419344207	-6.57\\
-7.13737217055795	-6.48\\
-7.14362152242017	-6.39\\
-7.14873352251056	-6.3\\
-7.15302352208313	-6.21\\
-7.15664601544143	-6.12\\
-7.15932018162253	-6.03\\
-7.16151727552735	-5.94\\
-7.16377519265887	-5.85\\
-7.16608571964119	-5.76\\
-7.16686962488339	-5.67\\
-7.16876667986837	-5.58\\
-7.16994925490491	-5.49\\
-7.17100854710798	-5.4\\
-7.17124399945502	-5.31\\
-7.17270164518548	-5.22\\
-7.17286631810918	-5.13\\
-7.17321678746562	-5.04\\
-7.17448794777458	-4.95\\
-7.17475267018083	-4.86\\
-7.17535540811484	-4.77\\
-7.1756519809333	-4.68\\
-7.17589571735148	-4.59\\
-7.1749744182885	-4.5\\
-7.17607585498238	-4.41\\
-7.17512309278094	-4.32\\
-7.17552444659377	-4.23\\
-7.17601288779868	-4.14\\
-7.17604664351468	-4.05\\
-7.17694687554758	-3.96\\
-7.17659673284194	-3.87\\
-7.17701692996982	-3.78\\
-7.17646285301069	-3.69\\
-7.1769951289843	-3.6\\
-7.17624644223242	-3.51\\
-7.17734213597931	-3.42\\
-7.17677502012623	-3.33\\
-7.17675277357566	-3.24\\
-7.17763782084545	-3.15\\
-7.17779163237446	-3.06\\
-7.1769881200352	-2.97\\
-7.17795061466745	-2.88\\
-7.17628108952411	-2.79\\
-7.17735106510209	-2.7\\
-7.17701523313792	-2.61\\
-7.17688674133837	-2.52\\
-7.1771413809501	-2.43\\
-7.17615326499459	-2.34\\
-7.177265561113	-2.25\\
-7.17800948961331	-2.16\\
-7.17669449640087	-2.07\\
-7.17587793602637	-1.98\\
-7.17798100079893	-1.89\\
-7.17774449104353	-1.8\\
-7.17803535002242	-1.71\\
-7.17765599160654	-1.62\\
-7.17763514773454	-1.53\\
-7.17704109342524	-1.44\\
-7.17771944251323	-1.35\\
-7.17751136791257	-1.26\\
-7.17820781822102	-1.17\\
-7.17768992404412	-1.08\\
-7.17703961054234	-0.99\\
-7.17702718857559	-0.900000000000001\\
-7.17596686016393	-0.810000000000001\\
-7.17718518609459	-0.719999999999999\\
-7.17814374806838	-0.629999999999999\\
-7.17752102450698	-0.54\\
-7.17770614306852	-0.45\\
-7.17738524088883	-0.359999999999999\\
-7.1771442536044	-0.27\\
-7.17835784846768	-0.18\\
-7.17672415275265	-0.0900000000000001\\
-7.17716304801165	0\\
-7.17672415275265	0.0899999999999989\\
-7.17835784846768	0.18\\
-7.1771442536044	0.27\\
-7.17738524088883	0.36\\
-7.17770614306852	0.449999999999999\\
-7.17752102450698	0.54\\
-7.17814374806838	0.630000000000001\\
-7.17718518609459	0.719999999999999\\
-7.17596686016393	0.81\\
-7.17702718857559	0.9\\
-7.17703961054234	0.990000000000001\\
-7.17768992404412	1.08\\
-7.17820781822102	1.17\\
-7.17751136791257	1.26\\
-7.17771944251323	1.35\\
-7.17704109342524	1.44\\
-7.17763514773454	1.53\\
-7.17765599160654	1.62\\
-7.17803535002242	1.71\\
-7.17774449104353	1.8\\
-7.17798100079893	1.89\\
-7.17587793602637	1.98\\
-7.17669449640087	2.07\\
-7.17800948961331	2.16\\
-7.177265561113	2.25\\
-7.17615326499459	2.34\\
-7.1771413809501	2.43\\
-7.17688674133837	2.52\\
-7.17701523313792	2.61\\
-7.17735106510209	2.7\\
-7.17628108952411	2.79\\
-7.17795061466745	2.88\\
-7.1769881200352	2.97\\
-7.17779163237446	3.06\\
-7.17763782084545	3.15\\
-7.17675277357566	3.24\\
-7.17677502012623	3.33\\
-7.17734213597931	3.42\\
-7.17624644223242	3.51\\
-7.1769951289843	3.6\\
-7.17646285301069	3.69\\
-7.17701692996982	3.78\\
-7.17659673284194	3.87\\
-7.17694687554758	3.96\\
-7.17604664351468	4.05\\
-7.17601288779868	4.14\\
-7.17552444659377	4.23\\
-7.17512309278094	4.32\\
-7.17607585498238	4.41\\
-7.1749744182885	4.5\\
-7.17589571735148	4.59\\
-7.1756519809333	4.68\\
-7.17535540811484	4.77\\
-7.17475267018083	4.86\\
-7.17448794777458	4.95\\
-7.17321678746562	5.04\\
-7.17286631810918	5.13\\
-7.17270164518548	5.22\\
-7.17124399945502	5.31\\
-7.17100854710798	5.4\\
-7.16994925490491	5.49\\
-7.16876667986837	5.58\\
-7.16686962488339	5.67\\
-7.16608571964119	5.76\\
-7.16377519265887	5.85\\
-7.16151727552735	5.94\\
-7.15932018162253	6.03\\
-7.15664601544143	6.12\\
-7.15302352208313	6.21\\
-7.14873352251056	6.3\\
-7.14362152242017	6.39\\
-7.13737217055795	6.48\\
-7.13007419344207	6.57\\
-7.12225288874396	6.66\\
-7.11072049374624	6.75\\
-7.10183803603476	6.81508579442844\\
-7.09650823536508	6.84\\
-7.07473541311415	6.93\\
-7.04548811979912	7.02\\
-7.02274795420288	7.07696717378647\\
-6.94365787237099	7.07189254164727\\
-6.91973239069251	7.02\\
-6.88769622164641	6.93\\
-6.86456779053911	6.85295141287304\\
-6.86188863599712	6.84\\
-6.84621278671928	6.75\\
-6.83180651147849	6.66\\
-6.8211258856136	6.57\\
-6.81383585015719	6.48\\
-6.80298384408655	6.39\\
-6.7973525843704	6.3\\
-6.79447899173357	6.21\\
-6.78924400111922	6.12\\
-6.78547770870722	6.06048095882692\\
-6.77937478835034	6.12\\
-6.77011399700886	6.21\\
-6.76476088033844	6.3\\
-6.75492093371515	6.39\\
-6.73641798685499	6.48\\
-6.72168808180184	6.57\\
-6.70638762687534	6.65431337327546\\
-6.70559633394956	6.66\\
-6.69040394319181	6.75\\
-6.66826265465158	6.84\\
-6.63984826565699	6.93\\
-6.62729754504345	6.96947865127391\\
-6.59675712959866	7.02\\
-6.54820746321157	7.07823218590058\\
-6.46911738137968	7.09525201402179\\
-6.3900272995478	7.03783073900246\\
-6.37604317493395	7.02\\
-6.31607917211913	6.93\\
-6.31093721771591	6.92066482635104\\
-6.27931894608223	6.84\\
-6.25285763978234	6.75\\
-6.23184713588403	6.67555902277868\\
-6.22739141571314	6.66\\
-6.20178925211169	6.57\\
-6.18807851864835	6.48\\
-6.17142210190305	6.39\\
-6.16133264546604	6.3\\
-6.15275705405214	6.24320176566443\\
-6.14741025304103	6.21\\
-6.13486986142146	6.12\\
-6.1301796495714	6.03\\
-6.12398477265017	5.94\\
-6.119897338089	5.85\\
-6.11421821102999	5.76\\
-6.10630333596652	5.67\\
-6.10177599319917	5.58\\
-6.09836493962221	5.49\\
-6.0971205413014	5.4\\
-6.09429195602505	5.31\\
-6.09244084989651	5.22\\
-6.08646344501901	5.13\\
-6.09168036182107	5.04\\
-6.08762986520331	4.95\\
-6.08414177811311	4.86\\
-6.08114072400525	4.77\\
-6.08689001270809	4.68\\
-6.08176435239194	4.59\\
-6.08500732203594	4.5\\
-6.08283273957229	4.41\\
-6.0780196049642	4.32\\
-6.0820463659079	4.23\\
-6.07864845989438	4.14\\
-6.07845843512506	4.05\\
-6.08020835287605	3.96\\
-6.07694834196859	3.87\\
-6.08277950615539	3.78\\
-6.0757951999531	3.69\\
-6.07664333934906	3.6\\
-6.07760364246664	3.51\\
-6.07758620510391	3.42\\
-6.07866657259844	3.33\\
-6.07865890930123	3.24\\
-6.07611852222793	3.15\\
-6.07694190784207	3.06\\
-6.07935774260602	2.97\\
-6.08063193775811	2.88\\
-6.08108097301618	2.79\\
-6.07366697222026	2.71112555956804\\
-6.07234242608714	2.7\\
-6.07366697222026	2.65509808136553\\
-6.07487731269758	2.61\\
-6.07960109676106	2.52\\
-6.08121672175488	2.43\\
-6.0780719620203	2.34\\
-6.08173478368893	2.25\\
-6.07709962778541	2.16\\
-6.07891095657853	2.07\\
-6.07694160708382	1.98\\
-6.07947030802545	1.89\\
-6.07751724062598	1.8\\
-6.07904709514506	1.71\\
-6.07629859322068	1.62\\
-6.08264578436419	1.53\\
-6.08196688406126	1.44\\
-6.07582232977493	1.35\\
-6.0740712908201	1.26\\
-6.07663256327883	1.17\\
-6.07717913369201	1.08\\
-6.07980670863347	0.990000000000001\\
-6.07752536889648	0.9\\
-6.07366697222026	0.814392878246581\\
-6.07343985704808	0.81\\
-6.07366697222026	0.807971502986003\\
-6.0820972453598	0.719999999999999\\
-6.07825000840059	0.630000000000001\\
-6.07876447541885	0.54\\
-6.07777614801564	0.449999999999999\\
-6.07565724370086	0.36\\
-6.07606618946036	0.27\\
-6.07665419681935	0.18\\
-6.07582661691925	0.0899999999999989\\
-6.07759992426468	0\\
-6.07582661691925	-0.0900000000000001\\
-6.07665419681935	-0.18\\
-6.07606618946036	-0.27\\
-6.07565724370086	-0.359999999999999\\
-6.07777614801564	-0.45\\
-6.07876447541885	-0.54\\
-6.07825000840059	-0.629999999999999\\
-6.0820972453598	-0.719999999999999\\
-6.07366697222026	-0.807971502986005\\
-6.07343985704808	-0.810000000000001\\
-6.07366697222026	-0.814392878246582\\
-6.07752536889648	-0.900000000000001\\
-6.07980670863347	-0.99\\
-6.07717913369201	-1.08\\
-6.07663256327883	-1.17\\
-6.0740712908201	-1.26\\
-6.07582232977493	-1.35\\
-6.08196688406126	-1.44\\
-6.08264578436419	-1.53\\
-6.07629859322068	-1.62\\
-6.07904709514506	-1.71\\
-6.07751724062598	-1.8\\
-6.07947030802545	-1.89\\
-6.07694160708382	-1.98\\
-6.07891095657853	-2.07\\
-6.07709962778541	-2.16\\
-6.08173478368893	-2.25\\
-6.0780719620203	-2.34\\
-6.08121672175488	-2.43\\
-6.07960109676106	-2.52\\
-6.07487731269758	-2.61\\
-6.07366697222026	-2.65509808136553\\
-6.07234242608714	-2.7\\
-6.07366697222026	-2.71112555956804\\
-6.08108097301618	-2.79\\
-6.08063193775811	-2.88\\
-6.07935774260602	-2.97\\
-6.07694190784207	-3.06\\
-6.07611852222793	-3.15\\
-6.07865890930123	-3.24\\
-6.07866657259844	-3.33\\
-6.07758620510391	-3.42\\
-6.07760364246664	-3.51\\
-6.07664333934906	-3.6\\
-6.0757951999531	-3.69\\
-6.08277950615539	-3.78\\
-6.07694834196859	-3.87\\
-6.08020835287605	-3.96\\
-6.07845843512506	-4.05\\
-6.07864845989438	-4.14\\
-6.0820463659079	-4.23\\
-6.0780196049642	-4.32\\
-6.08283273957229	-4.41\\
-6.08500732203594	-4.5\\
-6.08176435239194	-4.59\\
-6.08689001270809	-4.68\\
-6.08114072400525	-4.77\\
-6.08414177811311	-4.86\\
-6.08762986520331	-4.95\\
-6.09168036182107	-5.04\\
-6.08646344501901	-5.13\\
-6.09244084989651	-5.22\\
-6.09429195602505	-5.31\\
-6.0971205413014	-5.4\\
-6.09836493962221	-5.49\\
-6.10177599319917	-5.58\\
-6.10630333596652	-5.67\\
-6.11421821102999	-5.76\\
-6.119897338089	-5.85\\
-6.12398477265017	-5.94\\
-6.1301796495714	-6.03\\
-6.13486986142146	-6.12\\
-6.14741025304103	-6.21\\
-6.15275705405214	-6.24320176566443\\
-6.16133264546604	-6.3\\
-6.17142210190305	-6.39\\
-6.18807851864835	-6.48\\
-6.20178925211169	-6.57\\
-6.22739141571314	-6.66\\
-6.23184713588403	-6.67555902277868\\
-6.25285763978234	-6.75\\
-6.27931894608223	-6.84\\
-6.31093721771591	-6.92066482635103\\
-6.31607917211913	-6.93\\
-6.37604317493395	-7.02\\
-6.3900272995478	-7.03783073900245\\
-6.46911738137968	-7.09525201402178\\
-6.54820746321157	-7.07823218590058\\
-6.59675712959866	-7.02\\
-6.62729754504345	-6.96947865127391\\
-6.63984826565699	-6.93\\
-6.66826265465158	-6.84\\
-6.69040394319181	-6.75\\
-6.70559633394956	-6.66\\
-6.70638762687534	-6.65431337327546\\
-6.72168808180184	-6.57\\
-6.73641798685499	-6.48\\
-6.75492093371515	-6.39\\
-6.76476088033844	-6.3\\
-6.77011399700886	-6.21\\
-6.77937478835034	-6.12\\
-6.78547770870722	-6.06048095882692\\
-6.78924400111922	-6.12\\
-6.79447899173357	-6.21\\
-6.7973525843704	-6.3\\
-6.80298384408655	-6.39\\
-6.81383585015719	-6.48\\
-6.8211258856136	-6.57\\
-6.83180651147849	-6.66\\
-6.84621278671928	-6.75\\
-6.86188863599712	-6.84\\
-6.86456779053911	-6.85295141287303\\
-6.88769622164641	-6.93\\
-6.91973239069251	-7.02\\
2	601\\
4.06698337280875	-7.02\\
4.04986350226102	-7.02213891421579\\
3.97077342042914	-7.03543236241452\\
3.89168333859725	-7.04431290965782\\
3.81259325676537	-7.04855859574792\\
3.73350317493348	-7.05240773807764\\
3.6544130931016	-7.06015115242358\\
3.57532301126971	-7.06199615353527\\
3.49623292943783	-7.06797691358217\\
3.41714284760594	-7.07140771170123\\
3.33805276577406	-7.07372696728147\\
3.25896268394217	-7.07419139503122\\
3.17987260211029	-7.07664022786198\\
3.1007825202784	-7.08085592637541\\
3.02169243844651	-7.08022235076853\\
2.94260235661463	-7.08370501214148\\
2.86351227478275	-7.08495785107687\\
2.78442219295086	-7.08456112493031\\
2.70533211111898	-7.08888917190958\\
2.62624202928709	-7.08625107835157\\
2.54715194745521	-7.08720936080655\\
2.46806186562332	-7.09074069258467\\
2.38897178379144	-7.09050771640523\\
2.30988170195955	-7.09080500915487\\
2.23079162012767	-7.0907596178718\\
2.15170153829578	-7.09256220446647\\
2.0726114564639	-7.0923010335759\\
1.99352137463201	-7.09234843002559\\
1.91443129280013	-7.09236431677983\\
1.83534121096824	-7.0921534603102\\
1.75625112913636	-7.09407441344805\\
1.67716104730447	-7.09411540799424\\
1.59807096547259	-7.09268827466115\\
1.5189808836407	-7.09444949713149\\
1.43989080180882	-7.09493825980202\\
1.36080071997693	-7.09359561078062\\
1.28171063814505	-7.09459137315531\\
1.20262055631316	-7.09385668369047\\
1.12353047448128	-7.09451475694555\\
1.04444039264939	-7.09506810901708\\
0.965350310817507	-7.0939309217666\\
0.886260228985622	-7.09643237173542\\
0.807170147153736	-7.0957622407291\\
0.728080065321853	-7.09486910011794\\
0.648989983489967	-7.09643712141746\\
0.569899901658081	-7.09602186099712\\
0.490809819826198	-7.09569833187331\\
0.411719737994312	-7.09688572565456\\
0.332629656162427	-7.09547140903702\\
0.253539574330541	-7.09545017514462\\
0.174449492498656	-7.09588642692281\\
0.0953594106667719	-7.09538814388798\\
0.0162693288348863	-7.09497362115316\\
-0.0628207529969992	-7.09631433923196\\
-0.141910834828883	-7.09594442128033\\
-0.221000916660768	-7.09660028849374\\
-0.300090998492654	-7.09637451026146\\
-0.379181080324538	-7.09617775891821\\
-0.458271162156423	-7.09599575191969\\
-0.537361243988309	-7.09543184778791\\
-0.616451325820192	-7.09599094499855\\
-0.695541407652078	-7.09623021249412\\
-0.774631489483964	-7.09621433523159\\
-0.853721571315847	-7.09573841766118\\
-0.932811653147733	-7.09606131499787\\
-1.01190173497962	-7.09570519423139\\
-1.0909918168115	-7.09676040109032\\
-1.17008189864339	-7.09669933602013\\
-1.24917198047527	-7.09665483582936\\
-1.32826206230716	-7.09710160838557\\
-1.40735214413904	-7.09719785760368\\
-1.48644222597093	-7.09737327406956\\
-1.56553230780281	-7.09742848088412\\
-1.6446223896347	-7.09643404423516\\
-1.72371247146658	-7.09760833876297\\
-1.80280255329847	-7.09802763056709\\
-1.88189263513035	-7.09722648435314\\
-1.96098271696224	-7.09814490423092\\
-2.04007279879412	-7.09792436668824\\
-2.11916288062601	-7.09822432254548\\
-2.19825296245789	-7.09815829103794\\
-2.27734304428978	-7.09814320278926\\
-2.35643312612166	-7.09840341690496\\
-2.43552320795355	-7.09852602356284\\
-2.51461328978543	-7.09811014472349\\
-2.59370337161732	-7.09844088467845\\
-2.6727934534492	-7.0979579566617\\
-2.75188353528109	-7.09798669543012\\
-2.83097361711297	-7.09794466401333\\
-2.91006369894486	-7.09792894790933\\
-2.98915378077674	-7.09754315579411\\
-3.06824386260863	-7.09735511928351\\
-3.14733394444051	-7.09603390314156\\
-3.2264240262724	-7.09556302048935\\
-3.30551410810428	-7.09526525590053\\
-3.38460418993617	-7.09398953314311\\
-3.46369427176805	-7.09245094348479\\
-3.54278435359994	-7.09071910389965\\
-3.62187443543182	-7.08868455151261\\
-3.70096451726371	-7.08566161140901\\
-3.78005459909559	-7.08247582488195\\
-3.85914468092748	-7.08157599132191\\
-3.93823476275936	-7.07497337854523\\
-4.01732484459125	-7.0722556800561\\
-4.09641492642313	-7.06544697473093\\
-4.17550500825502	-7.05604675745602\\
-4.2545950900869	-7.0501560714545\\
-4.33368517191879	-7.03586674212333\\
-4.41277525375067	-7.02633291956117\\
-4.4563331824154	-7.02\\
-4.49186533558256	-7.01438425975134\\
-4.57095541741444	-6.9913608217558\\
-4.65004549924633	-6.97215864902437\\
-4.72270928702753	-6.93\\
-4.72913558107821	-6.92706573639487\\
-4.8082256629101	-6.89775656498952\\
-4.88731574474198	-6.84896491215274\\
-4.90418016512318	-6.84\\
-4.96640582657387	-6.80737584185923\\
-5.02200989057188	-6.75\\
-5.04549590840575	-6.72031064339427\\
-5.10506347923071	-6.66\\
-5.12458599023764	-6.63222420663222\\
-5.15422763290941	-6.57\\
-5.19701587909556	-6.48\\
-5.20367607206952	-6.46895976495187\\
-5.2566700486477	-6.39\\
-5.27072339193855	-6.3\\
-5.28276615390141	-6.25114976990995\\
-5.29552259056062	-6.21\\
-5.32458490810544	-6.12\\
-5.33567993498262	-6.03\\
-5.35222176925955	-5.94\\
-5.36185623573329	-5.87245909211436\\
-5.36516040841649	-5.85\\
-5.38411744465886	-5.76\\
-5.39416975880018	-5.67\\
-5.40418947358468	-5.58\\
-5.403397399335	-5.49\\
-5.41358123259512	-5.4\\
-5.41386358593077	-5.31\\
-5.41809294694221	-5.22\\
-5.42432292217669	-5.13\\
-5.42713573813593	-5.04\\
-5.44094631756518	-4.95625682897016\\
-5.44189490242791	-4.95\\
-5.44094631756518	-4.9347100513019\\
-5.43534250421517	-4.86\\
-5.43203674822736	-4.77\\
-5.43850412800923	-4.68\\
-5.4380869087888	-4.59\\
-5.43679239895397	-4.5\\
-5.44094631756518	-4.4730718557447\\
-5.45082701070466	-4.41\\
-5.45219328603335	-4.32\\
-5.44356113276156	-4.23\\
-5.44427388445547	-4.14\\
-5.4423146643406	-4.05\\
-5.44283365687567	-3.96\\
-5.45534745865861	-3.87\\
-5.455448115332	-3.78\\
-5.44774457716271	-3.69\\
-5.45575470488798	-3.6\\
-5.45527047975345	-3.51\\
-5.44554467621881	-3.42\\
-5.45011655610439	-3.33\\
-5.4509145899363	-3.24\\
-5.45765412706101	-3.15\\
-5.45477999035664	-3.06\\
-5.44797347282252	-2.97\\
-5.44914136709088	-2.88\\
-5.44532957771274	-2.79\\
-5.45316560268885	-2.7\\
-5.45999352072216	-2.61\\
-5.44890621575825	-2.52\\
-5.45390794807429	-2.43\\
-5.45512798170104	-2.34\\
-5.44895396106979	-2.25\\
-5.45269548016339	-2.16\\
-5.45475881538558	-2.07\\
-5.44553301648982	-1.98\\
-5.46143606184132	-1.89\\
-5.45011208508919	-1.8\\
-5.45485222162417	-1.71\\
-5.45211609034658	-1.62\\
-5.46541150494281	-1.53\\
-5.45989991366349	-1.44\\
-5.45108295490609	-1.35\\
-5.44094631756518	-1.28598714972581\\
-5.4371777672374	-1.26\\
-5.44094631756518	-1.22294979556652\\
-5.44791435225954	-1.17\\
-5.45565722571859	-1.08\\
-5.45532734502164	-0.99\\
-5.44381531508621	-0.900000000000001\\
-5.46107480229575	-0.810000000000001\\
-5.45723026227988	-0.719999999999999\\
-5.45304897996612	-0.629999999999999\\
-5.4589449827407	-0.54\\
-5.44094631756518	-0.466209727625884\\
-5.43848220108292	-0.45\\
-5.44094631756518	-0.43702370971486\\
-5.45663743986064	-0.359999999999999\\
-5.46764846103083	-0.27\\
-5.45625731053457	-0.18\\
-5.45911247495721	-0.0900000000000001\\
-5.44623090036463	0\\
-5.45911247495721	0.0899999999999989\\
-5.45625731053457	0.18\\
-5.46764846103083	0.27\\
-5.45663743986064	0.36\\
-5.44094631756518	0.437023709714859\\
-5.43848220108292	0.449999999999999\\
-5.44094631756518	0.466209727625883\\
-5.4589449827407	0.54\\
-5.45304897996612	0.630000000000001\\
-5.45723026227988	0.719999999999999\\
-5.46107480229575	0.81\\
-5.44381531508621	0.9\\
-5.45532734502164	0.990000000000001\\
-5.45565722571859	1.08\\
-5.44791435225954	1.17\\
-5.44094631756518	1.22294979556651\\
-5.4371777672374	1.26\\
-5.44094631756518	1.28598714972581\\
-5.45108295490609	1.35\\
-5.45989991366349	1.44\\
-5.46541150494281	1.53\\
-5.45211609034658	1.62\\
-5.45485222162417	1.71\\
-5.45011208508919	1.8\\
-5.46143606184132	1.89\\
-5.44553301648982	1.98\\
-5.45475881538558	2.07\\
-5.45269548016339	2.16\\
-5.44895396106979	2.25\\
-5.45512798170104	2.34\\
-5.45390794807429	2.43\\
-5.44890621575825	2.52\\
-5.45999352072216	2.61\\
-5.45316560268885	2.7\\
-5.44532957771274	2.79\\
-5.44914136709088	2.88\\
-5.44797347282252	2.97\\
-5.45477999035664	3.06\\
-5.45765412706101	3.15\\
-5.4509145899363	3.24\\
-5.45011655610439	3.33\\
-5.44554467621881	3.42\\
-5.45527047975345	3.51\\
-5.45575470488798	3.6\\
-5.44774457716271	3.69\\
-5.455448115332	3.78\\
-5.45534745865861	3.87\\
-5.44283365687567	3.96\\
-5.4423146643406	4.05\\
-5.44427388445547	4.14\\
-5.44356113276156	4.23\\
-5.45219328603335	4.32\\
-5.45082701070466	4.41\\
-5.44094631756518	4.4730718557447\\
-5.43679239895397	4.5\\
-5.4380869087888	4.59\\
-5.43850412800923	4.68\\
-5.43203674822736	4.77\\
-5.43534250421517	4.86\\
-5.44094631756518	4.9347100513019\\
-5.44189490242791	4.95\\
-5.44094631756518	4.95625682897016\\
-5.42713573813593	5.04\\
-5.42432292217669	5.13\\
-5.41809294694221	5.22\\
-5.41386358593077	5.31\\
-5.41358123259512	5.4\\
-5.403397399335	5.49\\
-5.40418947358468	5.58\\
-5.39416975880018	5.67\\
-5.38411744465886	5.76\\
-5.36516040841649	5.85\\
-5.36185623573329	5.87245909211436\\
-5.35222176925955	5.94\\
-5.33567993498262	6.03\\
-5.32458490810544	6.12\\
-5.29552259056062	6.21\\
-5.28276615390141	6.25114976990995\\
-5.27072339193855	6.3\\
-5.2566700486477	6.39\\
-5.20367607206952	6.46895976495187\\
-5.19701587909556	6.48\\
-5.15422763290941	6.57\\
-5.12458599023764	6.63222420663222\\
-5.10506347923071	6.66\\
-5.04549590840575	6.72031064339427\\
-5.02200989057188	6.75\\
-4.96640582657387	6.80737584185923\\
-4.90418016512318	6.84\\
-4.88731574474198	6.84896491215274\\
-4.8082256629101	6.89775656498952\\
-4.72913558107821	6.92706573639487\\
-4.72270928702753	6.93\\
-4.65004549924633	6.97215864902437\\
-4.57095541741444	6.9913608217558\\
-4.49186533558256	7.01438425975134\\
-4.4563331824154	7.02\\
-4.41277525375067	7.02633291956117\\
-4.33368517191879	7.03586674212333\\
-4.2545950900869	7.05015607145451\\
-4.17550500825502	7.05604675745603\\
-4.09641492642313	7.06544697473094\\
-4.01732484459125	7.0722556800561\\
-3.93823476275936	7.07497337854523\\
-3.85914468092748	7.08157599132191\\
-3.78005459909559	7.08247582488195\\
-3.70096451726371	7.08566161140901\\
-3.62187443543182	7.08868455151261\\
-3.54278435359994	7.09071910389965\\
-3.46369427176805	7.09245094348479\\
-3.38460418993617	7.09398953314311\\
-3.30551410810428	7.09526525590053\\
-3.2264240262724	7.09556302048935\\
-3.14733394444051	7.09603390314156\\
-3.06824386260863	7.09735511928352\\
-2.98915378077674	7.09754315579411\\
-2.91006369894486	7.09792894790933\\
-2.83097361711297	7.09794466401333\\
-2.75188353528109	7.09798669543012\\
-2.6727934534492	7.0979579566617\\
-2.59370337161732	7.09844088467845\\
-2.51461328978543	7.09811014472349\\
-2.43552320795355	7.09852602356284\\
-2.35643312612166	7.09840341690496\\
-2.27734304428978	7.09814320278926\\
-2.19825296245789	7.09815829103794\\
-2.11916288062601	7.09822432254548\\
-2.04007279879412	7.09792436668825\\
-1.96098271696224	7.09814490423092\\
-1.88189263513035	7.09722648435314\\
-1.80280255329847	7.09802763056709\\
-1.72371247146658	7.09760833876297\\
-1.6446223896347	7.09643404423516\\
-1.56553230780281	7.09742848088412\\
-1.48644222597093	7.09737327406957\\
-1.40735214413904	7.09719785760368\\
-1.32826206230716	7.09710160838557\\
-1.24917198047527	7.09665483582936\\
-1.17008189864339	7.09669933602013\\
-1.0909918168115	7.09676040109032\\
-1.01190173497962	7.09570519423139\\
-0.932811653147733	7.09606131499787\\
-0.853721571315847	7.09573841766118\\
-0.774631489483964	7.0962143352316\\
-0.695541407652078	7.09623021249412\\
-0.616451325820192	7.09599094499855\\
-0.537361243988309	7.09543184778791\\
-0.458271162156423	7.09599575191969\\
-0.379181080324538	7.09617775891821\\
-0.300090998492654	7.09637451026146\\
-0.221000916660768	7.09660028849374\\
-0.141910834828883	7.09594442128033\\
-0.0628207529969992	7.09631433923197\\
0.0162693288348863	7.09497362115316\\
0.0953594106667719	7.09538814388798\\
0.174449492498656	7.09588642692281\\
0.253539574330541	7.09545017514462\\
0.332629656162427	7.09547140903703\\
0.411719737994312	7.09688572565456\\
0.490809819826198	7.09569833187331\\
0.569899901658081	7.09602186099712\\
0.648989983489967	7.09643712141746\\
0.728080065321853	7.09486910011794\\
0.807170147153736	7.0957622407291\\
0.886260228985622	7.09643237173542\\
0.965350310817507	7.0939309217666\\
1.04444039264939	7.09506810901708\\
1.12353047448128	7.09451475694555\\
1.20262055631316	7.09385668369047\\
1.28171063814505	7.09459137315531\\
1.36080071997693	7.09359561078062\\
1.43989080180882	7.09493825980203\\
1.5189808836407	7.09444949713149\\
1.59807096547259	7.09268827466115\\
1.67716104730447	7.09411540799424\\
1.75625112913636	7.09407441344805\\
1.83534121096824	7.0921534603102\\
1.91443129280013	7.09236431677983\\
1.99352137463201	7.09234843002559\\
2.0726114564639	7.09230103357591\\
2.15170153829578	7.09256220446647\\
2.23079162012767	7.0907596178718\\
2.30988170195955	7.09080500915487\\
2.38897178379144	7.09050771640523\\
2.46806186562332	7.09074069258467\\
2.54715194745521	7.08720936080655\\
2.62624202928709	7.08625107835157\\
2.70533211111898	7.08888917190958\\
2.78442219295086	7.08456112493031\\
2.86351227478275	7.08495785107687\\
2.94260235661463	7.08370501214148\\
3.02169243844651	7.08022235076853\\
3.1007825202784	7.08085592637542\\
3.17987260211029	7.07664022786198\\
3.25896268394217	7.07419139503122\\
3.33805276577406	7.07372696728147\\
3.41714284760594	7.07140771170124\\
3.49623292943783	7.06797691358217\\
3.57532301126971	7.06199615353527\\
3.6544130931016	7.06015115242358\\
3.73350317493348	7.05240773807764\\
3.81259325676537	7.04855859574793\\
3.89168333859725	7.04431290965782\\
3.97077342042914	7.03543236241452\\
4.04986350226102	7.02213891421579\\
4.06698337280875	7.02\\
4.12895358409291	7.01144159392097\\
4.20804366592479	7.00100081164243\\
4.28713374775667	6.97959844131239\\
4.36622382958856	6.96317128437216\\
4.44531391142045	6.94819737801252\\
4.49728592227378	6.93\\
4.52440399325233	6.92128422297806\\
4.60349407508422	6.88959550473393\\
4.6825841569161	6.85882948955055\\
4.72174619267459	6.84\\
4.76167423874798	6.81421559979197\\
4.83727496777322	6.75\\
4.84076432057987	6.74705630539746\\
4.91355358976879	6.66\\
4.91985440241175	6.65316804778404\\
4.99894448424364	6.57759160993636\\
5.00400523363517	6.57\\
5.05197293623637	6.48\\
5.07803456607553	6.43270015935539\\
5.09877338364127	6.39\\
5.15343006543589	6.3\\
5.15281758993399	6.21\\
5.15712464790741	6.19063141664638\\
5.18007316627516	6.12\\
5.20368747579521	6.03\\
5.22267358919288	5.94\\
5.2362147297393	5.87884960898518\\
5.24206236257192	5.85\\
5.25570721945893	5.76\\
5.26575290442036	5.67\\
5.27301755713263	5.58\\
5.28582503987558	5.49\\
5.28916223679741	5.4\\
5.28902592255932	5.31\\
5.30940385942667	5.22\\
5.29958066885607	5.13\\
5.30586940943668	5.04\\
5.30552785591324	4.95\\
5.30873277759829	4.86\\
5.31530481157118	4.77932407433462\\
5.3160661683502	4.77\\
5.32160139679614	4.68\\
5.32009275970454	4.59\\
5.32607347134537	4.5\\
5.32772348530907	4.41\\
5.31836464502036	4.32\\
5.32359369885323	4.23\\
5.32962189611814	4.14\\
5.33587541796425	4.05\\
5.32698763634473	3.96\\
5.33090289483155	3.87\\
5.32995568315648	3.78\\
5.32316791585901	3.69\\
5.33278572925946	3.6\\
5.33530163692004	3.51\\
5.32197837870641	3.42\\
5.33208406518793	3.33\\
5.33786228906447	3.24\\
5.32942807208317	3.15\\
5.32382972550029	3.06\\
5.33416999731303	2.97\\
5.32928115616782	2.88\\
5.33546915944865	2.79\\
5.32858392343812	2.7\\
5.33194964315702	2.61\\
5.32269410381099	2.52\\
5.33925539099509	2.43\\
5.33281238755987	2.34\\
5.32509125199801	2.25\\
5.33506320731769	2.16\\
5.33209468148573	2.07\\
5.33685655337049	1.98\\
5.33784598940866	1.89\\
5.33278394494061	1.8\\
5.3285306622811	1.71\\
5.33605146294232	1.62\\
5.3382358212089	1.53\\
5.33910236596846	1.44\\
5.33076191580912	1.35\\
5.33914543786579	1.26\\
5.33701369691792	1.17\\
5.31909187963656	1.08\\
5.33825471906763	0.990000000000001\\
5.33917084943443	0.9\\
5.33594897325419	0.81\\
5.33492582204485	0.719999999999999\\
5.33541674678517	0.630000000000001\\
5.3330239199856	0.54\\
5.32819482021235	0.449999999999999\\
5.3349637712298	0.36\\
5.33557497991974	0.27\\
5.32815569290249	0.18\\
5.3274114777259	0.0899999999999989\\
5.32655789423885	0\\
5.3274114777259	-0.0900000000000001\\
5.32815569290249	-0.18\\
5.33557497991974	-0.27\\
5.3349637712298	-0.359999999999999\\
5.32819482021235	-0.45\\
5.3330239199856	-0.54\\
5.33541674678517	-0.629999999999999\\
5.33492582204485	-0.719999999999999\\
5.33594897325419	-0.810000000000001\\
5.33917084943443	-0.900000000000001\\
5.33825471906763	-0.99\\
5.31909187963656	-1.08\\
5.33701369691792	-1.17\\
5.33914543786579	-1.26\\
5.33076191580912	-1.35\\
5.33910236596846	-1.44\\
5.3382358212089	-1.53\\
5.33605146294232	-1.62\\
5.3285306622811	-1.71\\
5.33278394494061	-1.8\\
5.33784598940866	-1.89\\
5.33685655337049	-1.98\\
5.33209468148573	-2.07\\
5.33506320731769	-2.16\\
5.32509125199801	-2.25\\
5.33281238755987	-2.34\\
5.33925539099509	-2.43\\
5.32269410381099	-2.52\\
5.33194964315702	-2.61\\
5.32858392343812	-2.7\\
5.33546915944865	-2.79\\
5.32928115616782	-2.88\\
5.33416999731303	-2.97\\
5.32382972550029	-3.06\\
5.32942807208317	-3.15\\
5.33786228906447	-3.24\\
5.33208406518793	-3.33\\
5.32197837870641	-3.42\\
5.33530163692004	-3.51\\
5.33278572925946	-3.6\\
5.32316791585901	-3.69\\
5.32995568315648	-3.78\\
5.33090289483155	-3.87\\
5.32698763634473	-3.96\\
5.33587541796425	-4.05\\
5.32962189611814	-4.14\\
5.32359369885323	-4.23\\
5.31836464502036	-4.32\\
5.32772348530907	-4.41\\
5.32607347134537	-4.5\\
5.32009275970454	-4.59\\
5.32160139679614	-4.68\\
5.3160661683502	-4.77\\
5.31530481157118	-4.77932407433462\\
5.30873277759829	-4.86\\
5.30552785591324	-4.95\\
5.30586940943668	-5.04\\
5.29958066885607	-5.13\\
5.30940385942667	-5.22\\
5.28902592255932	-5.31\\
5.28916223679741	-5.4\\
5.28582503987558	-5.49\\
5.27301755713263	-5.58\\
5.26575290442036	-5.67\\
5.25570721945893	-5.76\\
5.24206236257192	-5.85\\
5.2362147297393	-5.87884960898518\\
5.22267358919288	-5.94\\
5.20368747579521	-6.03\\
5.18007316627516	-6.12\\
5.15712464790741	-6.19063141664638\\
5.15281758993399	-6.21\\
5.15343006543589	-6.3\\
5.09877338364127	-6.39\\
5.07803456607553	-6.4327001593554\\
5.05197293623637	-6.48\\
5.00400523363517	-6.57\\
4.99894448424364	-6.57759160993636\\
4.91985440241175	-6.65316804778404\\
4.91355358976879	-6.66\\
4.84076432057987	-6.74705630539746\\
4.83727496777322	-6.75\\
4.76167423874798	-6.81421559979197\\
4.72174619267459	-6.84\\
4.6825841569161	-6.85882948955055\\
4.60349407508422	-6.88959550473393\\
4.52440399325233	-6.92128422297806\\
4.49728592227378	-6.93\\
4.44531391142045	-6.94819737801252\\
4.36622382958856	-6.96317128437216\\
4.28713374775667	-6.97959844131239\\
4.20804366592479	-7.00100081164243\\
4.12895358409291	-7.01144159392096\\
4.06698337280875	-7.02\\
2	265\\
-8.84169454508203	-5.85\\
-8.84181983633623	-5.87496122684226\\
-8.84216069007897	-5.85\\
-8.8440122684972	-5.76\\
-8.84614731158896	-5.67\\
-8.84731748169902	-5.58\\
-8.84856170342858	-5.49\\
-8.8490792348801	-5.4\\
-8.8496377723742	-5.31\\
-8.8516300793179	-5.22\\
-8.85117293983141	-5.13\\
-8.85121829668257	-5.04\\
-8.85270485718626	-4.95\\
-8.85287588729336	-4.86\\
-8.85315820125999	-4.77\\
-8.85351807467411	-4.68\\
-8.85449078303361	-4.59\\
-8.85480903650052	-4.5\\
-8.85441481817303	-4.41\\
-8.85503902395549	-4.32\\
-8.85513598435743	-4.23\\
-8.85574329681999	-4.14\\
-8.85560395037982	-4.05\\
-8.85573488270558	-3.96\\
-8.85671420667456	-3.87\\
-8.85612816134184	-3.78\\
-8.85616251148173	-3.69\\
-8.85649875231982	-3.6\\
-8.85578087671163	-3.51\\
-8.85626844705974	-3.42\\
-8.8564154453899	-3.33\\
-8.85549095720191	-3.24\\
-8.85561173555987	-3.15\\
-8.85628733788783	-3.06\\
-8.8563594806018	-2.97\\
-8.85646301802296	-2.88\\
-8.85609553948493	-2.79\\
-8.85660840513174	-2.7\\
-8.85543965999363	-2.61\\
-8.85627317520889	-2.52\\
-8.85589304057217	-2.43\\
-8.85640918424175	-2.34\\
-8.85595090917755	-2.25\\
-8.8556767313946	-2.16\\
-8.85535447175667	-2.07\\
-8.85613548466195	-1.98\\
-8.85620462694413	-1.89\\
-8.85631356730071	-1.8\\
-8.85597048474041	-1.71\\
-8.85632665905662	-1.62\\
-8.85640996284799	-1.53\\
-8.85667612650473	-1.44\\
-8.85668646365951	-1.35\\
-8.85695511003477	-1.26\\
-8.85599289691319	-1.17\\
-8.85554322896641	-1.08\\
-8.85634935037701	-0.99\\
-8.85646592544063	-0.900000000000001\\
-8.85655264378726	-0.810000000000001\\
-8.85644811768839	-0.719999999999999\\
-8.85590380247478	-0.629999999999999\\
-8.85607590808683	-0.54\\
-8.8570277254347	-0.45\\
-8.85633667403888	-0.359999999999999\\
-8.85596042056693	-0.27\\
-8.8563167798267	-0.18\\
-8.85578786003623	-0.0900000000000001\\
-8.85673685405116	0\\
-8.85578786003623	0.0899999999999989\\
-8.8563167798267	0.18\\
-8.85596042056693	0.27\\
-8.85633667403888	0.36\\
-8.8570277254347	0.449999999999999\\
-8.85607590808683	0.54\\
-8.85590380247478	0.630000000000001\\
-8.85644811768839	0.719999999999999\\
-8.85655264378726	0.81\\
-8.85646592544063	0.9\\
-8.85634935037701	0.990000000000001\\
-8.85554322896641	1.08\\
-8.85599289691319	1.17\\
-8.85695511003477	1.26\\
-8.85668646365951	1.35\\
-8.85667612650473	1.44\\
-8.85640996284799	1.53\\
-8.85632665905662	1.62\\
-8.85597048474041	1.71\\
-8.85631356730071	1.8\\
-8.85620462694413	1.89\\
-8.85613548466195	1.98\\
-8.85535447175667	2.07\\
-8.8556767313946	2.16\\
-8.85595090917755	2.25\\
-8.85640918424175	2.34\\
-8.85589304057217	2.43\\
-8.85627317520889	2.52\\
-8.85543965999363	2.61\\
-8.85660840513174	2.7\\
-8.85609553948493	2.79\\
-8.85646301802296	2.88\\
-8.8563594806018	2.97\\
-8.85628733788783	3.06\\
-8.85561173555987	3.15\\
-8.85549095720191	3.24\\
-8.8564154453899	3.33\\
-8.85626844705974	3.42\\
-8.85578087671163	3.51\\
-8.85649875231982	3.6\\
-8.85616251148173	3.69\\
-8.85612816134184	3.78\\
-8.85671420667456	3.87\\
-8.85573488270558	3.96\\
-8.85560395037982	4.05\\
-8.85574329681999	4.14\\
-8.85513598435743	4.23\\
-8.85503902395549	4.32\\
-8.85441481817303	4.41\\
-8.85480903650052	4.5\\
-8.85449078303361	4.59\\
-8.85351807467411	4.68\\
-8.85315820125999	4.77\\
-8.85287588729336	4.86\\
-8.85270485718626	4.95\\
-8.85121829668257	5.04\\
-8.85117293983141	5.13\\
-8.8516300793179	5.22\\
-8.8496377723742	5.31\\
-8.8490792348801	5.4\\
-8.84856170342858	5.49\\
-8.84731748169902	5.58\\
-8.84614731158896	5.67\\
-8.8440122684972	5.76\\
-8.84216069007897	5.85\\
-8.84181983633623	5.87496122684226\\
-8.84169454508203	5.85\\
-8.84099643051874	5.76\\
-8.84018048797912	5.67\\
-8.83972734329225	5.58\\
-8.83922703023381	5.49\\
-8.83907356751643	5.4\\
-8.83882390546895	5.31\\
-8.83804646229689	5.22\\
-8.83826000717884	5.13\\
-8.83825200104677	5.04\\
-8.83766340688752	4.95\\
-8.8376138689936	4.86\\
-8.83745152441626	4.77\\
-8.83734536728472	4.68\\
-8.83686661651286	4.59\\
-8.83676293215268	4.5\\
-8.83693092106608	4.41\\
-8.83664061581995	4.32\\
-8.83665252991276	4.23\\
-8.83642346063137	4.14\\
-8.83648815687365	4.05\\
-8.8363889884097	3.96\\
-8.8360098802971	3.87\\
-8.83629067891197	3.78\\
-8.83623108099499	3.69\\
-8.83610309310262	3.6\\
-8.83638269116289	3.51\\
-8.83617011026804	3.42\\
-8.83615811632952	3.33\\
-8.83651721008357	3.24\\
-8.8364642643154	3.15\\
-8.83620649237779	3.06\\
-8.83612951268649	2.97\\
-8.83613068485544	2.88\\
-8.83628680707614	2.79\\
-8.83604741072894	2.7\\
-8.83658110373437	2.61\\
-8.83619373877586	2.52\\
-8.83635724608191	2.43\\
-8.83614168595773	2.34\\
-8.83620817550116	2.25\\
-8.83646418690819	2.16\\
-8.83661655618474	2.07\\
-8.83620062304325	1.98\\
-8.83615774613287	1.89\\
-8.83616351856075	1.8\\
-8.83638820044293	1.71\\
-8.83618222547056	1.62\\
-8.8361284842584	1.53\\
-8.83602633315201	1.44\\
-8.83601359163291	1.35\\
-8.83592713911784	1.26\\
-8.83634991992144	1.17\\
-8.83647178118897	1.08\\
-8.83618124581677	0.990000000000001\\
-8.83613790872953	0.9\\
-8.83612409051412	0.81\\
-8.83614997225228	0.719999999999999\\
-8.83631797137978	0.630000000000001\\
-8.83627773046089	0.54\\
-8.83589612526394	0.449999999999999\\
-8.83621319390702	0.36\\
-8.83634349024193	0.27\\
-8.83616933684003	0.18\\
-8.83643130557412	0.0899999999999989\\
-8.83588871203246	0\\
-8.83643130557412	-0.0900000000000001\\
-8.83616933684003	-0.18\\
-8.83634349024193	-0.27\\
-8.83621319390702	-0.359999999999999\\
-8.83589612526394	-0.45\\
-8.83627773046089	-0.54\\
-8.83631797137978	-0.629999999999999\\
-8.83614997225228	-0.719999999999999\\
-8.83612409051412	-0.810000000000001\\
-8.83613790872953	-0.900000000000001\\
-8.83618124581677	-0.99\\
-8.83647178118897	-1.08\\
-8.83634991992144	-1.17\\
-8.83592713911784	-1.26\\
-8.83601359163291	-1.35\\
-8.83602633315201	-1.44\\
-8.8361284842584	-1.53\\
-8.83618222547056	-1.62\\
-8.83638820044293	-1.71\\
-8.83616351856075	-1.8\\
-8.83615774613287	-1.89\\
-8.83620062304325	-1.98\\
-8.83661655618474	-2.07\\
-8.83646418690819	-2.16\\
-8.83620817550116	-2.25\\
-8.83614168595773	-2.34\\
-8.83635724608191	-2.43\\
-8.83619373877586	-2.52\\
-8.83658110373437	-2.61\\
-8.83604741072894	-2.7\\
-8.83628680707614	-2.79\\
-8.83613068485544	-2.88\\
-8.83612951268649	-2.97\\
-8.83620649237779	-3.06\\
-8.8364642643154	-3.15\\
-8.83651721008357	-3.24\\
-8.83615811632952	-3.33\\
-8.83617011026804	-3.42\\
-8.83638269116289	-3.51\\
-8.83610309310262	-3.6\\
-8.83623108099499	-3.69\\
-8.83629067891197	-3.78\\
-8.8360098802971	-3.87\\
-8.8363889884097	-3.96\\
-8.83648815687365	-4.05\\
-8.83642346063137	-4.14\\
-8.83665252991276	-4.23\\
-8.83664061581995	-4.32\\
-8.83693092106608	-4.41\\
-8.83676293215268	-4.5\\
-8.83686661651286	-4.59\\
-8.83734536728472	-4.68\\
-8.83745152441626	-4.77\\
-8.8376138689936	-4.86\\
-8.83766340688752	-4.95\\
-8.83825200104677	-5.04\\
-8.83826000717884	-5.13\\
-8.83804646229689	-5.22\\
-8.83882390546895	-5.31\\
-8.83907356751643	-5.4\\
-8.83922703023381	-5.49\\
-8.83972734329225	-5.58\\
-8.84018048797912	-5.67\\
-8.84099643051874	-5.76\\
-8.84169454508203	-5.85\\
};
\end{axis}
\end{tikzpicture}%
% This file was created by matlab2tikz.
%
\tikzsetnextfilename{waveIdealPS}
\definecolor{mycolor1}{rgb}{0.00000,0.44700,0.74100}%
%
\begin{tikzpicture}

\begin{axis}[%
colormap={mymap}{[1pt] rgb(0pt)=(0.19376,0.12032,0.52824); rgb(1pt)=(0.22328,0.21004,0.72516); rgb(2pt)=(0.21176,0.3224,0.7948); rgb(3pt)=(0.14128,0.43824,0.76304); rgb(4pt)=(0.0868,0.53352,0.69872); rgb(5pt)=(0.05632,0.59656,0.58064); rgb(6pt)=(0.22472,0.63712,0.42128); rgb(7pt)=(0.49832,0.62948,0.20572); rgb(8pt)=(0.73472,0.58464,0.1512); rgb(9pt)=(0.78568,0.66488,0.14416); rgb(10pt)=(0.78152,0.78712,0.0644)},
xmin=-9,
xmax=9,
xtick={-12.9433116758173,-10.3695107414678,-8.42663571916794,-6.49920831371325,-4.05740015411154,-2.47713871630925,0,2.47713871630925,4.05740015411154,6.49920831371325,8.42663571916794,10.3695107414678,12.9433116758173},
xticklabels={{-50},{-20},{-10},{ -5},{ -2},{ -1},{  0},{  1},{  2},{  5},{ 10},{ 20},{ 50}},
xticklabel style={rotate=40},
xlabel style={font=\color{white!15!black}},
xlabel={$\Re\lambda$},
ymin=-9,
ymax=9,
ytick={-8.30739802950736,-6.65545690764541,-5.40846258841032,-4.17138300390114,-2.60415872609827,-1.58990046798703,0,1.58990046798703,2.60415872609827,4.17138300390114,5.40846258841032,6.65545690764541,8.30739802950736},
yticklabels={{-50},{-20},{-10},{ -5},{ -2},{ -1},{  0},{  1},{  2},{  5},{ 10},{ 20},{ 50}},
ylabel style={font=\color{white!15!black}},
ylabel={$\Im\lambda$},
axis background/.style={fill=white},
title style={font=\bfseries},
title={spectrum (dots) and pseudo-spectra contours, via psa()},
xmajorgrids,
ymajorgrids,
\extraAxisOptions
]
\addplot [color=mycolor1, draw=none, mark=*, mark options={solid, mycolor1}, forget plot]
  table[row sep=crcr]{%
8.11285798009452e-15	8.67000731912958\\
8.11285798009452e-15	-8.67000731912958\\
1.99701119510019e-14	8.63317199284656\\
1.99701119510019e-14	-8.63317199284656\\
9.98505597550095e-15	7.86018981399197\\
9.98505597550095e-15	-7.86018981399197\\
1.24813199693762e-15	7.70409551232861\\
1.24813199693762e-15	-7.70409551232861\\
-9.98505597550095e-15	8.66153417816344\\
-9.98505597550095e-15	-8.66153417816344\\
-1.49775839632514e-14	8.66153417816344\\
-1.49775839632514e-14	-8.66153417816344\\
-9.98505597550095e-15	8.65257387218033\\
-9.98505597550095e-15	-8.65257387218033\\
0	8.65257387218032\\
0	-8.65257387218032\\
0	8.64312153801784\\
0	-8.64312153801784\\
-9.98505597550095e-15	8.64312153801784\\
-9.98505597550095e-15	-8.64312153801784\\
-5.92862698545369e-15	2.60389332021975\\
-5.92862698545369e-15	-2.60389332021975\\
-2.24663759448771e-14	7.74508042952248\\
-2.24663759448771e-14	-7.74508042952248\\
-9.98505597550095e-15	7.74508042952248\\
-9.98505597550095e-15	-7.74508042952248\\
-4.99252798775047e-15	7.78471994097705\\
-4.99252798775047e-15	-7.78471994097705\\
-4.99252798775047e-15	7.78471994097704\\
-4.99252798775047e-15	-7.78471994097704\\
-1.99701119510019e-14	7.8230718746309\\
-1.99701119510019e-14	-7.8230718746309\\
-1.04531054743526e-14	7.82307187463091\\
-1.04531054743526e-14	-7.82307187463091\\
-3.35435474176985e-15	1.58984801351519\\
-3.35435474176985e-15	-1.58984801351519\\
-1.95020624521503e-15	1.58984801351519\\
-1.95020624521503e-15	-1.58984801351519\\
-2.91704373443047e-15	0\\
4.03003027343326e-15	0\\
5.49178078652552e-14	8.36083530853762\\
5.49178078652552e-14	-8.36083530853762\\
0	8.36083530853763\\
0	-8.36083530853763\\
-3.99402239020038e-14	8.36083530853763\\
-3.99402239020038e-14	-8.36083530853763\\
-4.99252798775047e-15	8.36083530853763\\
-4.99252798775047e-15	-8.36083530853763\\
-1.99701119510019e-14	8.36083530853763\\
-1.99701119510019e-14	-8.36083530853763\\
-3.99402239020038e-14	8.36083530853763\\
-3.99402239020038e-14	-8.36083530853763\\
-1.99701119510019e-14	8.36083530853763\\
-1.99701119510019e-14	-8.36083530853763\\
-5.74140718591304e-14	8.36083530853762\\
-5.74140718591304e-14	-8.36083530853762\\
7.48879198162571e-15	6.62572364336256\\
7.48879198162571e-15	-6.62572364336256\\
-1.37294519663138e-14	6.62572364336257\\
-1.37294519663138e-14	-6.62572364336257\\
3.12032999234405e-16	6.62572364336256\\
3.12032999234405e-16	-6.62572364336256\\
-4.99252798775047e-15	6.62572364336255\\
-4.99252798775047e-15	-6.62572364336255\\
4.99252798775048e-15	6.62572364336256\\
4.99252798775048e-15	-6.62572364336256\\
-4.99252798775048e-15	6.62572364336256\\
-4.99252798775048e-15	-6.62572364336256\\
1.87219799540643e-14	6.62572364336256\\
1.87219799540643e-14	-6.62572364336256\\
-2.18423099464083e-15	6.62572364336256\\
-2.18423099464083e-15	-6.62572364336256\\
};
\addplot[contour prepared, contour prepared format=matlab] table[row sep=crcr] {%
%
-8	5\\
0.0214851139680976	0\\
0	-0.0210020847235491\\
-0.0214851139680972	0\\
0	0.0210020847235495\\
0.0214851139680976	0\\
-7	5\\
0.0322276767638916	0\\
0	-0.0315031327664089\\
-0.032227676763891	0\\
0	0.0315031327664095\\
0.0322276767638916	0\\
-6	5\\
0.0429702395596856	0\\
0	-0.0420041808092688\\
-0.0429702395596847	0\\
0	0.0420041808092695\\
0.0429702395596856	0\\
-5	5\\
0.0537128023554795	0\\
0	-0.0525052288521285\\
-0.0537128023554785	0\\
0	0.0525052288521295\\
0.0537128023554795	0\\
-4	5\\
0.0644553651512735	0\\
0	-0.0630062768949884\\
-0.0644553651512722	0\\
0	0.0630062768949895\\
0.0644553651512735	0\\
-3	5\\
0.0751979279470675	0\\
0	-0.0735073249378482\\
-0.075197927947066	0\\
0	0.0735073249378495\\
0.0751979279470675	0\\
-2	5\\
0.0175905463329023	-2.61\\
0	-2.61929562480267\\
-0.0175905463329039	-2.61\\
0	-2.60005836939336\\
0.0175905463329023	-2.61\\
-2	5\\
0.0859404907428614	0\\
0	-0.084008372980708\\
-0.0859404907428597	0\\
0	0.0840083729807095\\
0.0859404907428614	0\\
-2	5\\
0.0175905463329023	2.61\\
0	2.60005836939336\\
-0.0175905463329039	2.61\\
0	2.61929562480267\\
0.0175905463329023	2.61\\
-1	13\\
0.254993060582121	-8.64\\
0.18	-8.64357376011285\\
0.0900000000000004	-8.64666238143251\\
0	-8.64784662815591\\
-0.0899999999999992	-8.64664872241612\\
-0.180000000000001	-8.64351857930341\\
-0.254993095861339	-8.64\\
-0.180000000000001	-8.63629434575866\\
-0.0899999999999992	-8.63309329994451\\
0	-8.63181979613297\\
0.0900000000000004	-8.6331291491903\\
0.18	-8.63630623179356\\
0.254993060582121	-8.64\\
-1	13\\
0.246174972226958	-7.74\\
0.18	-7.77400671555604\\
0.0900000000000004	-7.79929970172817\\
0	-7.80686394974038\\
-0.0899999999999992	-7.79929970343779\\
-0.180000000000001	-7.77400671464466\\
-0.246174977821835	-7.74\\
-0.180000000000001	-7.73597096253989\\
-0.0899999999999992	-7.7322435886297\\
0	-7.73071705994096\\
0.0900000000000004	-7.73231743340335\\
0.18	-7.73598758958553\\
0.246174972226958	-7.74\\
-1	17\\
0.284221196268088	-2.61\\
0.27	-2.61858642789482\\
0.18	-2.66313030872511\\
0.0900000000000004	-2.68510838620566\\
0	-2.6938783930228\\
-0.0899999999999992	-2.68510838620567\\
-0.180000000000001	-2.66313030923183\\
-0.27	-2.61858642793091\\
-0.284221196616315	-2.61\\
-0.27	-2.59903133775913\\
-0.180000000000001	-2.54530760428588\\
-0.0899999999999992	-2.52425101062484\\
0	-2.52029242068036\\
0.0900000000000004	-2.52425101062483\\
0.18	-2.54530760506393\\
0.27	-2.59903133846196\\
0.284221196268088	-2.61\\
-1	27\\
0.263319189667969	-1.71\\
0.18	-1.73944953886266\\
0.0900000000000004	-1.75663291499406\\
0	-1.76193760062456\\
-0.0899999999999992	-1.75663291424096\\
-0.180000000000001	-1.7394495391832\\
-0.263319190144214	-1.71\\
-0.27	-1.70658708231493\\
-0.359999999999999	-1.6329996310834\\
-0.372278195634417	-1.62\\
-0.359999999999999	-1.56000505762842\\
-0.354535430255371	-1.53\\
-0.27	-1.46803390129204\\
-0.198469474336045	-1.44\\
-0.180000000000001	-1.43356466640169\\
-0.0899999999999992	-1.41373218042658\\
0	-1.40751192397219\\
0.0900000000000004	-1.41373218073825\\
0.18	-1.43356466640167\\
0.198469474336119	-1.44\\
0.27	-1.46803390124951\\
0.354535429304443	-1.53\\
0.36	-1.56000506180831\\
0.372278194384832	-1.62\\
0.36	-1.63299963098272\\
0.27	-1.70658708207154\\
0.263319189667969	-1.71\\
-1	29\\
0.250647636296709	-0.180000000000001\\
0.18	-0.214490968203599\\
0.0900000000000004	-0.238023355422069\\
0	-0.24502237391748\\
-0.0899999999999992	-0.238023355353819\\
-0.180000000000001	-0.214490968078089\\
-0.250647636206339	-0.180000000000001\\
-0.27	-0.167791846714283\\
-0.34998380562643	-0.0899999999999989\\
-0.359999999999999	-0.0558083166333886\\
-0.377466261512926	0\\
-0.359999999999999	0.0558083166333896\\
-0.34998380562643	0.0900000000000005\\
-0.27	0.167791846714282\\
-0.250647636206339	0.18\\
-0.180000000000001	0.214490968078088\\
-0.0899999999999992	0.238023355353818\\
0	0.245022373917479\\
0.0900000000000004	0.238023355422068\\
0.18	0.214490968203598\\
0.250647636296709	0.18\\
0.27	0.167791846715666\\
0.349983805960343	0.0900000000000005\\
0.36	0.055808317035941\\
0.377466261205084	0\\
0.36	-0.0558083170359401\\
0.349983805960343	-0.0899999999999989\\
0.27	-0.167791846715667\\
0.250647636296709	-0.180000000000001\\
-1	27\\
0.198469474336119	1.44\\
0.18	1.43356466640167\\
0.0900000000000004	1.41373218073825\\
0	1.40751192397219\\
-0.0899999999999992	1.41373218042658\\
-0.180000000000001	1.43356466640169\\
-0.198469474336045	1.44\\
-0.27	1.46803390129204\\
-0.354535430255371	1.53\\
-0.359999999999999	1.56000505762842\\
-0.372278195634417	1.62\\
-0.359999999999999	1.6329996310834\\
-0.27	1.70658708231493\\
-0.263319190144214	1.71\\
-0.180000000000001	1.73944953918321\\
-0.0899999999999992	1.75663291424096\\
0	1.76193760062456\\
0.0900000000000004	1.75663291499406\\
0.18	1.73944953886266\\
0.263319189667969	1.71\\
0.27	1.70658708207154\\
0.36	1.63299963098272\\
0.372278194384832	1.62\\
0.36	1.56000506180831\\
0.354535429304443	1.53\\
0.27	1.46803390124951\\
0.198469474336119	1.44\\
-1	17\\
0.284221196268088	2.61\\
0.27	2.59903133846196\\
0.18	2.54530760506393\\
0.0900000000000004	2.52425101062483\\
0	2.52029242068036\\
-0.0899999999999992	2.52425101062483\\
-0.180000000000001	2.54530760428588\\
-0.27	2.59903133775913\\
-0.284221196616315	2.61\\
-0.27	2.61858642793091\\
-0.180000000000001	2.66313030923183\\
-0.0899999999999992	2.68510838620567\\
0	2.6938783930228\\
0.0900000000000004	2.68510838620566\\
0.18	2.66313030872511\\
0.27	2.61858642789482\\
0.284221196268088	2.61\\
-1	13\\
0.246174972226958	7.74\\
0.18	7.73598758958553\\
0.0900000000000004	7.73231743340335\\
0	7.73071705994096\\
-0.0899999999999992	7.7322435886297\\
-0.180000000000001	7.73597096253988\\
-0.246174977821835	7.74\\
-0.180000000000001	7.77400671464466\\
-0.0899999999999992	7.79929970343779\\
0	7.80686394974038\\
0.0900000000000004	7.79929970172817\\
0.18	7.77400671555604\\
0.246174972226958	7.74\\
-1	13\\
0.254993060582121	8.64\\
0.18	8.63630623179355\\
0.0900000000000004	8.6331291491903\\
0	8.63181979613297\\
-0.0899999999999992	8.6330932999445\\
-0.180000000000001	8.63629434575866\\
-0.254993095861339	8.64\\
-0.180000000000001	8.64351857930341\\
-0.0899999999999992	8.64664872241612\\
0	8.64784662815591\\
0.0900000000000004	8.64666238143251\\
0.18	8.64357376011284\\
0.254993060582121	8.64\\
0	141\\
3.07426555714403	-8.64\\
3.06	-8.6406260131435\\
2.97	-8.64616748467725\\
2.88	-8.64470940883\\
2.79	-8.65051082491568\\
2.7	-8.65320884150843\\
2.61	-8.65909473619361\\
2.52	-8.66221117184926\\
2.43	-8.66738148230337\\
2.34	-8.66828518816054\\
2.25	-8.67216336459559\\
2.16	-8.67042154774278\\
2.07	-8.67616404131609\\
1.98	-8.67979106752923\\
1.89	-8.68236385803139\\
1.8	-8.68262524094216\\
1.71	-8.68414141950193\\
1.62	-8.68749231078421\\
1.53	-8.6892591079503\\
1.44	-8.6905021541847\\
1.35	-8.69215876116337\\
1.26	-8.69394985998853\\
1.17	-8.69637848872214\\
1.08	-8.69652545691872\\
0.99	-8.69904787873519\\
0.900000000000001	-8.7008381676931\\
0.809999999999999	-8.70062962245798\\
0.72	-8.7026018977698\\
0.630000000000001	-8.70414612299936\\
0.54	-8.70470942397028\\
0.45	-8.70732299383658\\
0.36	-8.7066516458758\\
0.27	-8.70904859208239\\
0.18	-8.71031315112965\\
0.0900000000000004	-8.71093583602135\\
0	-8.7114961012132\\
-0.0899999999999992	-8.71079039911935\\
-0.180000000000001	-8.7092274907469\\
-0.27	-8.70955498342034\\
-0.359999999999999	-8.70788384908267\\
-0.450000000000001	-8.70681407819075\\
-0.54	-8.70578454813465\\
-0.629999999999999	-8.70485059127063\\
-0.720000000000001	-8.70322759014744\\
-0.81	-8.70205154757806\\
-0.899999999999999	-8.70031671502652\\
-0.990000000000001	-8.69902946079589\\
-1.08	-8.69729970818427\\
-1.17	-8.69606955366394\\
-1.26	-8.69408034125561\\
-1.35	-8.69216350582445\\
-1.44	-8.69109896783158\\
-1.53	-8.68978314680358\\
-1.62	-8.68751480111048\\
-1.71	-8.68628337481382\\
-1.8	-8.68452251694527\\
-1.89	-8.68140086912523\\
-1.98	-8.67932768345981\\
-2.07	-8.67822233549866\\
-2.16	-8.67345285788212\\
-2.25	-8.67106803274912\\
-2.34	-8.6686569725382\\
-2.43	-8.66649168260724\\
-2.52	-8.66409498515088\\
-2.61	-8.6581875693909\\
-2.7	-8.65833936005057\\
-2.79	-8.6528024924938\\
-2.88	-8.64971563198658\\
-2.97	-8.64655349686542\\
-3.06	-8.64300290762108\\
-3.12680906562695	-8.64\\
-3.06	-8.63634635274039\\
-2.97	-8.6318303013204\\
-2.88	-8.62834277648234\\
-2.79	-8.62385692768109\\
-2.7	-8.61828506302034\\
-2.61	-8.61797613032572\\
-2.52	-8.61165624978025\\
-2.43	-8.60980268858652\\
-2.34	-8.60673734141756\\
-2.25	-8.60489289181605\\
-2.16	-8.60239257175855\\
-2.07	-8.59777957492141\\
-1.98	-8.59540136193555\\
-1.89	-8.59264060021096\\
-1.8	-8.59075145956048\\
-1.71	-8.58956614583673\\
-1.62	-8.58700416181801\\
-1.53	-8.58593020594005\\
-1.44	-8.58407267068153\\
-1.35	-8.58324654957424\\
-1.26	-8.58087854495096\\
-1.17	-8.58081985374467\\
-1.08	-8.57809235069011\\
-0.990000000000001	-8.57651525348836\\
-0.899999999999999	-8.57745312412157\\
-0.81	-8.57448439365092\\
-0.720000000000001	-8.57440707295866\\
-0.629999999999999	-8.5716130527251\\
-0.54	-8.57067086540781\\
-0.450000000000001	-8.57001071090212\\
-0.359999999999999	-8.56991327451088\\
-0.27	-8.56746981481064\\
-0.180000000000001	-8.56709183946633\\
-0.0899999999999992	-8.56646285663241\\
0	-8.56546445785367\\
0.0900000000000004	-8.56684454480707\\
0.18	-8.56732570907492\\
0.27	-8.56858636702875\\
0.36	-8.56824388587211\\
0.45	-8.56970604455632\\
0.54	-8.57127641184236\\
0.630000000000001	-8.57197141805182\\
0.72	-8.57353314550217\\
0.809999999999999	-8.57408325588864\\
0.900000000000001	-8.57538928914114\\
0.99	-8.57798653134293\\
1.08	-8.58026606739753\\
1.17	-8.57960707839395\\
1.26	-8.58082722579396\\
1.35	-8.58223371122879\\
1.44	-8.58512178455029\\
1.53	-8.5852805324477\\
1.62	-8.58741730159524\\
1.71	-8.58895575926021\\
1.8	-8.59077792235494\\
1.89	-8.59452566383855\\
1.98	-8.59435387943036\\
2.07	-8.5989016114701\\
2.16	-8.60482277443981\\
2.25	-8.60284209705292\\
2.34	-8.60828498779337\\
2.43	-8.60865966167238\\
2.52	-8.61362106254729\\
2.61	-8.617070210487\\
2.7	-8.6235571431827\\
2.79	-8.62556626501119\\
2.88	-8.63421743334994\\
2.97	-8.63216682782249\\
3.06	-8.63923450639513\\
3.07426555714403	-8.64\\
0	147\\
3.17936558407341	-8.37\\
3.15	-8.37060626326869\\
3.06	-8.37835345487279\\
2.97	-8.38547957160116\\
2.88	-8.38924202045925\\
2.79	-8.39251716806728\\
2.7	-8.39494122607348\\
2.61	-8.3968830498554\\
2.52	-8.401916794397\\
2.43	-8.40449430793453\\
2.34	-8.40664996971475\\
2.25	-8.41027855943992\\
2.16	-8.41120268926407\\
2.07	-8.41439478616071\\
1.98	-8.41728353734729\\
1.89	-8.41935335866159\\
1.8	-8.42074222525608\\
1.71	-8.42243428117095\\
1.62	-8.42389305799308\\
1.53	-8.425076724132\\
1.44	-8.42623646606817\\
1.35	-8.42813353913956\\
1.26	-8.43002375350674\\
1.17	-8.43007689248962\\
1.08	-8.43155393497168\\
0.99	-8.43251820473151\\
0.900000000000001	-8.43406755205663\\
0.809999999999999	-8.43414220556827\\
0.72	-8.43556907459595\\
0.630000000000001	-8.43496517671995\\
0.54	-8.43656393698453\\
0.45	-8.43807293198473\\
0.36	-8.43821367882511\\
0.27	-8.43869622624282\\
0.18	-8.43949391852646\\
0.0900000000000004	-8.43960187864003\\
0	-8.43986449084486\\
-0.0899999999999992	-8.43919387919353\\
-0.180000000000001	-8.43908264154186\\
-0.27	-8.43843234366928\\
-0.359999999999999	-8.43775477640022\\
-0.450000000000001	-8.43824329426632\\
-0.54	-8.43587168731161\\
-0.629999999999999	-8.43683101604354\\
-0.720000000000001	-8.4352251902518\\
-0.81	-8.43505777564816\\
-0.899999999999999	-8.43313554904947\\
-0.990000000000001	-8.43247065998328\\
-1.08	-8.43196521122143\\
-1.17	-8.4304230973013\\
-1.26	-8.42941332845235\\
-1.35	-8.42819197254833\\
-1.44	-8.42729619856515\\
-1.53	-8.42630638701143\\
-1.62	-8.42362514019882\\
-1.71	-8.42190174209208\\
-1.8	-8.42092867696163\\
-1.89	-8.41955780166694\\
-1.98	-8.41828250307768\\
-2.07	-8.41370122407239\\
-2.16	-8.4122044676013\\
-2.25	-8.40823642418855\\
-2.34	-8.40731999026723\\
-2.43	-8.40356477319405\\
-2.52	-8.4018443347701\\
-2.61	-8.39846781588563\\
-2.7	-8.39511112428608\\
-2.79	-8.39155762133659\\
-2.88	-8.38892653819933\\
-2.97	-8.38545980408351\\
-3.06	-8.37927036043157\\
-3.15	-8.37547178746147\\
-3.24	-8.37054475033046\\
-3.24801569439853	-8.37\\
-3.24	-8.36947711889683\\
-3.15	-8.364697502627\\
-3.06	-8.36086847686457\\
-2.97	-8.35475377946891\\
-2.88	-8.35060395494328\\
-2.79	-8.3489335066317\\
-2.7	-8.34613431334261\\
-2.61	-8.34086903658632\\
-2.52	-8.3376091883592\\
-2.43	-8.33556799018105\\
-2.34	-8.33208965252036\\
-2.25	-8.33045790537037\\
-2.16	-8.32732688620566\\
-2.07	-8.32635633829549\\
-1.98	-8.32206013366964\\
-1.89	-8.32048462465376\\
-1.8	-8.3191685815576\\
-1.71	-8.31636968124536\\
-1.62	-8.31455603947862\\
-1.53	-8.31336528116271\\
-1.44	-8.3123626974732\\
-1.35	-8.31031666582778\\
-1.26	-8.30936683767829\\
-1.17	-8.30832133263437\\
-1.08	-8.30709696781149\\
-0.990000000000001	-8.30585598794645\\
-0.899999999999999	-8.30550695412294\\
-0.81	-8.30381622629396\\
-0.720000000000001	-8.30374441901838\\
-0.629999999999999	-8.30225298574171\\
-0.54	-8.30169034243728\\
-0.450000000000001	-8.3006738983193\\
-0.359999999999999	-8.30073466477433\\
-0.27	-8.30023611253485\\
-0.180000000000001	-8.30025431745743\\
-0.0899999999999992	-8.29929058456211\\
0	-8.29903084076386\\
0.0900000000000004	-8.29982631369221\\
0.18	-8.29973535518548\\
0.27	-8.29977904481463\\
0.36	-8.30007302912771\\
0.45	-8.30033254485871\\
0.54	-8.30231017943654\\
0.630000000000001	-8.30262971976052\\
0.72	-8.30350822163628\\
0.809999999999999	-8.30415093808613\\
0.900000000000001	-8.3049461066706\\
0.99	-8.30611937566176\\
1.08	-8.30649898125688\\
1.17	-8.3084093060827\\
1.26	-8.30947409758745\\
1.35	-8.31056770050826\\
1.44	-8.31178191704415\\
1.53	-8.31386938029272\\
1.62	-8.31475487888544\\
1.71	-8.31632668141613\\
1.8	-8.31802613497885\\
1.89	-8.32081838079408\\
1.98	-8.32260571476926\\
2.07	-8.32547742195443\\
2.16	-8.32756875964508\\
2.25	-8.3299254668318\\
2.34	-8.33370644811692\\
2.43	-8.33614618827819\\
2.52	-8.33843431163749\\
2.61	-8.34311909435637\\
2.7	-8.34469082774623\\
2.79	-8.34764277380629\\
2.88	-8.35048063639982\\
2.97	-8.35438179450588\\
3.06	-8.36157192307823\\
3.15	-8.36940826613322\\
3.17936558407341	-8.37\\
0	143\\
3.06181947796548	-7.83\\
3.06	-7.83032915232239\\
2.97	-7.8365213740711\\
2.88	-7.84139170148282\\
2.79	-7.84611831233404\\
2.7	-7.85054822751362\\
2.61	-7.85492004071353\\
2.52	-7.85898530294209\\
2.43	-7.86418470269247\\
2.34	-7.86706499546692\\
2.25	-7.87057273297642\\
2.16	-7.87346436369078\\
2.07	-7.87749033380783\\
1.98	-7.88070206741322\\
1.89	-7.88155702625476\\
1.8	-7.8824143066868\\
1.71	-7.88431563849968\\
1.62	-7.88500563394791\\
1.53	-7.88713711589768\\
1.44	-7.89271115924535\\
1.35	-7.89579689159161\\
1.26	-7.89465914533371\\
1.17	-7.89771911276927\\
1.08	-7.89741242809846\\
0.99	-7.89981678050385\\
0.900000000000001	-7.9001110044559\\
0.809999999999999	-7.90367522421297\\
0.72	-7.90474987350304\\
0.630000000000001	-7.9021175131347\\
0.54	-7.9064705836388\\
0.45	-7.90770023209256\\
0.36	-7.90564935660714\\
0.27	-7.90737104185525\\
0.18	-7.90977281874953\\
0.0900000000000004	-7.90765010090296\\
0	-7.90914788282218\\
-0.0899999999999992	-7.90805470774096\\
-0.180000000000001	-7.90896870931194\\
-0.27	-7.9077726346857\\
-0.359999999999999	-7.90853496833152\\
-0.450000000000001	-7.90738165631529\\
-0.54	-7.90601255876149\\
-0.629999999999999	-7.90637040512995\\
-0.720000000000001	-7.90553495610695\\
-0.81	-7.90207630393947\\
-0.899999999999999	-7.9029280488611\\
-0.990000000000001	-7.89680174210107\\
-1.08	-7.89907485748059\\
-1.17	-7.89823856063538\\
-1.26	-7.8952933764434\\
-1.35	-7.89362418818851\\
-1.44	-7.89080963819095\\
-1.53	-7.89119818873131\\
-1.62	-7.89025937221837\\
-1.71	-7.8862438332113\\
-1.8	-7.88456959053903\\
-1.89	-7.88021498085567\\
-1.98	-7.8776260428049\\
-2.07	-7.87546867206209\\
-2.16	-7.87272666750543\\
-2.25	-7.86944726515862\\
-2.34	-7.86581532108156\\
-2.43	-7.86271223290541\\
-2.52	-7.85866733782398\\
-2.61	-7.85491912916901\\
-2.7	-7.8513798795262\\
-2.79	-7.846014373291\\
-2.88	-7.83749992657547\\
-2.97	-7.83626749453061\\
-3.06	-7.83069969225748\\
-3.06983393833575	-7.83\\
-3.07237271187583	-7.74\\
-3.06	-7.7381553230578\\
-2.97	-7.73248170116962\\
-2.88	-7.72500579189372\\
-2.79	-7.71852041548379\\
-2.7	-7.7138267989377\\
-2.61	-7.7100048002551\\
-2.52	-7.7026794103854\\
-2.43	-7.7000474460873\\
-2.34	-7.69622079533598\\
-2.25	-7.69342002852294\\
-2.16	-7.68910931877904\\
-2.07	-7.68607473056825\\
-1.98	-7.67948380545293\\
-1.89	-7.67668843126116\\
-1.8	-7.68044104903232\\
-1.71	-7.67883949774702\\
-1.62	-7.67655116904485\\
-1.53	-7.66584030655906\\
-1.44	-7.66476527922408\\
-1.35	-7.66398636431453\\
-1.26	-7.66651698269535\\
-1.17	-7.66575188135054\\
-1.08	-7.66032427495602\\
-0.990000000000001	-7.66067989129615\\
-0.899999999999999	-7.66006505355259\\
-0.81	-7.65690992492465\\
-0.720000000000001	-7.65673971715211\\
-0.629999999999999	-7.65805356541476\\
-0.54	-7.65435104744985\\
-0.450000000000001	-7.65314257504395\\
-0.359999999999999	-7.65240974291008\\
-0.27	-7.6549947439978\\
-0.180000000000001	-7.65188514793186\\
-0.0899999999999992	-7.65385230373288\\
0	-7.6527029554752\\
0.0900000000000004	-7.6546724712053\\
0.18	-7.65224877798419\\
0.27	-7.65201452941749\\
0.36	-7.65248578456272\\
0.45	-7.65430611500245\\
0.54	-7.65377036879134\\
0.630000000000001	-7.65508665622699\\
0.72	-7.6545081007319\\
0.809999999999999	-7.65730201720357\\
0.900000000000001	-7.65859222841777\\
0.99	-7.66247796585307\\
1.08	-7.66002481586583\\
1.17	-7.66431286035719\\
1.26	-7.66570390140716\\
1.35	-7.66308937100174\\
1.44	-7.6647287701464\\
1.53	-7.67045360187728\\
1.62	-7.66942420422147\\
1.71	-7.67099656164868\\
1.8	-7.67529339600293\\
1.89	-7.6797703676887\\
1.98	-7.68315092422121\\
2.07	-7.68811693586284\\
2.16	-7.69175055072538\\
2.25	-7.69433857471608\\
2.34	-7.69378650347703\\
2.43	-7.7004328111605\\
2.52	-7.7053192661333\\
2.61	-7.71000236809282\\
2.7	-7.7135785127552\\
2.79	-7.71924735103614\\
2.88	-7.72686109764693\\
2.97	-7.73087503195363\\
3.06	-7.73614330098167\\
3.10761236136571	-7.74\\
3.06181947796548	-7.83\\
0	151\\
-0.969628182362424	-6.75\\
-0.990000000000001	-6.75132077534785\\
-1.08	-6.75114479800207\\
-1.09970907139838	-6.75\\
-1.17	-6.74767043042664\\
-1.26	-6.74966847820095\\
-1.35	-6.74910666344647\\
-1.44	-6.74770166473828\\
-1.53	-6.74400553311555\\
-1.62	-6.74545521575717\\
-1.71	-6.74112736234527\\
-1.8	-6.73873851754102\\
-1.89	-6.7361042937011\\
-1.98	-6.73497053205094\\
-2.07	-6.7297096976098\\
-2.16	-6.73003539591027\\
-2.25	-6.72860451504654\\
-2.34	-6.72342477115898\\
-2.43	-6.71796775871983\\
-2.52	-6.71378546494174\\
-2.61	-6.70889491554215\\
-2.7	-6.70525022322563\\
-2.79	-6.69404290345667\\
-2.88	-6.68846698230139\\
-2.97	-6.67960380343245\\
-3.06	-6.67406303593756\\
-3.14067695387728	-6.66\\
-3.06	-6.59599459097807\\
-3.00984894536183	-6.57\\
-2.97	-6.56566074144652\\
-2.88	-6.5572362106993\\
-2.79	-6.5456265138729\\
-2.7	-6.54451068412502\\
-2.61	-6.53652562371175\\
-2.52	-6.52988983381975\\
-2.43	-6.52598607260621\\
-2.34	-6.52142447404241\\
-2.25	-6.51966537056851\\
-2.16	-6.51159126011191\\
-2.07	-6.51350805943663\\
-1.98	-6.50862185045394\\
-1.89	-6.50657787137353\\
-1.8	-6.50332588448038\\
-1.71	-6.50033837508693\\
-1.62	-6.49936629973282\\
-1.53	-6.49783197217441\\
-1.44	-6.49549215675524\\
-1.35	-6.4944606900439\\
-1.26	-6.49203973811113\\
-1.17	-6.49323307357878\\
-1.08	-6.4909169229275\\
-0.990000000000001	-6.48942259574313\\
-0.899999999999999	-6.4892823736017\\
-0.81	-6.48920318696608\\
-0.720000000000001	-6.49017844132221\\
-0.629999999999999	-6.48697849621542\\
-0.54	-6.48586301663887\\
-0.450000000000001	-6.48480253911377\\
-0.359999999999999	-6.48704844071128\\
-0.27	-6.48645880996246\\
-0.180000000000001	-6.48742765618129\\
-0.0899999999999992	-6.48776239144164\\
0	-6.48599388441398\\
0.0900000000000004	-6.48526901447726\\
0.18	-6.48757781124774\\
0.27	-6.48527714450328\\
0.36	-6.48675878008218\\
0.45	-6.48670812625664\\
0.54	-6.48631310840591\\
0.630000000000001	-6.48921892722879\\
0.72	-6.4880055262662\\
0.809999999999999	-6.48813486066555\\
0.900000000000001	-6.48857028697329\\
0.99	-6.49199102541682\\
1.08	-6.49265809481939\\
1.17	-6.49278091498604\\
1.26	-6.49077727855318\\
1.35	-6.49417877887659\\
1.44	-6.49426317102344\\
1.53	-6.49542859550301\\
1.62	-6.49894438706652\\
1.71	-6.49973738563948\\
1.8	-6.50262898980124\\
1.89	-6.50547928279707\\
1.98	-6.50744154823508\\
2.07	-6.5083023368987\\
2.16	-6.51623464995456\\
2.25	-6.51671865632384\\
2.34	-6.52256773792777\\
2.43	-6.52568624196067\\
2.52	-6.52722736011009\\
2.61	-6.53653981586522\\
2.7	-6.54066051482148\\
2.79	-6.54815468447823\\
2.88	-6.56073009328191\\
2.97	-6.56678757345824\\
2.99637748778528	-6.57\\
3.06	-6.60819455445823\\
3.15	-6.63727400723275\\
3.18139419504412	-6.66\\
3.15	-6.6638681803136\\
3.06	-6.67098043224635\\
2.97	-6.68085980171754\\
2.88	-6.68865412578163\\
2.79	-6.69822000487052\\
2.7	-6.70512948221623\\
2.61	-6.70569549782222\\
2.52	-6.7125932664355\\
2.43	-6.7210337671129\\
2.34	-6.72037861942355\\
2.25	-6.72622125508137\\
2.16	-6.73159994276669\\
2.07	-6.73314914046059\\
1.98	-6.73340454789496\\
1.89	-6.73863829891307\\
1.8	-6.73960087870526\\
1.71	-6.74312725109833\\
1.62	-6.74467075484596\\
1.53	-6.74617731241284\\
1.44	-6.74694772034084\\
1.35	-6.74653261173099\\
1.26	-6.74937866996957\\
1.17	-6.74946626760866\\
1.08	-6.74890388106488\\
1.03544980675713	-6.75\\
0.99	-6.75209364849886\\
0.936327460355385	-6.75\\
0.900000000000001	-6.74930269248451\\
0.875484071681454	-6.75\\
0.809999999999999	-6.75376475981498\\
0.72	-6.75384975453879\\
0.630000000000001	-6.75285171872554\\
0.54	-6.75495766332954\\
0.45	-6.75446472721803\\
0.36	-6.75485439830799\\
0.27	-6.75511499023566\\
0.18	-6.75785831271307\\
0.0900000000000004	-6.75658413562158\\
0	-6.75421003246147\\
-0.0899999999999992	-6.7563636854818\\
-0.180000000000001	-6.75271782753502\\
-0.27	-6.7534866863677\\
-0.359999999999999	-6.75620984443674\\
-0.450000000000001	-6.75267949027315\\
-0.54	-6.75587900737588\\
-0.629999999999999	-6.75361840406343\\
-0.720000000000001	-6.75267817244956\\
-0.81	-6.75326850569504\\
-0.847627997977426	-6.75\\
-0.899999999999999	-6.74781270147626\\
-0.969628182362424	-6.75\\
0	353\\
-0.630553519731764	-3.24\\
-0.720000000000001	-3.26475971824591\\
-0.81	-3.25818312914786\\
-0.899999999999999	-3.24742579804765\\
-0.909701293945632	-3.24\\
-0.990000000000001	-3.20695677866136\\
-1.08	-3.21943961807365\\
-1.17	-3.20899077396365\\
-1.26	-3.2160421852474\\
-1.35	-3.20337284250035\\
-1.40236483082271	-3.15\\
-1.44	-3.12851149356073\\
-1.52289042301889	-3.15\\
-1.53	-3.15412172153615\\
-1.62	-3.15094234915416\\
-1.62043116006883	-3.15\\
-1.71	-3.08648201289154\\
-1.8	-3.06351887880381\\
-1.89	-3.09716973061651\\
-1.98	-3.06833243202193\\
-1.9918374439857	-3.06\\
-2.07	-3.0101715603512\\
-2.13482988373265	-2.97\\
-2.16	-2.9475582402111\\
-2.25	-2.9010020969805\\
-2.28193679363947	-2.88\\
-2.34	-2.81779141684768\\
-2.37900044755647	-2.79\\
-2.34	-2.7612784544995\\
-2.2896608124704	-2.7\\
-2.34	-2.61864685205394\\
-2.43	-2.63021278048657\\
-2.4609890275893	-2.61\\
-2.49980197370715	-2.52\\
-2.52	-2.50323713111049\\
-2.60098379047363	-2.43\\
-2.61	-2.42467256641354\\
-2.69185050949112	-2.34\\
-2.7	-2.33242152654699\\
-2.79	-2.32434508148028\\
-2.82420315802709	-2.25\\
-2.88	-2.19476480096999\\
-2.94253545592251	-2.16\\
-2.97	-2.10428101244093\\
-2.98852668192373	-2.07\\
-3.03587012117953	-1.98\\
-3.05896892817245	-1.89\\
-3.03062398228217	-1.8\\
-3.06	-1.75056204997608\\
-3.08846566359934	-1.71\\
-3.07683632490535	-1.62\\
-3.06799318224535	-1.53\\
-3.06	-1.44647145882011\\
-3.05773864957453	-1.44\\
-3.0514415339146	-1.35\\
-3.01587061625653	-1.26\\
-2.97	-1.18237486410288\\
-2.96225643867913	-1.17\\
-2.97	-1.12327175624517\\
-2.97914701092346	-1.08\\
-2.97	-1.05091611598581\\
-2.95017717292923	-0.990000000000001\\
-2.94430011349423	-0.899999999999999\\
-2.93935222890956	-0.81\\
-2.96091766877415	-0.72\\
-2.95988398478355	-0.630000000000001\\
-2.97	-0.609983629349726\\
-2.99749389982099	-0.54\\
-3.03402296705244	-0.45\\
-2.98695725637042	-0.360000000000001\\
-3.06	-0.288635563785598\\
-3.07382246770398	-0.27\\
-3.06	-0.261038352883771\\
-3.0038385574789	-0.180000000000001\\
-3.06	-0.0995706416953951\\
-3.07757417116839	-0.0899999999999989\\
-3.06756596155111	0\\
-3.07757417116839	0.0900000000000005\\
-3.06	0.0995706416953964\\
-3.0038385574789	0.18\\
-3.06	0.261038352883771\\
-3.07382246770398	0.27\\
-3.06	0.288635563785598\\
-2.98695725637042	0.359999999999999\\
-3.03402296705244	0.45\\
-2.99749389982099	0.54\\
-2.97	0.609983629349726\\
-2.95988398478355	0.630000000000001\\
-2.96091766877415	0.720000000000001\\
-2.93935222890956	0.81\\
-2.94430011349423	0.900000000000001\\
-2.95017717292923	0.99\\
-2.97	1.05091611598581\\
-2.97914701092346	1.08\\
-2.97	1.12327175624517\\
-2.96225643867913	1.17\\
-2.97	1.18237486410288\\
-3.01587061625653	1.26\\
-3.0514415339146	1.35\\
-3.05773864957453	1.44\\
-3.06	1.44647145882011\\
-3.06799318224535	1.53\\
-3.07683632490535	1.62\\
-3.08846566359934	1.71\\
-3.06	1.75056204997608\\
-3.03062398228217	1.8\\
-3.05896892817245	1.89\\
-3.03587012117953	1.98\\
-2.98852668192373	2.07\\
-2.97	2.10428101244093\\
-2.94253545592251	2.16\\
-2.88	2.19476480096999\\
-2.82420315802709	2.25\\
-2.79	2.32434508148028\\
-2.7	2.33242152654699\\
-2.69185050949112	2.34\\
-2.61	2.42467256641354\\
-2.60098379047363	2.43\\
-2.52	2.50323713111049\\
-2.49980197370715	2.52\\
-2.4609890275893	2.61\\
-2.43	2.63021278048657\\
-2.34	2.61864685205394\\
-2.2896608124704	2.7\\
-2.34	2.7612784544995\\
-2.37900044755647	2.79\\
-2.34	2.81779141684768\\
-2.28193679363947	2.88\\
-2.25	2.9010020969805\\
-2.16	2.9475582402111\\
-2.13482988373265	2.97\\
-2.07	3.0101715603512\\
-1.9918374439857	3.06\\
-1.98	3.06833243202193\\
-1.89	3.09716973061651\\
-1.8	3.06351887880381\\
-1.71	3.08648201289154\\
-1.62043116006883	3.15\\
-1.62	3.15094234915416\\
-1.53	3.15412172153615\\
-1.52289042301889	3.15\\
-1.44	3.12851149356073\\
-1.40236483082271	3.15\\
-1.35	3.20337284250035\\
-1.26	3.2160421852474\\
-1.17	3.20899077396365\\
-1.08	3.21943961807365\\
-0.990000000000001	3.20695677866136\\
-0.909701293945632	3.24\\
-0.899999999999999	3.24742579804765\\
-0.81	3.25818312914786\\
-0.720000000000001	3.26475971824591\\
-0.630553519731764	3.24\\
-0.629999999999999	3.23988812047847\\
-0.629594955155129	3.24\\
-0.54	3.27431239634367\\
-0.450000000000001	3.27117439270268\\
-0.359999999999999	3.27666446247255\\
-0.27	3.28080893966865\\
-0.180000000000001	3.28108404887818\\
-0.0899999999999992	3.28270044037788\\
0	3.27790120251886\\
0.0900000000000004	3.27936470087747\\
0.18	3.28462272294669\\
0.27	3.27786943925636\\
0.315288568434234	3.24\\
0.36	3.21883491565908\\
0.410832359337469	3.24\\
0.45	3.26436044335535\\
0.54	3.27189414681374\\
0.630000000000001	3.26717970512898\\
0.72	3.26679744260751\\
0.809999999999999	3.25665495339974\\
0.900000000000001	3.25223301050472\\
0.99	3.24564934185985\\
1.03374863953752	3.24\\
1.08	3.23399415248544\\
1.17	3.22483649331578\\
1.26	3.21665708679597\\
1.35	3.19986739129563\\
1.44	3.19100534225411\\
1.53	3.17006879771022\\
1.62	3.15632810622507\\
1.6707957573804	3.15\\
1.71	3.14207849005674\\
1.8	3.11275100815045\\
1.89	3.09543525948735\\
1.97833578740956	3.06\\
1.93219522382384	2.97\\
1.98	2.93398375876556\\
2.03703215334569	2.97\\
2.07	3.01994918014958\\
2.16	2.97987688162579\\
2.18059528033872	2.97\\
2.21398315965695	2.88\\
2.16	2.83459979490516\\
2.14416222645701	2.79\\
2.16	2.76642346876975\\
2.25	2.75034873008209\\
2.32236859536407	2.79\\
2.32253925549893	2.88\\
2.34	2.88654055303103\\
2.34708999087627	2.88\\
2.35270395632586	2.79\\
2.35659882858401	2.7\\
2.43	2.62127316029835\\
2.44688651399102	2.61\\
2.49130705191295	2.52\\
2.52	2.49782224367398\\
2.60670749923597	2.43\\
2.61	2.42810511266739\\
2.7	2.38323698604922\\
2.73186551262767	2.34\\
2.79	2.30169063222775\\
2.86840810046018	2.25\\
2.88	2.2342523041922\\
2.93291907047248	2.16\\
2.97	2.10987535878578\\
2.99576998744292	2.07\\
3.02558654501931	1.98\\
3.06	1.89622160375953\\
3.06229668644566	1.89\\
3.06913321779232	1.8\\
3.06	1.74696368555415\\
3.05353054181048	1.71\\
3.06	1.68531527924993\\
3.07543603335611	1.62\\
3.09739627347941	1.53\\
3.07267519302994	1.44\\
3.06	1.42503416604626\\
3.01150892922305	1.35\\
3.00481007287359	1.26\\
3.00444563747939	1.17\\
2.97	1.12529007420545\\
2.94731079281237	1.08\\
2.97	1.0026602455389\\
2.9748472668243	0.99\\
2.97	0.975328532640137\\
2.94116998632866	0.900000000000001\\
2.92878810755812	0.81\\
2.94407852888419	0.720000000000001\\
2.97	0.670834327455804\\
3.0006639484119	0.630000000000001\\
3.00082578253817	0.54\\
2.99758777823218	0.45\\
3.05183074193012	0.359999999999999\\
3.06	0.330946007185629\\
3.06629273636889	0.27\\
3.06	0.236520900981828\\
3.02710588303873	0.18\\
3.06	0.144766668707073\\
3.0802613329328	0.0900000000000005\\
3.07710233602148	0\\
3.0802613329328	-0.0899999999999989\\
3.06	-0.144766668707074\\
3.02710588303873	-0.180000000000001\\
3.06	-0.236520900981828\\
3.06629273636889	-0.27\\
3.06	-0.33094600718563\\
3.05183074193012	-0.360000000000001\\
2.99758777823218	-0.45\\
3.00082578253817	-0.54\\
3.0006639484119	-0.630000000000001\\
2.97	-0.670834327455803\\
2.94407852888419	-0.72\\
2.92878810755812	-0.81\\
2.94116998632866	-0.899999999999999\\
2.97	-0.975328532640138\\
2.9748472668243	-0.990000000000001\\
2.97	-1.0026602455389\\
2.94731079281237	-1.08\\
2.97	-1.12529007420545\\
3.00444563747939	-1.17\\
3.00481007287359	-1.26\\
3.01150892922305	-1.35\\
3.06	-1.42503416604626\\
3.07267519302994	-1.44\\
3.09739627347941	-1.53\\
3.07543603335611	-1.62\\
3.06	-1.68531527924993\\
3.05353054181048	-1.71\\
3.06	-1.74696368555415\\
3.06913321779232	-1.8\\
3.06229668644566	-1.89\\
3.06	-1.89622160375953\\
3.02558654501931	-1.98\\
2.99576998744292	-2.07\\
2.97	-2.10987535878578\\
2.93291907047248	-2.16\\
2.88	-2.2342523041922\\
2.86840810046018	-2.25\\
2.79	-2.30169063222775\\
2.73186551262767	-2.34\\
2.7	-2.38323698604922\\
2.61	-2.42810511266739\\
2.60670749923597	-2.43\\
2.52	-2.49782224367398\\
2.49130705191295	-2.52\\
2.44688651399102	-2.61\\
2.43	-2.62127316029835\\
2.35659882858401	-2.7\\
2.35270395632586	-2.79\\
2.34708999087627	-2.88\\
2.34	-2.88654055303103\\
2.32253925549893	-2.88\\
2.32236859536407	-2.79\\
2.25	-2.75034873008209\\
2.16	-2.76642346876975\\
2.14416222645701	-2.79\\
2.16	-2.83459979490516\\
2.21398315965695	-2.88\\
2.18059528033872	-2.97\\
2.16	-2.97987688162579\\
2.07	-3.01994918014958\\
2.03703215334569	-2.97\\
1.98	-2.93398375876556\\
1.93219522382384	-2.97\\
1.97833578740956	-3.06\\
1.89	-3.09543525948735\\
1.8	-3.11275100815045\\
1.71	-3.14207849005674\\
1.6707957573804	-3.15\\
1.62	-3.15632810622507\\
1.53	-3.17006879771022\\
1.44	-3.19100534225411\\
1.35	-3.19986739129563\\
1.26	-3.21665708679597\\
1.17	-3.22483649331578\\
1.08	-3.23399415248544\\
1.03374863953752	-3.24\\
0.99	-3.24564934185985\\
0.900000000000001	-3.25223301050472\\
0.809999999999999	-3.25665495339974\\
0.72	-3.26679744260751\\
0.630000000000001	-3.26717970512898\\
0.54	-3.27189414681374\\
0.45	-3.26436044335535\\
0.410832359337469	-3.24\\
0.36	-3.21883491565908\\
0.315288568434234	-3.24\\
0.27	-3.27786943925636\\
0.18	-3.28462272294669\\
0.0900000000000004	-3.27936470087747\\
0	-3.27790120251886\\
-0.0899999999999992	-3.28270044037788\\
-0.180000000000001	-3.28108404887818\\
-0.27	-3.28080893966865\\
-0.359999999999999	-3.27666446247256\\
-0.450000000000001	-3.27117439270268\\
-0.54	-3.27431239634367\\
-0.629594955155129	-3.24\\
-0.629999999999999	-3.23988812047847\\
-0.630553519731764	-3.24\\
0	5\\
-2.17813455081138	-2.7\\
-2.16	-2.72261676738498\\
-2.15348166486859	-2.7\\
-2.16	-2.69117861900401\\
-2.17813455081138	-2.7\\
0	5\\
-2.17813455081138	2.7\\
-2.16	2.69117861900401\\
-2.15348166486859	2.7\\
-2.16	2.72261676738497\\
-2.17813455081138	2.7\\
0	151\\
2.99637748778528	6.57\\
2.97	6.56678757345824\\
2.88	6.56073009328191\\
2.79	6.54815468447823\\
2.7	6.54066051482148\\
2.61	6.53653981586522\\
2.52	6.52722736011009\\
2.43	6.52568624196067\\
2.34	6.52256773792777\\
2.25	6.51671865632384\\
2.16	6.51623464995456\\
2.07	6.5083023368987\\
1.98	6.50744154823508\\
1.89	6.50547928279707\\
1.8	6.50262898980124\\
1.71	6.49973738563948\\
1.62	6.49894438706652\\
1.53	6.49542859550301\\
1.44	6.49426317102344\\
1.35	6.49417877887659\\
1.26	6.49077727855318\\
1.17	6.49278091498604\\
1.08	6.49265809481939\\
0.99	6.49199102541682\\
0.900000000000001	6.48857028697329\\
0.809999999999999	6.48813486066555\\
0.72	6.4880055262662\\
0.630000000000001	6.48921892722879\\
0.54	6.48631310840591\\
0.45	6.48670812625664\\
0.36	6.48675878008218\\
0.27	6.48527714450328\\
0.18	6.48757781124774\\
0.0900000000000004	6.48526901447726\\
0	6.48599388441398\\
-0.0899999999999992	6.48776239144164\\
-0.180000000000001	6.48742765618129\\
-0.27	6.48645880996246\\
-0.359999999999999	6.48704844071128\\
-0.450000000000001	6.48480253911377\\
-0.54	6.48586301663887\\
-0.629999999999999	6.48697849621542\\
-0.720000000000001	6.49017844132221\\
-0.81	6.48920318696608\\
-0.899999999999999	6.4892823736017\\
-0.990000000000001	6.48942259574313\\
-1.08	6.4909169229275\\
-1.17	6.49323307357878\\
-1.26	6.49203973811113\\
-1.35	6.4944606900439\\
-1.44	6.49549215675524\\
-1.53	6.49783197217441\\
-1.62	6.49936629973282\\
-1.71	6.50033837508693\\
-1.8	6.50332588448038\\
-1.89	6.50657787137353\\
-1.98	6.50862185045394\\
-2.07	6.51350805943663\\
-2.16	6.51159126011191\\
-2.25	6.51966537056851\\
-2.34	6.52142447404241\\
-2.43	6.52598607260621\\
-2.52	6.52988983381975\\
-2.61	6.53652562371175\\
-2.7	6.54451068412502\\
-2.79	6.5456265138729\\
-2.88	6.5572362106993\\
-2.97	6.56566074144652\\
-3.00984894536183	6.57\\
-3.06	6.59599459097807\\
-3.14067695387728	6.66\\
-3.06	6.67406303593755\\
-2.97	6.67960380343245\\
-2.88	6.68846698230138\\
-2.79	6.69404290345667\\
-2.7	6.70525022322563\\
-2.61	6.70889491554215\\
-2.52	6.71378546494174\\
-2.43	6.71796775871983\\
-2.34	6.72342477115898\\
-2.25	6.72860451504654\\
-2.16	6.73003539591026\\
-2.07	6.7297096976098\\
-1.98	6.73497053205094\\
-1.89	6.7361042937011\\
-1.8	6.73873851754102\\
-1.71	6.74112736234526\\
-1.62	6.74545521575717\\
-1.53	6.74400553311555\\
-1.44	6.74770166473828\\
-1.35	6.74910666344647\\
-1.26	6.74966847820095\\
-1.17	6.74767043042664\\
-1.09970907139838	6.75\\
-1.08	6.75114479800206\\
-0.990000000000001	6.75132077534785\\
-0.969628182362424	6.75\\
-0.899999999999999	6.74781270147626\\
-0.847627997977426	6.75\\
-0.81	6.75326850569504\\
-0.720000000000001	6.75267817244956\\
-0.629999999999999	6.75361840406343\\
-0.54	6.75587900737587\\
-0.450000000000001	6.75267949027315\\
-0.359999999999999	6.75620984443674\\
-0.27	6.7534866863677\\
-0.180000000000001	6.75271782753502\\
-0.0899999999999992	6.7563636854818\\
0	6.75421003246147\\
0.0900000000000004	6.75658413562158\\
0.18	6.75785831271307\\
0.27	6.75511499023566\\
0.36	6.75485439830799\\
0.45	6.75446472721803\\
0.54	6.75495766332954\\
0.630000000000001	6.75285171872554\\
0.72	6.75384975453879\\
0.809999999999999	6.75376475981497\\
0.875484071681454	6.75\\
0.900000000000001	6.74930269248451\\
0.936327460355385	6.75\\
0.99	6.75209364849885\\
1.03544980675713	6.75\\
1.08	6.74890388106488\\
1.17	6.74946626760866\\
1.26	6.74937866996956\\
1.35	6.74653261173099\\
1.44	6.74694772034084\\
1.53	6.74617731241284\\
1.62	6.74467075484596\\
1.71	6.74312725109833\\
1.8	6.73960087870526\\
1.89	6.73863829891307\\
1.98	6.73340454789495\\
2.07	6.73314914046059\\
2.16	6.73159994276669\\
2.25	6.72622125508137\\
2.34	6.72037861942355\\
2.43	6.7210337671129\\
2.52	6.7125932664355\\
2.61	6.70569549782222\\
2.7	6.70512948221623\\
2.79	6.69822000487052\\
2.88	6.68865412578163\\
2.97	6.68085980171754\\
3.06	6.67098043224635\\
3.15	6.6638681803136\\
3.18139419504412	6.66\\
3.15	6.63727400723275\\
3.06	6.60819455445823\\
2.99637748778528	6.57\\
0	143\\
3.10761236136571	7.74\\
3.06	7.73614330098167\\
2.97	7.73087503195363\\
2.88	7.72686109764693\\
2.79	7.71924735103614\\
2.7	7.71357851275519\\
2.61	7.71000236809281\\
2.52	7.7053192661333\\
2.43	7.7004328111605\\
2.34	7.69378650347703\\
2.25	7.69433857471607\\
2.16	7.69175055072538\\
2.07	7.68811693586284\\
1.98	7.6831509242212\\
1.89	7.6797703676887\\
1.8	7.67529339600293\\
1.71	7.67099656164868\\
1.62	7.66942420422146\\
1.53	7.67045360187728\\
1.44	7.6647287701464\\
1.35	7.66308937100174\\
1.26	7.66570390140716\\
1.17	7.66431286035718\\
1.08	7.66002481586583\\
0.99	7.66247796585307\\
0.900000000000001	7.65859222841776\\
0.809999999999999	7.65730201720357\\
0.72	7.6545081007319\\
0.630000000000001	7.65508665622699\\
0.54	7.65377036879134\\
0.45	7.65430611500245\\
0.36	7.65248578456272\\
0.27	7.65201452941749\\
0.18	7.65224877798419\\
0.0900000000000004	7.6546724712053\\
0	7.6527029554752\\
-0.0899999999999992	7.65385230373288\\
-0.180000000000001	7.65188514793186\\
-0.27	7.6549947439978\\
-0.359999999999999	7.65240974291008\\
-0.450000000000001	7.65314257504395\\
-0.54	7.65435104744985\\
-0.629999999999999	7.65805356541476\\
-0.720000000000001	7.65673971715211\\
-0.81	7.65690992492465\\
-0.899999999999999	7.66006505355259\\
-0.990000000000001	7.66067989129615\\
-1.08	7.66032427495602\\
-1.17	7.66575188135054\\
-1.26	7.66651698269535\\
-1.35	7.66398636431453\\
-1.44	7.66476527922408\\
-1.53	7.66584030655906\\
-1.62	7.67655116904485\\
-1.71	7.67883949774701\\
-1.8	7.68044104903232\\
-1.89	7.67668843126116\\
-1.98	7.67948380545293\\
-2.07	7.68607473056825\\
-2.16	7.68910931877904\\
-2.25	7.69342002852294\\
-2.34	7.69622079533598\\
-2.43	7.7000474460873\\
-2.52	7.7026794103854\\
-2.61	7.7100048002551\\
-2.7	7.7138267989377\\
-2.79	7.71852041548378\\
-2.88	7.72500579189372\\
-2.97	7.73248170116962\\
-3.06	7.7381553230578\\
-3.07237271187583	7.74\\
-3.06983393833575	7.83\\
-3.06	7.83069969225748\\
-2.97	7.83626749453061\\
-2.88	7.83749992657547\\
-2.79	7.846014373291\\
-2.7	7.8513798795262\\
-2.61	7.85491912916901\\
-2.52	7.85866733782398\\
-2.43	7.86271223290541\\
-2.34	7.86581532108156\\
-2.25	7.86944726515862\\
-2.16	7.87272666750543\\
-2.07	7.87546867206209\\
-1.98	7.8776260428049\\
-1.89	7.88021498085567\\
-1.8	7.88456959053903\\
-1.71	7.8862438332113\\
-1.62	7.89025937221837\\
-1.53	7.89119818873131\\
-1.44	7.89080963819095\\
-1.35	7.89362418818851\\
-1.26	7.8952933764434\\
-1.17	7.89823856063538\\
-1.08	7.89907485748059\\
-0.990000000000001	7.89680174210107\\
-0.899999999999999	7.9029280488611\\
-0.81	7.90207630393947\\
-0.720000000000001	7.90553495610695\\
-0.629999999999999	7.90637040512995\\
-0.54	7.90601255876149\\
-0.450000000000001	7.90738165631529\\
-0.359999999999999	7.90853496833152\\
-0.27	7.9077726346857\\
-0.180000000000001	7.90896870931194\\
-0.0899999999999992	7.90805470774096\\
0	7.90914788282218\\
0.0900000000000004	7.90765010090296\\
0.18	7.90977281874953\\
0.27	7.90737104185525\\
0.36	7.90564935660714\\
0.45	7.90770023209256\\
0.54	7.9064705836388\\
0.630000000000001	7.9021175131347\\
0.72	7.90474987350304\\
0.809999999999999	7.90367522421297\\
0.900000000000001	7.9001110044559\\
0.99	7.89981678050385\\
1.08	7.89741242809846\\
1.17	7.89771911276927\\
1.26	7.89465914533371\\
1.35	7.89579689159161\\
1.44	7.89271115924535\\
1.53	7.88713711589768\\
1.62	7.88500563394791\\
1.71	7.88431563849968\\
1.8	7.8824143066868\\
1.89	7.88155702625476\\
1.98	7.88070206741322\\
2.07	7.87749033380783\\
2.16	7.87346436369078\\
2.25	7.87057273297642\\
2.34	7.86706499546692\\
2.43	7.86418470269247\\
2.52	7.85898530294209\\
2.61	7.85492004071353\\
2.7	7.85054822751362\\
2.79	7.84611831233404\\
2.88	7.84139170148282\\
2.97	7.8365213740711\\
3.06	7.83032915232239\\
3.06181947796548	7.83\\
3.10761236136571	7.74\\
0	147\\
3.17936558407341	8.37\\
3.15	8.36940826613322\\
3.06	8.36157192307823\\
2.97	8.35438179450588\\
2.88	8.35048063639982\\
2.79	8.34764277380629\\
2.7	8.34469082774623\\
2.61	8.34311909435637\\
2.52	8.33843431163749\\
2.43	8.33614618827819\\
2.34	8.33370644811692\\
2.25	8.3299254668318\\
2.16	8.32756875964508\\
2.07	8.32547742195443\\
1.98	8.32260571476926\\
1.89	8.32081838079408\\
1.8	8.31802613497885\\
1.71	8.31632668141613\\
1.62	8.31475487888544\\
1.53	8.31386938029272\\
1.44	8.31178191704415\\
1.35	8.31056770050826\\
1.26	8.30947409758745\\
1.17	8.3084093060827\\
1.08	8.30649898125688\\
0.99	8.30611937566176\\
0.900000000000001	8.3049461066706\\
0.809999999999999	8.30415093808613\\
0.72	8.30350822163628\\
0.630000000000001	8.30262971976052\\
0.54	8.30231017943653\\
0.45	8.30033254485871\\
0.36	8.30007302912771\\
0.27	8.29977904481463\\
0.18	8.29973535518548\\
0.0900000000000004	8.29982631369221\\
0	8.29903084076386\\
-0.0899999999999992	8.29929058456211\\
-0.180000000000001	8.30025431745743\\
-0.27	8.30023611253485\\
-0.359999999999999	8.30073466477433\\
-0.450000000000001	8.3006738983193\\
-0.54	8.30169034243728\\
-0.629999999999999	8.3022529857417\\
-0.720000000000001	8.30374441901838\\
-0.81	8.30381622629396\\
-0.899999999999999	8.30550695412294\\
-0.990000000000001	8.30585598794645\\
-1.08	8.30709696781149\\
-1.17	8.30832133263437\\
-1.26	8.30936683767829\\
-1.35	8.31031666582778\\
-1.44	8.3123626974732\\
-1.53	8.31336528116271\\
-1.62	8.31455603947862\\
-1.71	8.31636968124536\\
-1.8	8.3191685815576\\
-1.89	8.32048462465376\\
-1.98	8.32206013366964\\
-2.07	8.32635633829549\\
-2.16	8.32732688620566\\
-2.25	8.33045790537036\\
-2.34	8.33208965252036\\
-2.43	8.33556799018105\\
-2.52	8.3376091883592\\
-2.61	8.34086903658632\\
-2.7	8.34613431334261\\
-2.79	8.3489335066317\\
-2.88	8.35060395494328\\
-2.97	8.35475377946891\\
-3.06	8.36086847686457\\
-3.15	8.364697502627\\
-3.24	8.36947711889683\\
-3.24801569439853	8.37\\
-3.24	8.37054475033046\\
-3.15	8.37547178746147\\
-3.06	8.37927036043157\\
-2.97	8.38545980408351\\
-2.88	8.38892653819933\\
-2.79	8.39155762133659\\
-2.7	8.39511112428608\\
-2.61	8.39846781588563\\
-2.52	8.4018443347701\\
-2.43	8.40356477319405\\
-2.34	8.40731999026723\\
-2.25	8.40823642418855\\
-2.16	8.4122044676013\\
-2.07	8.41370122407239\\
-1.98	8.41828250307768\\
-1.89	8.41955780166694\\
-1.8	8.42092867696163\\
-1.71	8.42190174209208\\
-1.62	8.42362514019882\\
-1.53	8.42630638701143\\
-1.44	8.42729619856515\\
-1.35	8.42819197254833\\
-1.26	8.42941332845235\\
-1.17	8.4304230973013\\
-1.08	8.43196521122143\\
-0.990000000000001	8.43247065998328\\
-0.899999999999999	8.43313554904947\\
-0.81	8.43505777564816\\
-0.720000000000001	8.4352251902518\\
-0.629999999999999	8.43683101604354\\
-0.54	8.43587168731161\\
-0.450000000000001	8.43824329426632\\
-0.359999999999999	8.43775477640022\\
-0.27	8.43843234366928\\
-0.180000000000001	8.43908264154186\\
-0.0899999999999992	8.43919387919353\\
0	8.43986449084486\\
0.0900000000000004	8.43960187864003\\
0.18	8.43949391852646\\
0.27	8.43869622624282\\
0.36	8.43821367882511\\
0.45	8.43807293198473\\
0.54	8.43656393698453\\
0.630000000000001	8.43496517671995\\
0.72	8.43556907459595\\
0.809999999999999	8.43414220556827\\
0.900000000000001	8.43406755205663\\
0.99	8.43251820473151\\
1.08	8.43155393497168\\
1.17	8.43007689248962\\
1.26	8.43002375350674\\
1.35	8.42813353913956\\
1.44	8.42623646606817\\
1.53	8.425076724132\\
1.62	8.42389305799308\\
1.71	8.42243428117095\\
1.8	8.42074222525608\\
1.89	8.41935335866159\\
1.98	8.41728353734729\\
2.07	8.41439478616071\\
2.16	8.41120268926407\\
2.25	8.41027855943992\\
2.34	8.40664996971475\\
2.43	8.40449430793453\\
2.52	8.401916794397\\
2.61	8.3968830498554\\
2.7	8.39494122607348\\
2.79	8.39251716806728\\
2.88	8.38924202045925\\
2.97	8.38547957160116\\
3.06	8.37835345487279\\
3.15	8.37060626326869\\
3.17936558407341	8.37\\
0	141\\
3.07426555714403	8.64\\
3.06	8.63923450639513\\
2.97	8.63216682782248\\
2.88	8.63421743334994\\
2.79	8.62556626501118\\
2.7	8.6235571431827\\
2.61	8.61707021048699\\
2.52	8.61362106254729\\
2.43	8.60865966167238\\
2.34	8.60828498779337\\
2.25	8.60284209705292\\
2.16	8.60482277443981\\
2.07	8.5989016114701\\
1.98	8.59435387943036\\
1.89	8.59452566383855\\
1.8	8.59077792235494\\
1.71	8.58895575926021\\
1.62	8.58741730159524\\
1.53	8.5852805324477\\
1.44	8.58512178455029\\
1.35	8.58223371122879\\
1.26	8.58082722579396\\
1.17	8.57960707839395\\
1.08	8.58026606739753\\
0.99	8.57798653134293\\
0.900000000000001	8.57538928914114\\
0.809999999999999	8.57408325588864\\
0.72	8.57353314550217\\
0.630000000000001	8.57197141805182\\
0.54	8.57127641184236\\
0.45	8.56970604455632\\
0.36	8.56824388587211\\
0.27	8.56858636702875\\
0.18	8.56732570907492\\
0.0900000000000004	8.56684454480707\\
0	8.56546445785367\\
-0.0899999999999992	8.56646285663241\\
-0.180000000000001	8.56709183946633\\
-0.27	8.56746981481064\\
-0.359999999999999	8.56991327451088\\
-0.450000000000001	8.57001071090212\\
-0.54	8.57067086540781\\
-0.629999999999999	8.5716130527251\\
-0.720000000000001	8.57440707295866\\
-0.81	8.57448439365092\\
-0.899999999999999	8.57745312412157\\
-0.990000000000001	8.57651525348836\\
-1.08	8.57809235069011\\
-1.17	8.58081985374467\\
-1.26	8.58087854495096\\
-1.35	8.58324654957424\\
-1.44	8.58407267068153\\
-1.53	8.58593020594005\\
-1.62	8.58700416181801\\
-1.71	8.58956614583673\\
-1.8	8.59075145956048\\
-1.89	8.59264060021095\\
-1.98	8.59540136193555\\
-2.07	8.59777957492141\\
-2.16	8.60239257175855\\
-2.25	8.60489289181605\\
-2.34	8.60673734141756\\
-2.43	8.60980268858652\\
-2.52	8.61165624978025\\
-2.61	8.61797613032572\\
-2.7	8.61828506302034\\
-2.79	8.62385692768109\\
-2.88	8.62834277648234\\
-2.97	8.6318303013204\\
-3.06	8.63634635274039\\
-3.12680906562695	8.64\\
-3.06	8.64300290762108\\
-2.97	8.64655349686542\\
-2.88	8.64971563198658\\
-2.79	8.6528024924938\\
-2.7	8.65833936005057\\
-2.61	8.6581875693909\\
-2.52	8.66409498515088\\
-2.43	8.66649168260724\\
-2.34	8.6686569725382\\
-2.25	8.67106803274912\\
-2.16	8.67345285788212\\
-2.07	8.67822233549866\\
-1.98	8.67932768345981\\
-1.89	8.68140086912523\\
-1.8	8.68452251694527\\
-1.71	8.68628337481382\\
-1.62	8.68751480111048\\
-1.53	8.68978314680358\\
-1.44	8.69109896783158\\
-1.35	8.69216350582445\\
-1.26	8.69408034125561\\
-1.17	8.69606955366394\\
-1.08	8.69729970818427\\
-0.990000000000001	8.69902946079589\\
-0.899999999999999	8.70031671502652\\
-0.81	8.70205154757806\\
-0.720000000000001	8.70322759014744\\
-0.629999999999999	8.70485059127064\\
-0.54	8.70578454813465\\
-0.450000000000001	8.70681407819076\\
-0.359999999999999	8.70788384908268\\
-0.27	8.70955498342034\\
-0.180000000000001	8.70922749074691\\
-0.0899999999999992	8.71079039911935\\
0	8.7114961012132\\
0.0900000000000004	8.71093583602135\\
0.18	8.71031315112966\\
0.27	8.70904859208239\\
0.36	8.7066516458758\\
0.45	8.70732299383658\\
0.54	8.70470942397028\\
0.630000000000001	8.70414612299936\\
0.72	8.7026018977698\\
0.809999999999999	8.70062962245798\\
0.900000000000001	8.7008381676931\\
0.99	8.69904787873519\\
1.08	8.69652545691872\\
1.17	8.69637848872214\\
1.26	8.69394985998853\\
1.35	8.69215876116337\\
1.44	8.6905021541847\\
1.53	8.6892591079503\\
1.62	8.68749231078421\\
1.71	8.68414141950193\\
1.8	8.68262524094216\\
1.89	8.68236385803139\\
1.98	8.67979106752923\\
2.07	8.67616404131609\\
2.16	8.67042154774278\\
2.25	8.67216336459559\\
2.34	8.66828518816054\\
2.43	8.66738148230337\\
2.52	8.66221117184926\\
2.61	8.65909473619361\\
2.7	8.65320884150843\\
2.79	8.65051082491568\\
2.88	8.64470940883\\
2.97	8.64616748467725\\
3.06	8.6406260131435\\
3.07426555714403	8.64\\
1	209\\
9	-8.67308179917197\\
8.91	-8.69316744402855\\
8.83766218910404	-8.73\\
8.82	-8.73605337201833\\
8.73	-8.75351258384619\\
8.64	-8.77989843594866\\
8.55	-8.79402892149892\\
8.46	-8.81906728368691\\
8.45327475020487	-8.82\\
8.37	-8.82866789876247\\
8.28	-8.83671303696613\\
8.19	-8.84930635638013\\
8.1	-8.85466066878995\\
8.01	-8.85828691923017\\
7.92	-8.86529137226101\\
7.83	-8.87912978106364\\
7.74	-8.88145494737243\\
7.65	-8.8850724626466\\
7.56	-8.88878642366554\\
7.47	-8.90018539016858\\
7.38	-8.90413429370637\\
7.29	-8.90897660738112\\
7.25257955103822	-8.91\\
7.2	-8.91145420301484\\
7.1588647867009	-8.91\\
7.11	-8.90845154668253\\
7.08387829803686	-8.91\\
7.02	-8.91438408374901\\
6.93	-8.91891049681947\\
6.84	-8.92136547326968\\
6.75	-8.92501819928007\\
6.66	-8.92889438756821\\
6.57	-8.92912052811275\\
6.48	-8.93440291120731\\
6.39	-8.93299606091538\\
6.3	-8.93559443588082\\
6.21	-8.94165646703837\\
6.12	-8.93713301399942\\
6.03	-8.93950157409456\\
5.94	-8.94124159989049\\
5.85	-8.94056302047034\\
5.76	-8.94373069408716\\
5.67	-8.94541862030836\\
5.58	-8.94206013676725\\
5.49	-8.94672383089812\\
5.4	-8.94614490757321\\
5.31	-8.94898232515965\\
5.22	-8.95037206241272\\
5.13	-8.95063211727038\\
5.04	-8.95256180571761\\
4.95	-8.9541908843883\\
4.86	-8.95390927416299\\
4.77	-8.95132856197865\\
4.68	-8.95447906170418\\
4.59	-8.95602771094129\\
4.5	-8.95461743045031\\
4.41	-8.95787162989878\\
4.32	-8.95831095076348\\
4.23	-8.95447719005646\\
4.14	-8.95848353002759\\
4.05	-8.95671569818812\\
3.96	-8.95522637777054\\
3.87	-8.95943181448927\\
3.78	-8.95647405331057\\
3.69	-8.96097499722536\\
3.6	-8.96174883711891\\
3.51	-8.95941164761378\\
3.42	-8.95839905342422\\
3.33	-8.96256121926205\\
3.24	-8.95633536860735\\
3.15	-8.95704450468433\\
3.06	-8.96252551429377\\
2.97	-8.96013173831493\\
2.88	-8.96173348176129\\
2.79	-8.95878106791691\\
2.7	-8.9584824282255\\
2.61	-8.95895531977279\\
2.52	-8.9618238218256\\
2.43	-8.96376538567302\\
2.34	-8.9610424887232\\
2.25	-8.95868505539735\\
2.16	-8.96246011324746\\
2.07	-8.95915804596601\\
1.98	-8.96336339679328\\
1.89	-8.96065990673688\\
1.8	-8.95932668115548\\
1.71	-8.96264804083361\\
1.62	-8.96262177967491\\
1.53	-8.96154102719861\\
1.44	-8.96232977688226\\
1.35	-8.96026873956257\\
1.26	-8.96210172283176\\
1.17	-8.96243662085709\\
1.08	-8.96188162564506\\
0.99	-8.96045056792061\\
0.900000000000001	-8.95926421957248\\
0.809999999999999	-8.9583530094415\\
0.72	-8.96392132852773\\
0.630000000000001	-8.96350467221548\\
0.54	-8.96211190486457\\
0.45	-8.96423494286682\\
0.36	-8.96124167326986\\
0.27	-8.96188027141624\\
0.18	-8.96455179872673\\
0.0900000000000004	-8.96345870694823\\
0	-8.96205957956115\\
-0.0899999999999992	-8.96255660931664\\
-0.180000000000001	-8.96147363896841\\
-0.27	-8.96497789571923\\
-0.359999999999999	-8.9605628839238\\
-0.450000000000001	-8.96255512134548\\
-0.54	-8.96283921886338\\
-0.629999999999999	-8.96059396298324\\
-0.720000000000001	-8.96075106523875\\
-0.81	-8.96235404847172\\
-0.899999999999999	-8.96305319015977\\
-0.990000000000001	-8.96064285572965\\
-1.08	-8.96051981068215\\
-1.17	-8.96301553239428\\
-1.26	-8.96218461163912\\
-1.35	-8.95969259503294\\
-1.44	-8.95897834026128\\
-1.53	-8.96324448809111\\
-1.62	-8.96236586714752\\
-1.71	-8.96078186467434\\
-1.8	-8.95991248323872\\
-1.89	-8.96142620678471\\
-1.98	-8.96030998663422\\
-2.07	-8.96058197799263\\
-2.16	-8.95918163966758\\
-2.25	-8.96238142362193\\
-2.34	-8.96090261909157\\
-2.43	-8.95900900271498\\
-2.52	-8.95928177011945\\
-2.61	-8.96148886214162\\
-2.7	-8.95901763894769\\
-2.79	-8.95923745692281\\
-2.88	-8.96016832690533\\
-2.97	-8.95772214639477\\
-3.06	-8.95838437463896\\
-3.15	-8.96167462163887\\
-3.24	-8.95956724850224\\
-3.33	-8.95735931192522\\
-3.42	-8.95866972486603\\
-3.51	-8.96147924638595\\
-3.6	-8.96043666464424\\
-3.69	-8.95773195298746\\
-3.78	-8.95517536313924\\
-3.87	-8.95841829736889\\
-3.96	-8.95383723051037\\
-4.05	-8.95469529379672\\
-4.14	-8.95281302618451\\
-4.23	-8.95669196284747\\
-4.32	-8.95528407319284\\
-4.41	-8.95636783260734\\
-4.5	-8.9550960530899\\
-4.59	-8.95241695767293\\
-4.68	-8.95421101584158\\
-4.77	-8.95608777842344\\
-4.86	-8.95456207541785\\
-4.95	-8.95260059617358\\
-5.04	-8.95131965969894\\
-5.13	-8.95374818580919\\
-5.22	-8.95005379865559\\
-5.31	-8.95128889995684\\
-5.4	-8.94867727765898\\
-5.49	-8.94853486113534\\
-5.58	-8.94399541670674\\
-5.67	-8.94924385423664\\
-5.76	-8.94466689981959\\
-5.85	-8.94391841129613\\
-5.94	-8.94440960336517\\
-6.03	-8.93825237430833\\
-6.12	-8.93720858292103\\
-6.21	-8.93679149789102\\
-6.3	-8.93278154999837\\
-6.39	-8.93601559517191\\
-6.48	-8.92566991006189\\
-6.57	-8.9280639277103\\
-6.66	-8.92694495120228\\
-6.75	-8.92540088272599\\
-6.84	-8.92335812364439\\
-6.93	-8.91930228031538\\
-7.02	-8.92193145765649\\
-7.11	-8.91123186165311\\
-7.1352904465368	-8.91\\
-7.2	-8.90726044617367\\
-7.29	-8.90397801383579\\
-7.38	-8.90305798100013\\
-7.47	-8.89924433890208\\
-7.56	-8.8946288152959\\
-7.65	-8.88532270742543\\
-7.74	-8.8795432804847\\
-7.83	-8.87596912154214\\
-7.92	-8.86949644706477\\
-8.01	-8.86377242865187\\
-8.1	-8.85212549414229\\
-8.19	-8.84320148710767\\
-8.28	-8.83648357768513\\
-8.36506237698077	-8.82\\
-8.37	-8.81895273632186\\
-8.46	-8.81444544502332\\
-8.55	-8.79373361121134\\
-8.64	-8.78147694563643\\
-8.73	-8.76179598002076\\
-8.82	-8.74181705920236\\
-8.86303582049368	-8.73\\
-8.91	-8.71012012082385\\
-9	-8.66490622183365\\
1	209\\
-9	8.66490622183365\\
-8.91	8.71012012082385\\
-8.86303582049368	8.73\\
-8.82	8.74181705920236\\
-8.73	8.76179598002076\\
-8.64	8.78147694563643\\
-8.55	8.79373361121134\\
-8.46	8.81444544502332\\
-8.37	8.81895273632187\\
-8.36506237698077	8.82\\
-8.28	8.83648357768513\\
-8.19	8.84320148710768\\
-8.1	8.85212549414229\\
-8.01	8.86377242865187\\
-7.92	8.86949644706477\\
-7.83	8.87596912154214\\
-7.74	8.8795432804847\\
-7.65	8.88532270742543\\
-7.56	8.89462881529591\\
-7.47	8.89924433890208\\
-7.38	8.90305798100013\\
-7.29	8.90397801383579\\
-7.2	8.90726044617367\\
-7.1352904465368	8.91\\
-7.11	8.91123186165311\\
-7.02	8.92193145765649\\
-6.93	8.91930228031538\\
-6.84	8.92335812364439\\
-6.75	8.92540088272599\\
-6.66	8.92694495120228\\
-6.57	8.9280639277103\\
-6.48	8.92566991006189\\
-6.39	8.93601559517191\\
-6.3	8.93278154999837\\
-6.21	8.93679149789102\\
-6.12	8.93720858292103\\
-6.03	8.93825237430833\\
-5.94	8.94440960336517\\
-5.85	8.94391841129613\\
-5.76	8.94466689981959\\
-5.67	8.94924385423664\\
-5.58	8.94399541670674\\
-5.49	8.94853486113534\\
-5.4	8.94867727765898\\
-5.31	8.95128889995684\\
-5.22	8.95005379865559\\
-5.13	8.95374818580919\\
-5.04	8.95131965969894\\
-4.95	8.95260059617358\\
-4.86	8.95456207541785\\
-4.77	8.95608777842344\\
-4.68	8.95421101584158\\
-4.59	8.95241695767293\\
-4.5	8.9550960530899\\
-4.41	8.95636783260734\\
-4.32	8.95528407319284\\
-4.23	8.95669196284747\\
-4.14	8.95281302618451\\
-4.05	8.95469529379672\\
-3.96	8.95383723051037\\
-3.87	8.95841829736889\\
-3.78	8.95517536313924\\
-3.69	8.95773195298746\\
-3.6	8.96043666464424\\
-3.51	8.96147924638595\\
-3.42	8.95866972486603\\
-3.33	8.95735931192522\\
-3.24	8.95956724850224\\
-3.15	8.96167462163887\\
-3.06	8.95838437463896\\
-2.97	8.95772214639477\\
-2.88	8.96016832690533\\
-2.79	8.95923745692281\\
-2.7	8.95901763894769\\
-2.61	8.96148886214162\\
-2.52	8.95928177011945\\
-2.43	8.95900900271498\\
-2.34	8.96090261909157\\
-2.25	8.96238142362193\\
-2.16	8.95918163966758\\
-2.07	8.96058197799263\\
-1.98	8.96030998663422\\
-1.89	8.96142620678471\\
-1.8	8.95991248323872\\
-1.71	8.96078186467434\\
-1.62	8.96236586714752\\
-1.53	8.96324448809111\\
-1.44	8.95897834026128\\
-1.35	8.95969259503294\\
-1.26	8.96218461163912\\
-1.17	8.96301553239428\\
-1.08	8.96051981068215\\
-0.990000000000001	8.96064285572965\\
-0.899999999999999	8.96305319015977\\
-0.81	8.96235404847172\\
-0.720000000000001	8.96075106523875\\
-0.629999999999999	8.96059396298324\\
-0.54	8.96283921886338\\
-0.450000000000001	8.96255512134548\\
-0.359999999999999	8.9605628839238\\
-0.27	8.96497789571923\\
-0.180000000000001	8.96147363896841\\
-0.0899999999999992	8.96255660931664\\
0	8.96205957956115\\
0.0900000000000004	8.96345870694823\\
0.18	8.96455179872673\\
0.27	8.96188027141624\\
0.36	8.96124167326986\\
0.45	8.96423494286682\\
0.54	8.96211190486457\\
0.630000000000001	8.96350467221548\\
0.72	8.96392132852773\\
0.809999999999999	8.9583530094415\\
0.900000000000001	8.95926421957248\\
0.99	8.96045056792061\\
1.08	8.96188162564506\\
1.17	8.96243662085709\\
1.26	8.96210172283176\\
1.35	8.96026873956257\\
1.44	8.96232977688226\\
1.53	8.96154102719861\\
1.62	8.96262177967491\\
1.71	8.96264804083361\\
1.8	8.95932668115548\\
1.89	8.96065990673688\\
1.98	8.96336339679328\\
2.07	8.95915804596601\\
2.16	8.96246011324746\\
2.25	8.95868505539735\\
2.34	8.9610424887232\\
2.43	8.96376538567302\\
2.52	8.9618238218256\\
2.61	8.95895531977279\\
2.7	8.9584824282255\\
2.79	8.95878106791691\\
2.88	8.96173348176129\\
2.97	8.96013173831493\\
3.06	8.96252551429377\\
3.15	8.95704450468433\\
3.24	8.95633536860735\\
3.33	8.96256121926205\\
3.42	8.95839905342422\\
3.51	8.95941164761378\\
3.6	8.96174883711891\\
3.69	8.96097499722536\\
3.78	8.95647405331057\\
3.87	8.95943181448927\\
3.96	8.95522637777054\\
4.05	8.95671569818812\\
4.14	8.95848353002759\\
4.23	8.95447719005646\\
4.32	8.95831095076348\\
4.41	8.95787162989878\\
4.5	8.95461743045031\\
4.59	8.95602771094129\\
4.68	8.95447906170418\\
4.77	8.95132856197865\\
4.86	8.95390927416299\\
4.95	8.9541908843883\\
5.04	8.95256180571761\\
5.13	8.95063211727038\\
5.22	8.95037206241272\\
5.31	8.94898232515965\\
5.4	8.94614490757321\\
5.49	8.94672383089812\\
5.58	8.94206013676725\\
5.67	8.94541862030836\\
5.76	8.94373069408716\\
5.85	8.94056302047034\\
5.94	8.94124159989049\\
6.03	8.93950157409456\\
6.12	8.93713301399942\\
6.21	8.94165646703837\\
6.3	8.93559443588082\\
6.39	8.93299606091538\\
6.48	8.93440291120731\\
6.57	8.92912052811275\\
6.66	8.92889438756821\\
6.75	8.92501819928007\\
6.84	8.92136547326968\\
6.93	8.91891049681947\\
7.02	8.91438408374901\\
7.08387829803686	8.91\\
7.11	8.90845154668253\\
7.1588647867009	8.91\\
7.2	8.91145420301484\\
7.25257955103822	8.91\\
7.29	8.90897660738112\\
7.38	8.90413429370638\\
7.47	8.90018539016858\\
7.56	8.88878642366554\\
7.65	8.8850724626466\\
7.74	8.88145494737243\\
7.83	8.87912978106364\\
7.92	8.86529137226101\\
8.01	8.85828691923017\\
8.1	8.85466066878995\\
8.19	8.84930635638013\\
8.28	8.83671303696613\\
8.37	8.82866789876248\\
8.45327475020487	8.82\\
8.46	8.81906728368691\\
8.55	8.79402892149892\\
8.64	8.77989843594866\\
8.73	8.7535125838462\\
8.82	8.73605337201833\\
8.83766218910404	8.73\\
8.91	8.69316744402855\\
9	8.67308179917197\\
};
\end{axis}
\end{tikzpicture}%


\end{document}
